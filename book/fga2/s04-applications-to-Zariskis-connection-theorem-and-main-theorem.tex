% !TeX root = ../../fga.tex
\section{Applications to Zariski's connection theorem and "main theorem"}\label{fga2-4}


Let $f\colon X\to Y$ be a proper morphism of schemes.
Then, by the finiteness theorem (\Cref{fga2-theorem-1}), $f_*(\mathcal{O}_X)=\underline{A}$ is a coherent sheaf on $Y$, and is also a sheaf of commutative algebras, and thus corresponds to a $Y$-scheme $g\colon Y'\to Y$ that is finite over $Y$ (defined by the condition of being affine over $Y$, i.e. the inverse image of an affine open is affine, and $g_*(\mathcal{O}_{Y'})=\underline{A}$).
It is immediate that $f$ then canonically factors as $f=gf'$, where $f'\colon X\to Y'$ is a morphism from $X$ to $Y$ that is now such that $f'_*(\mathcal{O}_X)=\mathcal{O}_{Y'}$.
This factorisation of $f$ is called the \emph{Stein factorisation} of $f$.
Applying the first comparison theorem (\Cref{fga2-theorem-2}) and its Corollary 1 \Cref{fga2-theorem-2-corollary-1} to $f'$ and the subset $Y'$ consisting of a single point $y'$, we see that $(f')^{-1}(y')=X'$ is connected (or, in other words, the formal sections of $X$ along $X'$ do not form a local ring, but the completion $f'_*(\mathcal{O}_X)_{y'}=\mathcal{O}_{y'}$ is local!)
We have proven:

\begin{theorem}[Zariski's "connection theorem"]\label{fga2-theorem-5}
    Let $f\colon X\to Y$ be a proper morphism.
    Then $f$ factors uniquely (up to isomorphism) as $f=gf'$, where $g\colon Y'\to Y$ is finite, and $f'\colon X\to Y'$ is such that $f'_*(\mathcal{O}_X)=\mathcal{O}_{Y'}$ (whence $g_*(\mathcal{O}_{Y'})=f_*(\mathcal{O}_X)$).
    Also, the fibres of $f'$ are connected, i.e. the set of connected components of a fibre $f^{-1}(y)$ of $f$ is in bijective correspondence with the set of points of $Y'$ over $y$, i.e. the set of maximal ideals in $f_*(\mathcal{O}_X)_y$.
\end{theorem}

From this, we immediately deduce the usual variants of the connection theorem.
We state here only the following:

\begin{corollary}\label{fga2-theorem-5-corollary-1}
    For a point $x$ of $X$ to be isolated in its fibre $f^{-1}(y)$, it is necessary and sufficient that the fibre $(f')^{-1}(y')$ (where $y'=f'(x)$) consist of a single point $x$, or that $f'$ induce an isomorphism from a neighbourhood of $x$ to a neighbourhood of $y'$.
    The set of these points is an open subset $U$, and $f'$ induces an isomorphism from $U$ to an open subset of $Y'$.
\end{corollary}

To show that $f'$ is a local isomorphism at $x$, we note that $f'$ induces an \emph{isomorphism} $\mathcal{O}_{y'}\to\mathcal{O}_x$, as we see thanks to $f'(\mathcal{O}_X)=\mathcal{O}_{Y'}$;
we also note that the $(f')^{-1}(V)$ give a fundamental system of neighbourhoods of $x$ when $V$ runs over a fundamental system of neighbourhoods of $y'$ (since $f'$ is a closed map whose fibre at $y'$ consists of the single point $x$).
We thus immediately deduce the following result, due to Chevalley in the "geometric" case:

\begin{corollary}\label{fga2-theorem-5-corollary-2}
    For $f$ to be a finite morphism, it is necessary and sufficient that it be proper with finite fibres.
\end{corollary}

If the hypotheses of \Cref{fga2-theorem-5-corollary-2} hold, then $f'$ is effectively an isomorphism, by the above.

Let $f\colon X\to Y$ be a morphism that is not necessarily proper, but suppose that $X$ is contained in some proper $Y$-scheme $\overline{f}\colon\overline{X}\to Y$ as an open subset (which is the case if, in particular, $\overline{f}$ is quasi-projective).
Applying Corollary 1 \Cref{fga2-theorem-5-corollary-1}, we see that $\overline{f'}$ induces an isomorphism from the set $U$ of points of $X$ that are isolated in their fibre to an open subset of $Y'$ (and that $U$ is indeed an open subset).
We thus deduce the following global version of Zariski's "main theorem":

\begin{theorem}\label{fga2-theorem-6}
    Let $f\colon X\to Y$ be a morphism of finite type.
    Then the set $U$ of points of $X$ that are isolated in their fibre is open, and, if $f$ is quasi-projective,
    \footnote{\emph{[Comp.]} This hypothesis can be replaced by the weaker hypothesis "if $f$ is separated", by means of the following result (see \cite{Gro1960b}, VIII, 6.2): every morphism $f\colon X\to Y$ which is quasi-finite and separated is also projective.}
    then $U$ is $Y$-isomorphic to an open subset of some scheme $Y'$ that is finite over $Y$.
\end{theorem}

Since a morphism of finite type is locally affine, and \emph{a fortiori} locally quasi-projective, we immediately deduce from \Cref{fga2-theorem-6} the usual local variants of the main theorem.
