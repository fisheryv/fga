% !TeX root = ../../fga.tex
\section{The three fundamental theorems}\label{fga2-3}


Let $f\colon X\to Y$ be a proper morphism of schemes (Noetherian, as always), and let $Y'$ be a closed subset of $Y'$, with $X'$ its inverse image in $X$, and consider the corresponding formal completions $\overline{Y}$ and $\overline{X}$.
Then $f$ induces a morphism $\overline{f}\colon\overline{X}\to\overline{Y}$ of formal schemes, which is also proper.
Let $\mathcal{F}$ be a coherent sheaf on $X$, then $\overline{\mathcal{F}}$ is a coherent sheaf on $\overline{X}$.
In \Cref{fga2-theorem-1}, we forget $X$, $Y$, and $\mathcal{F}$, and consider only the proper morphism $\overline{f}$ of formal schemes, along with the coherent sheaf $\overline{\mathcal{F}}$ on $\overline{X}$.
(However, the speaker has only written the complete proof in the case where we start with some $X$, $Y$, $f$, and $F$).

\begin{theorem}[Finiteness theorem]\label{fga2-theorem-1}
    \begin{enumerate}[i.]
        \item The $\operatorname{R}^q\overline{f}_*(\overline{\mathcal{F}})$ are coherent sheaves on $\overline{Y}$.
        \item The natural homomorphisms
              \[\operatorname{R}^q\overline{f}_*(\overline{\mathcal{F}}) \to \varprojlim_n\operatorname{R}^q (f_n)_*(\mathcal{F}_n)\]
              are isomorphisms.
    \end{enumerate}
\end{theorem}


In the statement of \Cref{fga2-theorem-1}, we suppose that we already have some coherent subsheaf $\mathcal{J}$ of $\mathcal{O}_Y$ that defines $Y'$, whence, by taking the inverse image, a coherent subsheaf of $\mathcal{O}_X$ that defines $X'$, whence, by definition, $\mathcal{F}_n$, $X_n$, $Y_n$, and $f_n\colon X_n\to Y_n$ as in \Cref{fga2-2}.
The minor changes that need to be made to the notation in the explanation if we started with an arbitrary proper morphism between two formal schemes are evident.

\Cref{fga2-theorem-1} deals only with "formal cohomology";
the following theorem relates it with "algebraic cohomology", and resembles a well-known theorem of Serre \cite{Ser1956} on the comparison between algebraic cohomology and analytic cohomology.

\begin{theorem}[First comparison theorem]\label{fga2-theorem-2}
    The $\operatorname{R}^q f_*(\mathcal{F})$ are coherent sheaves on $Y$ (which is a particular case of \Cref{fga2-theorem-1}), and the natural homomorphisms
    \[\overline{\operatorname{R}^q f_*(\mathcal{F})} \to \varprojlim_n \operatorname{R}^q (f_n)_*(\mathcal{F}_n)\]
    are isomorphisms.
\end{theorem}

\begin{corollary}\label{fga2-theorem-2-corollary-1}
    There are canonical isomorphisms $\overline{\operatorname{R}^q f_*(\mathcal{F})} = \operatorname{R}^q\overline{f}_*(\overline{\mathcal{F}})$.
\end{corollary}


This corollary is, for $q=0$, a generalisation of Zariski's "fundamental theorem of holomorphic functions", from which we will deduce a generalisation of Zariski's "connection theorem".
We also note that, while (ii) in \Cref{fga2-theorem-1} is trivial for $q=0$, this is not at all the case for \Cref{fga2-theorem-2} nor for its equivalent formulation (\Cref{fga2-theorem-2-corollary-1}).
In fact, the proof proceeds by decreasing induction on $q$ (being trivial for large $q$, since then both sides of the equation are zero), and the case $q=0$ thus appears as the last induction step, and so could be called the "most difficult" case.

\begin{corollary}\label{fga2-theorem-2-corollary-2}
    Let $Y=\operatorname{Spec}(A)$, and let $Y'$ be defined by an ideal $\mathcal{J}$ of $A$.
    Then, for every coherent sheaf $\mathcal{F}$ on $X$, the $\operatorname{H}^q(X,\mathcal{F})$ are $A$-modules of finite type, whose $\mathcal{J}$-adic completions are the $\operatorname{H}^q(\overline{X},\overline{\mathcal{F}})$.
\end{corollary}

Finally, applying this corollary to $H=\mathcal{H}om_{\mathcal{O}_X}(\mathcal{F},\mathcal{G})$, we obtain:

\begin{corollary}\label{fga2-theorem-2-corollary-3}
    Let $Y=\operatorname{Spec}(A)$, and let $Y'$ be defined by an ideal $\mathcal{J}$ of $A$.
    Let $\mathcal{F}$ and $\mathcal{G}$ be coherent sheaves on $X$.
    Then $\operatorname{Hom}(\mathcal{F},\mathcal{G})$ is an $A$-module of finite type, whose $\mathcal{J}$-adic completion can be identified with $\operatorname{Hom}(\overline{\mathcal{F}},\overline{\mathcal{G}})$.
\end{corollary}

Of course, the natural map $\operatorname{Hom}(\mathcal{F},\mathcal{G})\to\operatorname{Hom}(\overline{\mathcal{F}},\overline{\mathcal{G}})$ is that which sends a homomorphism $u\colon\mathcal{F}\to\mathcal{G}$ to its extension "by continuity" $\overline{u}\colon\overline{\mathcal{F}}\to\overline{\mathcal{G}}$ (so that $\overline{\mathcal{F}}$ becomes a functor in $\mathcal{F}$).

Now suppose that $A$ is separated and complete for its $\mathcal{J}$-adic topology.
Then \Cref{fga2-theorem-2-corollary-2} and \Cref{fga2-theorem-2-corollary-3} above give:
\[
    \begin{aligned}
        \operatorname{H}^q(X,\mathcal{F}) & = \operatorname{H}^q(\overline{X},\overline{\mathcal{F}}),
        \\\operatorname{Hom}(\mathcal{F},\mathcal{G}) &= \operatorname{Hom}(\overline{\mathcal{F}},\overline{\mathcal{G}}).
    \end{aligned}
\]

The latter identity shows that the category of coherent sheaves on $X$ can be identified with a \emph{subcategory} (with morphisms being the induced morphisms) of the category of coherent sheaves on $\overline{X}$.
In fact, we even have:

\begin{theorem}\label{fga2-theorem-3}
    For a sheaf of modules on $\overline{X}$ to be coherent, it is necessary and sufficient that it be isomorphic to a sheaf of the form $\overline{\mathcal{F}}$, where $\mathcal{F}$ is a coherent sheaf on $X$ (determined up to canonical isomorphism, by \Cref{fga2-theorem-2-corollary-3} above).
    (We recall that now $Y=\operatorname{Spec}(A)$, with $A$ being a complete and separated $\mathcal{J}$-adic Noetherian topological ring).
\end{theorem}

\begin{corollary}\label{fga2-theorem-3-corollary-1}
    The closed subschemes of $X$ are in bijective correspondence with the closed formal subschemes of $\overline{X}$.
\end{corollary}


Indeed, they correspond to coherent subsheaves of $\mathcal{O}_X$ (resp. of $\mathcal{O}_{\overline{X}}$).
Considering the graphs of morphisms as closed subschemes, \Cref{fga2-theorem-3-corollary-1} above implies:

\begin{corollary}\label{fga2-theorem-3-corollary-2}
    Let $X$ and $Z$ be proper schemes over $A$ (which is a separated complete $\mathcal{J}$-adic Noetherian ring).
    Then the map $g\mapsto\overline{g}$ defines a bijective correspondence between $Y$-morphism from $X$ to $Z$ and $\overline{Y}$-morphisms from $\overline{X}$ to $\overline{Z}$.
\end{corollary}

In other words, proper algebraic schemes over $A$ give a subcategory (with the morphisms being the induced morphisms) of the category of proper formal schemes over $\overline{Y}$.
We note, however, that \emph{there exist proper formal schemes over $\overline{Y}$ that are not "algebraisable"}, i.e. isomorphic to some $\overline{X}$, where $X$ is proper over $A$ (just as there exist compact complex-analytic varieties that do not come from algebraic varieties defined over the field of complex numbers).
Such formal schemes naturally appear in "module theory".
We note, however, an interesting case where a formal scheme is algebraisable:

\begin{theorem}\label{fga2-theorem-4}
    Let $A$ be a complete local Noetherian ring, with residue field $k$, and let $\mathfrak{X}$ be a proper formal scheme over $A$ (endowed with its $\mathfrak{r}(A)$-adic topology).
    We suppose that
    \begin{enumerate}[i.]
        \item the local rings of $\mathcal{O}_{\mathfrak{X}}$ are \emph{flat} $A$-modules, or, equivalently, that, if we endow $\mathcal{O}_{\mathfrak{X}}$ and $A$ with the filtration given by powers of the maximal ideal of $A$, then the associated graded algebras satisfy
              \[\operatorname{gr}(\mathcal{O}_{\mathfrak{X}}) \simeq \operatorname{gr}^0(\mathcal{O}_{\mathfrak{X}})\otimes_k\operatorname{gr}(A)\]
        \item $\operatorname{H}^2(\mathfrak{X}_0,\mathcal{O}_{\mathfrak{X}_0})=0$, where we consider $\mathfrak{X}_0=\mathfrak{X}\otimes_Ak$ as an algebraic scheme over $k$;
        \item $\mathfrak{X}_0$ is projective.
    \end{enumerate}
    Then, under these conditions, $\mathfrak{X}$ is algebraisable, and, more precisely, is isomorphic to $\overline{X}$, where $X$ is some projective $A$-scheme.
\end{theorem}


Conditions (ii) and (iii) in \Cref{fga2-theorem-4} will be satisfied if, in particular, $\mathfrak{X}_0$ is a \emph{simple curve} over $k$, and \Cref{fga2-theorem-4} can be applied, in particular, in the "module theory" of curves of a given genus...
We will give here a hint on how to prove \Cref{fga2-theorem-4}:
we can show (cf. \Cref{fga2-proposition-3} below) that (i) and (ii) imply that every coherent sheaf on $\mathfrak{X}_0$ that is locally isomorphic to a fundamental sheaf can be obtained by reduction starting from a sheaf of the same nature on $\mathfrak{X}$.
So, starting with an "ample" sheaf on $\mathfrak{X}_0$ (which, by (iii), exists), we lift it to obtain an invertible sheaf on $\mathfrak{X}$, and, using \Cref{fga2-theorem-1}, we prove that a multiple of this invertible sheaf defined an immersion of $X$ into the formal completion of a scheme $\mathbb{P}_A^r$ ("projective type" of dimension $r$ over $A$).

For the proof of \Cref{fga2-theorem-1}, \Cref{fga2-theorem-2}, and \Cref{fga2-theorem-3}, we refer the reader to \cite{GD1960}, I.
