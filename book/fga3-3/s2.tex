% !TeX root = ../../fga.tex
\section{Example: finite preschemes over $S$}\label{fga3.iii-2}


Let $\mathcal{C}$ be the category of finite preschemes over $S$, which is assumed to be locally Noetherian.
Then $\mathcal{C}$ is equivalent to the opposite category of the category of coherent sheaves of commutative algebras on $S$, or, if $S$ is affine of ring $A$, then it is equivalent to the opposite category of the category of finite $A$-algebras over $A$ (i.e. those that are modules of finite type over $A$).
We thus immediately conclude that, in $\mathcal{C}$, finite projective limits and finite inductive limits exist.
This is well known (without any finiteness hypotheses) for the former;
the fibre product of preschemes $X$ and $Y$ over $S$ corresponds to the tensor product $B\otimes_A C$ of corresponding algebras, and the kernel of two morphisms $X\rightrightarrows Y$, defined by two $A$-algebra homomorphisms $u,v\colon C\rightrightarrows B$, corresponds to the quotient of $B$ by the ideal generated by the $u(v)-v(c)$, etc.
For finite inductive limits, it suffices to consider, on one hand, finite sums, which correspond to the ordinary product of $A$-algebras, and, on the other hand, cokernels of pairs of morphisms $X\rightrightarrows Y$, which correspond (as we can immediately see) to the sub-ring of $C$ given by elements where the homomorphisms $u,v\colon C\rightrightarrows B$ agree (with this sub-ring being finite over $A$ thanks to the Noetherian hypothesis).
We also note that we can show, using the Noetherian hypothesis, that finite inductive limits, and, in particular, quotients, thus constructed in the category $\mathcal{C}$ of finite preschemes over $S$ are, in fact, quotients in the category of \emph{all} preschemes.

As we mentioned in \Cref{fga3.i}, \emph{there are non-effective epimorphisms in $\mathcal{C}$} (or even non-strict, which is the same, since fibre products exist).
\emph{I do not know if equivalence relations are still effective} if we have no flatness hypothesis.
I have only obtained, in this direction, very partial, positive, results, that are vital for the proof of the fundamental theorem of the formal theory of modules (cf. FGA 3.II, §B, Theorem 1 \Cref{fga3.ii-b-theorem-1}).
We note that it is easy, in the given problem, to reduce to the case where $S$ is the spectrum of a local Artinian ring, with an algebraically closed residue field.
But even if $A$ is a field, the answer is not known.


We can also consider the case of a prescheme $X$ over $S$ that is no longer assumed to be finite over $S$, but by considering an equivalence relation $R$ on $X$ such that $p_1\colon R\to X$ is a finite morphism.
We then say that $R$ is a \emph{finite equivalence relation}.
Supposing, for simplicity, that $S$ and $X$ are affine (which implies that $R$ is affine, so that the situation is reduced to one of pure commutative algebra), \emph{we do not know, even in this case, if there exists a quotient $X/R=Y$, and if the canonical morphism $X\to Y$ is finite}.
(The most simple case is that where we suppose that $S$ is the spectrum of a field $k$, and where $X$ is the spectrum of $k[t]$, i.e. the affine line).
Of course, if the two problems above turn out to be true, then we can conclude that, in the situation described, $R$ is effective.
Note that the problem of \emph{existence} of a quotient $Y$ and of the \emph{finiteness} of $f\colon X\to Y$ are stated in exactly the same terms if, instead of an equivalence graph in $X$, we only have an equivalence pregraph in $X$, in the sense of \Cref{fga3.iii-4}.



The question of passing to the quotient by a more or less arbitrary finite equivalence relation arises in the construction of preschemes by "gluing" given preschemes $X_i$ along certain closed sub-preschemes;
the gluing law is expressed precisely by a finite equivalence relation on the prescheme $X$ given by the sum of the $X_i$.
We also expect that the solutions of the problems stated here, as well as of their many variations, will be a preliminary condition for the clarification of a general technique for non-projective constructions, in the direction introduced in \Cref{fga3.ii}.


The only general positive fact known to the author is the following:

\begin{proposition}\label{fga3.iii-2-proposition-2.1}
    Let $S$ be a locally Noetherian prescheme, $s$ a point of $S$, and $\Omega$ an algebraically closed extension of $k(s)$.
    Consider the corresponding "fibre functor" $F$, that associates, to any $S$-scheme $X$ that is finite over $S$, the set of points of $X/S$ with values in $\Omega$.
    This functor (which is trivially left exact) is \emph{right exact}, i.e. it commutes with finite inductive limits, and, in particular, with the cokernel of pairs of morphisms.
\end{proposition}


By using this result for all the "geometric points" of $S$, we thus deduce that the "quotient" category $\mathcal{C}'$ of $\mathcal{C}$, given by arguing "modulo surjective radicial morphisms" (i.e. by formally adjoining inverses for such morphisms), is a "geometric" category, i.e. it satisfies the same "finite nature" properties as the category of sets.
In particular, every equivalence relation is effective.
This implies that, if $R$ is an equivalence relation on $X$, where $X$ is finite over $S$, then the canonical morphism $R\to X\times_Y X$ (where $Y=X/R$) is \emph{radicial and surjective} (and, in fact, a surjective closed immersion, since it is a monomorphism).
