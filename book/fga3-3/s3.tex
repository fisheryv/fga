% !TeX root = ../../fga.tex
\section{The case of a group with operators}\label{fga3.iii-3}

We now suppose that $\mathcal{C}$ is an arbitrary category.
Let $G$ and $X$ be objects of $\mathcal{C}$, and suppose that $G$ is a $\mathcal{C}$-group with operators on the object $X$.
This implies (cf. FGA 3.II, §A.1 \Cref{fga3.ii-a.1}) that, for every object $T$ of $\mathcal{C}$, we have a group structure on $G(T)$, and the structure of an operator domain on $X(T)$ acting on $G(T)$, such that, for variable $T$, the structures in question "vary functorially" in $T$.
If the products $G\times G$ and $G\times X$ exist in $\mathcal{C}$, then such a structure can also be defined as a pair of morphisms
\[
    \begin{gathered}
        G\times G\to G \\
        \pi\colon G\times X\to X
    \end{gathered}
\]
subject to the condition that, for every object $T$ of $\mathcal{C}$, the corresponding composition laws for the sets $G(T)$ and $X(T)$ make $G(T)$ into a group acting on $X(T)$.
Translating this axiom into the commutativity of certain diagrams in $\mathcal{C}$ is easy, but tedious, and, in fact, perfectly useless in all cases known to me.


Suppose that $G\times X$ exists, and consider the two morphisms
\[
    p_1,p_2\colon G\times X\rightrightarrows X
\]
with
\[
    \begin{aligned}
        p_1 & = \mathrm{pr}_1\\
        p_2 &= \pi
    \end{aligned}
\]
We immediately note that the pair $(p_1,p_2)$ is an equivalence pair if, and only if, for every object $T$ of $\mathcal{C}$, the map
\[
    G(T)\times X(T) \sim (G\times X)(T) \to X(T)\times X(T)
\]
defined by this pair is injective, i.e. if the group $G(T)$ acts \emph{freely} on the set $X(T)$, i.e. if $g\in G(T)$, $x\in X(T)$, and $g\cdot x=x$, then $g$ is the identity element of the group $G(T)$.
We then say that $G$ \emph{acts freely} on $X$ (or that $X$ is a \emph{principal $\mathcal{C}$-space under $G$}).
The equivalence relation associated to the pair $(p_1,p_2)$ is then called the \emph{equivalence relation defined by the group $G$} acting freely on $X$.
If $X\times X$ also exists, and we consider the morphism
\[ p\colon G\times X\to X\times X \]
defined by the pair $(p_1,p_2)$, then the condition that $G$ acts freely implies that $p$ is a \emph{monomorphism}.

Of course, even if $G$ dose not act freely on $X$, we still wish to have existence criteria for a quotient of $X$ by $G$, i.e. for the cokernel of the above pair $(p_1,p_2)$.

The cokernel in question will often be denoted by $X/G$, or by $X\backslash G$ if $G$ acts on the left (with the previous notation being reserved for when $G$ acts on the right).
We note that, even if the "image" of $G\times X$ under $p$ exists (this image being defined, for example, as the smallest sub-object of $X\times X$ through which we can factor $p$), say, $R$, then this is usually not an equivalence relation on $X$.
If we then try to pass directly to the quotient under $R$ (or, more precisely, under the pair of morphisms from $R$ to $X$ induced by the two projections $\mathrm{pr}_i$), then we lose the particular characteristics of the original pair $(p_1,p_2)$.
It is thus important to find a generalisation of the notion of equivalence relations, appealing directly to the pair defined by a $\mathcal{C}$-group with operators.
