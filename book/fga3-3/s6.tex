% !TeX root = ../../fga.tex
\section{Quotient by a proper and flat equivalence relation}\label{fga3.iii-6}

\begin{theorem}\label{fga3.iii-6-theorem-6.1}
    Let $S$ be a \emph{locally Noetherian} prescheme, $X$ a \emph{quasi-projective} $S$-scheme, and $\mathcal{R}$ an equivalence pre-relation on $X$, such that:
    \begin{enumerate}[label=\alph*.]
        \item $p_1\colon R_1\to X$ is proper and flat; and
        \item $R_1\to X\times_S X$ is a finite morphism (or, equivalently, by (a), a morphism with finite fibres, which is a condition that is automatically satisfied if $\mathcal{R}$ comes from an equivalence relation).
    \end{enumerate}
    Under these conditions:
    \begin{enumerate}[i.]
        \item $Y=X/\mathcal{R}$ exists, and (if $S$ is Noetherian) is quasi-projective over $S$.
              \footnote{\emph{[Comp.]} The fact that $Y=X/\mathcal{R}$ is quasi-projective over $S$ has only been proven, for now, in the case where $\mathcal{R}$ comes from an equivalence relation.}
        \item The canonical morphism $f\colon X\to Y$ is surjective, proper, and open, and its fibres are the equivalence classes $p_2(p_1^{-1}(x))$ in $X$ modulo $\mathcal{R}$, and so $Y$ can be identified with the topological quotient space of $X$ by the set-theoretic equivalence relation defined by $\mathcal{R}$.
              Finally, $R_1\to X\times_Y X$ is surjective.
        \item If $\mathcal{R}$ comes from an equivalence relation, then the equivalence relation in question is effective, i.e. $R_1\to X\times_Y X$ is an isomorphism, and, further, $f\colon X\to Y$ is flat (and thus faithfully flat).
    \end{enumerate}
\end{theorem}


\begin{cproof}
    For the proof, we can reduce to \Cref{fga3.iii-5-theorem-5.1} by considering suitable quasi-sections of $X$ for $\mathcal{R}$, with the proof being analogous to the construction of algebraic quotient groups in the \emph{Séminaire Chevalley}.
\end{cproof}


In summary:

\begin{aside}\label{fga3.iii-6-scholium}
    Let $X$ be quasi-projective over $S$, with $S$ locally Noetherian.
    Then the data of a proper, faithfully flat, and surjective morphism $f\colon X\to Y$ from $X$ to an $S$-prescheme $Y$ is equivalent to the data of an equivalence relation $R$ on $X$ such that $p_1\colon R\to X$ is proper and flat.
\end{aside}


The same method gives the following result:


\begin{theorem}\label{fga3.iii-6-theorem-6.2}
    Let $S$ be a Noetherian prescheme, $X$ a prescheme of finite type over $S$, and $\mathcal{R}$ an equivalence pre-relation on the $S$-prescheme $X$.
    Suppose that
    \begin{enumerate}[label=\alph*.]
        \item $p_1\colon R_1\to X$ is flat and of finite type; and
        \item the morphism $R_1\to X\times_S X$ is quasi-finite (i.e. has finite fibres).
    \end{enumerate}

    Then there exists a \emph{dense} open subset $U$ of $X$ that is \emph{saturated} for $\mathcal{R}$, such that:

    \begin{enumerate}[i.]
        \item $\mathcal{R}_U$ is the equivalence pre-relation induced by $\mathcal{R}$ on $U$, then $U/\mathcal{R}_U$ exists, and is of finite type over $S$.
        \item The canonical morphism $U\to U/\mathcal{R}_U$ is surjective and open, and its fibres are the set-theoretic equivalence classes for $\mathcal{R}_U$ (and thus $U/\mathcal{R}_U$ is a topological quotient space of $U$ by the set-theoretic equivalence relation defined by $\mathcal{R}_U$).
              Finally, the morphism $(\mathcal{R}_U)_1\to U\times_{U/\mathcal{R}_U}U$ is surjective.
        \item If $\mathcal{R}$ comes from an equivalence relation, then we can suppose that $U\to U/\mathcal{R}_U$ is faithfully flat and that $\mathcal{R}_U$ is effective.
    \end{enumerate}
\end{theorem}



This is a result of an essentially "birational" nature.


\begin{remark}\label{fga3.iii-6-remarks-6.3}
    ~
    \begin{enumerate}
        \item I do not know if, in \Cref{fga3.iii-6-theorem-6.1} and \Cref{fga3.iii-6-theorem-6.2}, hypothesis (b) is useless.
              In practice, it obliges us, in the passage to the quotient by groups, to restrict to he case where the stabilisers are all finite groups.
        \item We can ask if there are results analogous to \Cref{fga3.iii-6-theorem-6.1} and \Cref{fga3.iii-6-theorem-6.2} without any flatness hypothesis.
              I have no counter example in this direction.
              However, even keeping the flatness hypothesis, and restricting to equivalence relations such that $p_1\colon R\to X$ is flat and quasi-finite (but not finite), and with $X$ affine, it can still be the case that $R$ is not effective: take the equivalence relations induced on the affine open subsets covering the Nagata variety (or a group with two elements acting in a "non-admissible" way).
    \end{enumerate}
\end{remark}