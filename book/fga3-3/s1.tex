% !TeX root = ../../fga.tex
\section{Equivalence relations, effective equivalence relations}\label{fga3.iii-1}

Let $\mathcal{C}$ be a category, and $X$ an object of $\mathcal{C}$.
A pair of morphisms
\[
    p_1,p_2\colon R\rightrightarrows X,
\]
is said to be an "\emph{equivalence pair}" in $\mathcal{C}$, with \emph{target} $X$ and \emph{source} $R$, if, for every object $T$ of $\mathcal{C}$, the corresponding maps
\[
    p_1(T),p_2(T)\colon R(T)\rightrightarrows X(T)
\]
(where we set $Y(T)=\operatorname{Hom}(T,Y)$ for any object $Y$ of $\mathcal{C}$) define a map
\[
    R(T)\to X(T)\times X(T)
\]
that induces a bijection from $R(T)$ to the graph of an equivalence relation on the set $X(T)$.
We introduce an evident equivalence relation on equivalence pairs with target $X$, and call an equivalence class an \emph{equivalence $\mathcal{C}$-relation} on $X$, or simply an equivalence relation if no confusion may arise.


If $X\times X$ exists, then the data of an equivalence relation on $X$ is equivalent to the data of a sub-object $R$ of $X\times X$ such that, for every object $T$ of $\mathcal{C}$, the subset of $(X\times X)(T)=X(T)\times X(T)$ that corresponds to $R(T)$ is the graph of an equivalence relation on $X(T)$.
Denoting the morphisms from $R$ to $X$ induced by the projections $\mathrm{pr}_1$ and $\mathrm{pr}_2$ by $p_1$ and $p_2$ (respectively), the above condition says that $(p_1,p_2)$ is an equivalence pair.
We can also express the axioms of a set-theoretical equivalence relation for the $R(T)$ in the $X(T)$ diagrammatically in $\mathcal{C}$ (under the assumption that both $X\times X$ and the fibre product $(R,p_2)\times_X(R,p_1)$ exist), following the general principle of FGA 3.II, §A.1 \Cref{fga3.ii-a.1}.
We will not need this.

Every time that we have a pair of morphisms $(p_1,p_2)$ with the same source $R$ and the same target $X$, we can define the \emph{cokernel} of the pair as an object $Y$ of $\mathcal{C}$ that represents the contravariant (in $Z$) functor
\[
    \operatorname{Hom}_{p_1,p_2}(X,Z)
\]
which denotes the set of morphisms $u$ from $X$ to $Z$ such that $up_1=up_2$.
If $Y$ exists, then it is determined up to unique isomorphism.
We will denote it by $Y/(p_1,p_2)$, or, by an abuse of notation, $Y/R$, with the latter mostly being used when $(p_1,p_2)$ is an equivalence pair: it is then common to identify, in notation, the equivalence relation defined by the pair with the one defined by $R$.
Note that, if we consider $Y$ as a quotient of $X$, then it depends only on the equivalence relation defined by the pair $(p_1,p_2)$

We now start with a morphism
\[
    f\colon X\to Y
\]
which allows us to consider $X$ as an "object over $Y$", and we suppose that the fibre product
\[
    \mathcal{R}(f)
    = X\times_Y X
\]
exists.
Let $p_1$ and $p_2$ be its projections.
Then $(p_1,p_2)$ is an equivalence pair, and is said to be \emph{associated} with the morphism $f$.
It thus defines an equivalence relation, which is said to be \emph{associated} with the morphism $f$.

We say that a pair of morphisms $(p_1,p_2)$ with target $X$, and source $R$, is an \emph{effective equivalence pair} if

\begin{enumerate}[i.]
    \item the cokernel $Y=X/(p_1,p_2)$ exists;
    \item the fibre product $X\times_Y X$ exists; and
    \item the morphism $R\to X\times_Y X$ with components $p_1$ and $p_2$ is an isomorphism.
\end{enumerate}

Then the pair $(p_1,p_2)$ is indeed an equivalence pair.
We also say that the equivalence relation that it defines is an \emph{effective equivalence relation}.

We say that a morphism $f\colon X\to Y$ is an \emph{effective epimorphism} if

\begin{enumerate}[i.]
    \item the fibre product $R=X\times_Y X$ exists;
    \item the quotient $X/(p_1,p_2)$ exists, where $p_1$ and $p_2$ are the projections from $R$ to $X$; and
    \item the morphism $X/(p_1,p_2)\to Y$ induced by $f$ is an isomorphism.
\end{enumerate}

Then $f$ is indeed an epimorphism, and even a strict epimorphism (cf. FGA 3.I, §A.2.c \Cref{fga3.i-a.2.c}), with the converse being true if the fibre product $X\times_Y X$ exists.
We also say that the quotient object of $X$ defined by the epimorphism $f$ is an \emph{effective quotient} of $X$.

The above definitions imply the following "\emph{Galois correspondence}":

\begin{proposition}\label{fga3.iii-1-proposition-1.1}
    There is a bijective correspondence, respecting the natural orders, between the set of effective equivalence relations $R$ on $X$ and the set of effective quotients $Y$ of $X$, with such an $R$ corresponding to the effective quotient $X/R$, and such a $Y$ corresponding to the effective equivalence relation defined by the canonical projection $X\to Y$ (which is defined by the fibre product $X\times_Y X$ endowed with its two projections).
\end{proposition}


In very nice categories (sets, sheaves of sets, etc.), every quotient is effective, and every equivalence relation is effective.
This is no longer true in categories such as the category of preschemes over a given prescheme $S$, not even if $S$ is the spectrum of field, nor even if we restrict to finite schemes over $S$.
The question of effectiveness, and even (in the case of non-finite preschemes over $S$) the question of existence of quotients, more often than not turn out to be delicate.
