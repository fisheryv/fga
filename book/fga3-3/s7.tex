% !TeX root = ../../fga.tex
\section{Applications}\label{fga3.iii-7}



As we said in the introduction, the most important application of \Cref{fga3.iii-6-theorem-6.1} is the construction of Picard schemes, as well as solutions to various other problems of "modules", to which we will later return.



We obtain a simple proof of the following result of Shimura:

\begin{proposition}\label{fga3.iii-7-proposition-7.1}
    Let $A$ be an abelian scheme defined over a discrete valuation ring $V$ with field of fractions $K$.
    Then every abelian scheme $B'$ over $K$ that is isogenous to a quotient of $A\otimes_V K$ "simplifies well for $V$", i.e. is isomorphic to some $B\otimes_V A$, where $B$ is an abelian scheme over $V$ (essentially unique, we recall).
\end{proposition}


\begin{cproof}
    We can suppose that $B'$ is the quotient of $A_K$ by a subscheme in groups $C'$.
    (N.B. $C'$ will not, in general, be "reduced", i.e. its local rings will have nilpotent elements).
    Consider the closed subscheme $C$ of $A$ given by "the closure" of $C'$, i.e. the smallest closed subscheme of $A$ such that $C_K$ contains $C'$.
    Then $C_K=C'$, and, since $V$ is a discrete valuation ring, we easily deduce that $C$ is a subscheme in groups of $A$ over $V$.
    Since $A$ is proper over $\operatorname{Spec}(V)=S$, so too is $C$.
    Further, $A$ is projective over $S$.
    We can thus apply \Cref{fga3.iii-6-theorem-6.1} in order to construct $A/C=B$, which is the desired $B$.
\end{cproof}


Finally, essentially known arguments allow us to extract from \Cref{fga3.iii-6-theorem-6.2} the following result:


\begin{theorem}\label{fga3.iii-7-theorem-7.2}
    Let $S$ be the spectrum of an Artinian ring, $F$ and $G$ group schemes of finite type over $S$, and $u\colon F\to G$ a homomorphism of group schemes over $S$.
    Suppose that

    \begin{enumerate}[i.]
        \item $F$ is flat over $S$; and
        \item the kernel of $u$ is finite.
    \end{enumerate}

    Under these conditions, the quotient scheme $G/F$ exists, and the canonical morphism $G\to G/F$ is surjective and open, and its fibres are the set-theoretic equivalence classes defined by the right action of $F$ on $G$.
    Finally, if $u$ is a monomorphism, then the morphism $G\to G/F$ is flat, and the morphism $G\times F\to G\times_{(G/F)}G$ is an isomorphism, or, in other words, $G$ is a principal homogeneous space over $G/F$, with structure group $F$ (acting on the right), or rather $F\times_S(G/F)$ considered as a group scheme over $G/F$ (cf. FGA 3.I, §B.6 \Cref{fga3.i-b.6}).
\end{theorem}

\begin{corollary}\label{fga3.iii-7-corollary-7.3}
    Under these conditions, for $G$ to be flat over $S$, it is necessary and sufficient that $G/F$ be flat over $S$.
    If this condition is satisfied, then the passage to the quotient by $F$ commutes with every extension of the base $S$, and if $F$ is an invariant subgroup of $G$, then $G/F$ can be endowed with the structure of a \emph{quotient group} of $G$ by $F$.
\end{corollary}


The situation is particularly simple if $S$ is the spectrum of a field, since then every $S$-prescheme is automatically flat over $S$.
We find:

\begin{corollary}\label{fga3.iii-7-corollary-7.4}
    Let $F$ and $G$ be group schemes of finite type over a field $k$, and let $u\colon F\to G$ be a homomorphism of $k$-groups.
    Then $u$ factors as $F\to F'\to G$, where $F\to F'$ is a homomorphism given by passing to the quotient by the closed subgroup $\operatorname{Ker} u$ of $F$, and where $F'\to G$ is a group homomorphism that is a closed immersion.
    The quotient $G/F=G/F'$ exists.
    The usual formalism (as in the Noether theorems) holds amongst algebraic groups over $k$.
\end{corollary}

This result allows us to treat the passage to the quotient in a uniform way for algebraic (in the classical sense, i.e. irreducible over $k$ and simple over $k$) groups, and the passage to the quotient by "infinitesimal" subgroups considered by Cartier.
It is advantageous to consider the "hyperalgebras" introduced by Cartier, following from the work of Dieudonné on formal groups, as groups in the category of formal schemes over $k$, and, if necessary (if they correspond to hyperalgebras of finite rank over $k$), as algebraic groups that are finite over $k$.
