% !TeX root = ../../fga.tex
\section{Quotient by a finite and flat equivalence relation}\label{fga3.iii-5}

\begin{theorem}\label{fga3.iii-5-theorem-5.1}
    Let $X=\operatorname{Spec}(B)$ be an affine scheme, $\mathcal{R}$ an equivalence pre-relation on $X$, whose component $R_1$ is affine: say, $R_1=\operatorname{Spec}(C)$.
    We suppose that the first projection $p_1\colon R_1\to X$ is a finite and locally free morphism, i.e. that the corresponding homomorphism of rings $p'_1\colon B\to C$ makes $C$ a projective $B$-module of finite type.
    Let $A$ be the subring of $B$ given by the kernel of the pair of homomorphisms $p'_1,p'_2\colon B\rightrightarrows C$ (i.e. the set of elements $b$ such that $p'_1(b)=p'_2(b)$).
    Let $Y=\operatorname{Spec}(A)$, and $f\colon X\to Y$ the morphism defined by the embedding of $A$ into $B$.
    Under these conditions:

    \begin{enumerate}[i.]
        \item $B$ is integral over $A$, i.e. $f$ is an integral morphism.
        \item The morphism $f$ is surjective, and its fibres are the set-theoretic equivalence classes $p_2(p_1^{-1}(x))$ in $X$ modulo $\mathcal{R}$, and the topology of $Y$ is the quotient of that of $X$.
        \item $Y$ is the quotient of $X$ by $\mathcal{R}$ in the category of preschemes.
        \item If $\mathcal{R}$ comes from an equivalence \emph{relation}, then the morphism $f\colon X\to Y$ is finite and locally free (i.e. $B$ is a projective $A$-module of finite type), and the equivalence relation is effective, i.e. $R_1\to X\times_Y X$ is an isomorphism.
    \end{enumerate}
\end{theorem}



This theorem generalises the well-known theorem concerning the case of a finite group $G$ acting by automorphisms on the ring $B$, and with ring $A$ of invariants, and the proof is analogous to the known proof.
We can make (iii) more precise as follows:


\begin{corollary}\label{fga3.iii-5-corollary-5.2}
    The canonical morphism $R_1\to X\times_Y X$ is \emph{surjective}.
\end{corollary}


Let $\mathcal{R}$ continue to be a "finite and locally free" equivalence pre-relation on $X$, but with $X$ now being an arbitrary prescheme.
Suppose that we can find a prescheme $Y$ and a morphism $f\colon X\to Y$ such that $fp_1=fp_2$, and further such that the sequence
\[
    \mathcal{O}_Y \to f_*(\mathcal{O}_X) \rightrightarrows g_*(\mathcal{O}_R)
\]
of homomorphisms of sheaves of rings on $Y$ is exact (where $g=fp_i$).
It then follows from the theorem that we have conclusions (i) to (iv) analogous to those of the theorem, and, in particular, by (iii), $Y$ is the quotient of $X$ by $\mathcal{R}$, and thus determined up to unique isomorphism.
Under these conditions, we say that the equivalence pre-relation $\mathcal{R}$ on $X$ is \emph{admissible}.
With this definition:


\begin{theorem}\label{fga3.iii-5-theorem-5.3}
    Let $X$ be a prescheme, and $\mathcal{R}$ an equivalence pre-relation on $X$ such that $p_1\colon R_1\to X$ is a finite and locally free morphism.
    For $\mathcal{R}$ to be admissible, it is necessary and sufficient that every set-theoretic equivalence class $p_2(p_1^{-1}(x))$ in $X$ modulo $\mathcal{R}$ be contained in an affine open subset (a condition that is always satisfied if every finite subset of $X$ is contained in an affine open subset, for example if $X$ is quasi-projective over an affine scheme).
\end{theorem}


We can in fact easily show that every equivalence class modulo $\mathcal{R}$ in $X$ is then contained in an affine open subset that is \emph{stable} under $\mathcal{R}$, and we construct the quotient $Y$ by gluing the pieces obtained by applying \Cref{fga3.iii-5-theorem-5.1}.


\begin{corollary}\label{fga3.iii-5-corollary-5.4}
    Suppose that this condition is satisfied, and, further, that $\mathcal{R}$ comes from an equivalence relation.
    Then the equivalence relation is question is effective, i.e. $R_1\to X\times_Y X$ is an isomorphism, and $f\colon X\to Y$ is a finite and locally free morphism.
\end{corollary}


We then immediately conclude, by descent, the following:


\begin{corollary}\label{fga3.iii-5-corollary-5.5}
    Under the conditions of \Cref{fga3.iii-5-corollary-5.4}, for $X$ to be everywhere of rank $n$ over $Y$, it is necessary and sufficient that $(R_1,p_1)$ be everywhere of rank $n$ over $X$.
    If $X$ and $R_1$ are $Z$-preschemes, and $p_1$ and $p_2$ are $Z$-morphisms (and thus $Y$ a $Z$-prescheme), then $X$ is flat over $Z$ if and only if $Y$ is flat over $Z$.
\end{corollary}

In summary:

\begin{aside}\label{fga3.iii-5-scholium}
    The data of a finite, locally free, and surjective morphism $f\colon X\to Y$ of preschemes is equivalent to the data of a prescheme $X$ endowed with an equivalence relation $R$ such that $p_1\colon R\to X$ is finite and locally free, and such that every class $p_2(p_1^{-1}(x))$ is contained in an affine open subset.
\end{aside}

\begin{remark}\label{fga3.iii-5-remarks-5.6}
    ~
    \begin{enumerate}
        \item We have not needed to make any Noetherian hypothesis.
        \item This idea of passing to the quotient contains, as a particular case, the "inseparable descent" of Cartier, which corresponds to the determination of finite and locally free morphisms $f\colon X\to Y$ such that $f_*(\mathcal{O}_X)$ admits a $p$-basis with respect to $\mathcal{O}_Y$ (where $X$ is a given prescheme whose sheaf $\mathcal{O}_X$ is annihilated by the prime number $p>0$).
              We note that this result can also be easily expressed without any regularity hypothesis on the local rings, and without supposing that $X$ is an algebraic scheme over a field.
              The theory of Jacobson–Bourbaki is obtained by taking $X$ to be the spectrum of a field of characteristic $p$.
        \item Gabriel had already obtained a particular case of \Cref{fga3.iii-5-theorem-5.3} in the theory of passing to the quotient for finite commutative groups over a field $k$.
              (Compare with \Cref{fga3.iii-7-corollary-7.4}).
    \end{enumerate}
\end{remark}