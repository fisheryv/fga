% !TeX root = ../../fga.tex
\section{Preliminaries on categories}\label{fga3.i-a}

\subsection{Fibred categories, descent data, $\mathcal{F}$-descent morphisms}\label{fga3.i-a.1}

\label{fga3.i-a.1.a}
\begin{definition}\label{fga3.i-a.1-definition-1.1}
    A \emph{fibred category $\mathcal{F}$ with base $\mathcal{C}$} (or \emph{over $\mathcal{C}$}) consists of
    \begin{itemize}
        \item a category $\mathcal{C}$
        \item for every $X\in\mathcal{C}$, a category $\mathcal{F}_X$
        \item for every $\mathcal{C}$-morphism $f\colon X\to Y$, a functor $f^*\colon\mathcal{F}_Y\to\mathcal{F}_X$, which we also write as
              \[f^*(\xi) = \xi \times_Y X\]
              for $\xi\in\mathcal{F}_Y$ (with $X$ being thought of as an "object of $\mathcal{C}$ over $Y$", i.e. as being endowed with the morphism $f$)
        \item for any two composible morphisms $X\xrightarrow{f}Y\xrightarrow{g}Z$, an isomorphism of functors
              \[c_{f,g}\colon (gf)^* \to f^*g^*\]
    \end{itemize}
    with the above data being subject to the conditions that
    \begin{enumerate}[i.]
        \item $\operatorname{id}^*=\operatorname{id}$
        \item  $c_{f,g}$ is the identity isomorphism if $f$ or $g$ is an identity isomorphism
        \item for any three composible morphisms $X\xrightarrow{f}Y\xrightarrow{g}Z\xrightarrow{h}T$, the following diagram, given by using the isomorphisms of the form $c_{u,v}$, commutes:
              \[
                  \begin{CD}
                      (h(gf))^* @= ((hg)f)^*
                      \\@VVV @VVV
                      \\(gf)^*h^* @. f^*(hg)^*
                      \\@VVV @VVV
                      \\(f^*g^*)h^* @= f^*(g^*h^*)
                  \end{CD}
              \]
    \end{enumerate}
\end{definition}

\begin{example}\label{fga3.i-a.1-example-1}
    Let $\mathcal{C}$ be a category where all fibre products exist.
    We then define a fibred category $\mathcal{F}$ with base $\mathcal{C}$ by setting $\mathcal{F}_X$ to be the category of objects of $\mathcal{C}$ over $X$, and the functor $f^*\colon\mathcal{F}_Y\to\mathcal{F}_X$ corresponding to a morphism $f\colon X\to Y$ being defined by the \emph{fibre product} $Z\mapsto Z\times_Y X$.
\end{example}

\begin{example}\label{fga3.i-a.1-example-2}
    Let $\mathcal{C}$ be the category of preschemes, and, for $X\in\mathcal{C}$, let $\mathcal{F}_X$ be the category of quasi-coherent sheaves of modules on $X$.
    If $f\colon X\to Y$ is a morphism of preschemes, then $f^*\colon\mathcal{F}_Y\to\mathcal{F}_X$ is the \emph{inverse image of sheaves of modules} functor.
    We thus obtain a category fibred over $\mathcal{C}$.
\end{example}

\label{fga3.i-a.1.b}
\begin{definition}\label{fga3.i-a.1-definition-1.2}
    A diagram of maps of sets
    \[
        E \xrightarrow{u}
        E' \overset{v_1}{\underset{v_2}{\rightrightarrows}}
        E''
    \]
    is said to be \emph{exact} if $u$ is a bijection from $E$ to the subset of $E'$ consisting of the $x'\in E'$ such that $v_1(x')=v_2(x')$.
\end{definition}

\begin{definition}\label{fga3.i-a.1-definition-1.3}
    Let $\mathcal{F}$ be a fibred category with base $\mathcal{C}$, and consider a diagram of morphisms in $\mathcal{C}$
    \[
        S \xleftarrow{\alpha}
        S' \overset{\beta_1}{\underset{\beta_2}{\leftleftarrows}}
        S''
    \]
    such that $\alpha\beta_1=\alpha\beta_2$;
    this diagram is said to be \emph{$\mathcal{F}$-exact} if, for every pair $(\xi,\eta)$ of elements of $\mathcal{F}_S$, the diagram of sets

    \begin{equation}\tag{+}\label{fga3.i-a.1-definition-1.3-equation}
        \operatorname{Hom}(\xi,\eta) \xrightarrow{\alpha^*}
        \operatorname{Hom}(\alpha^*(\xi),\alpha^*(\eta)) \overset{\beta_1^*}{\underset{\beta_2^*}{\rightrightarrows}}
        \operatorname{Hom}(\gamma^*(\xi),\gamma^*(\eta))
    \end{equation}

    (where $\gamma=\alpha\beta_1=\alpha\beta_2$) is exact.

    In this diagram above, for simplicity, we have identified $\beta_i^*\alpha^*$ with $(\alpha\beta_i)^*=\gamma^*$, using $c_{\beta_i,\alpha}$.
\end{definition}


\begin{definition}\label{fga3.i-a.1-definition-1.4}
    Let $\mathcal{F}$ be a fibred category with base $\mathcal{C}$, and consider morphisms $\beta_1,\beta_2\colon S''\to S'$ in $\mathcal{C}$.
    Let $\xi'\in\mathcal{F}_{S'}$.
    We define a \emph{gluing data} on $\xi'$ (with respect to the pair $(\beta_1,\beta_2)$) to be an isomorphism from $\beta_1^*(\xi')$ to $\beta_2^*(\xi')$.
    If $\xi',\eta'\in\mathcal{F}_{S'}$ are both endowed with gluing data, then a morphism $u\colon\xi'\to\eta'$ in $\mathcal{F}_{S'}$ is said to be \emph{compatible with the gluing data} if the following diagram commutes:
    \[
        \begin{CD}
            \beta_1^*(\xi') @>>> \beta_2^*(\xi')
            \\@VVV @VVV
            \\\beta_1^*(\eta') @>>> \beta_2^*(\eta').
        \end{CD}
    \]
\end{definition}


With this definition, the objects of $\mathcal{F}_{S'}$ that are endowed with gluing data (with respect to $\beta_1$ and $\beta_2$) then form a \emph{category}.
If $\alpha\colon S'\to S$ is a morphism such that $\alpha\beta_1=\alpha\beta_2$, then, for every $\xi\in\mathcal{F}_{S'}$, the object $\xi'=\alpha^*(\xi)$ of $\mathcal{F}_{S'}$ is endowed with a canonical gluing data, since
\[
    \beta_i^*\alpha^*(\xi)
    \simeq (\alpha\beta_i)^*(\xi)
    = \gamma^*(\xi),
\]
where we again set $\gamma=\alpha\beta_1=\alpha\beta_2$;
furthermore, if $u\colon\xi\to\eta$ is a morphism in $\mathcal{F}_s$, then
\[
    \alpha^*(u)\colon \alpha^*(\xi) \to \alpha^*(\eta)
\]
is a morphism in $\mathcal{F}_{S'}$ that is compatible with the canonical gluing data.
We thus obtain a \emph{canonical functor} from the category $\mathcal{F}_S$ to the category of objects of $\mathcal{F}_{S'}$ endowed with gluing data with respect to the pair $(\beta_1,\beta_2)$.
With this, we can also rephrase \Cref{fga3.i-a.1-definition-1.3} by saying that \Cref{fga3.i-a.1-definition-1.3-equation} is \emph{$\mathcal{F}$-exact} if the above functor is \emph{fully faithful}, i.e. if the above functor defines an equivalence between the category $\mathcal{F}_S$ and a subcategory of the category of objects of $\mathcal{F}_{S'}$ endowed with gluing data with respect to $(\beta_1,\beta_2)$.


\begin{definition}\label{fga3.i-a.1-definition-1.5}
    We say that a gluing data on $\xi'\in\mathcal{F}_{S'}$ is \emph{effective} (with respect to $\alpha$) if $\xi'$, endowed with this data, is isomorphic to $\alpha^*(\xi)$ for some $\xi\in\mathcal{F}_S$.
\end{definition}


In the case where \Cref{fga3.i-a.1-definition-1.3-equation} is $\mathcal{F}$-exact, the object $\xi$ in \Cref{fga3.i-a.1-definition-1.5} is then determined up to unique isomorphism, and \emph{the category $\mathcal{F}_S$ is equivalent to the category of objects of $\mathcal{F}_{S'}$ endowed with effective gluing data}.

\label{fga3.i-a.1.c}
The most important case is that where
\[
    S'' = S' \times_S S',
\]
with the $\beta_i$ being the two projections $p_1$ and $p_2$ from $S'\times_S S'$ to its two factors (where we now suppose that $\mathcal{C}$ has all fibre products).
We then have two immediate necessary conditions for a gluing data $\varphi\colon p_1^*(\xi')\to p_2^*(\xi')$ on some $\xi'\in\mathcal{F}_S$ to be effective:

\begin{enumerate}[i.]
    \item $\Delta^*(\varphi) = \operatorname{id}_\xi$, where $\Delta\colon S'\to S'\times_S S'$ denotes the diagonal morphism, and where we identify $\Delta^* p_i^*(\xi')$ with $(p_i\Delta)^*(\xi')=\xi'$.
    \item $p_{31}^*(\varphi) = p_{32}^*(\varphi)p_{21}^*(\varphi)$, where $p_{ij}$ denotes the canonical projection from $S'\times_S S'\times_S S'$ to the partial product of its $i$th and $j$th factors.
\end{enumerate}

\begin{definition}\label{fga3.i-a.1-definition-1.6}
    We define \emph{descent data} on $\xi'\in\mathcal{F}_{S'}$, with respect to the morphism $\alpha\colon S'\to S$, to be a gluing data on $\xi'$ with respect to the pair $(p_1,p_2)$ of canonical projections $S'\times_S S'\to S'$ that satisfies conditions (i) and (ii) above.
\end{definition}

\begin{definition}\label{fga3.i-a.1-definition-1.7}
    A morphism $\alpha\colon S'\to S$ is said to be an \emph{$\mathcal{F}$-descent morphism} if the diagram of morphisms
    \[
        S \xleftarrow{\alpha}
        S' \overset{p_1}{\underset{p_2}{\leftleftarrows}}
        S'\times_S S'
    \]
    is $\mathcal{F}$-exact (\Cref{fga3.i-a.1-definition-1.3});
    we say that $\alpha$ is a \emph{strict $\mathcal{F}$-descent morphism} if, further, every descent data (\Cref{fga3.i-a.1-definition-1.6}) on any object of $\mathcal{F}_{S'}$ is effective.

    This latter condition (of strictness) can also be stated in a more evocative way:
    "giving an object of $\mathcal{F}_S$ is equivalent to giving an object of $\mathcal{F}_{S'}$ endowed with a descent data".
\end{definition}


Note that, if an $\mathcal{F}$-descent morphism
\footnote{\emph{[Comp.]} It is useless to assume here that $\alpha$ is an $\mathcal{F}$-descent morphism.}
$\alpha\colon S'\to S$ admits a \emph{section} $s\colon S\to S'$ (i.e. a morphism $s$ such that $\alpha s=\operatorname{id}_S$), then it is a strict $\mathcal{F}$-descent morphism:
if $\xi'\in\mathcal{F}_{S'}$ is endowed with descent data with respect to $\alpha$, then "it comes from" $\xi=s^*(\xi')$.

\label{fga3.i-a.1.d}
We can present the above notions in a more intuitive manner, by introducing, for an object $T$ of $\mathcal{C}$ over $S$, the set
\[
    \operatorname{Hom}_S(T,S') = S'(T),
\]
where elements will be denoted by $t$, $t'$, etc.
Given an object $\xi'\in\mathcal{F}_{S'}$, there then corresponds, to every $t\in S'(T)$, an object $t^*(\xi')$ of $\mathcal{F}_T$, which will also be denoted by $\xi'_t$.
A gluing data on $\xi'$ (with respect to $(p_1,p_2)$) is then defined by the data, for every $T$ over $S$, and every pair of points $t,t'\in S'(T)$, of an isomorphism
\[
    \varphi_{t',t}\colon \xi'_t \to \xi'_{t'}
\]
(satisfying the evident conditions of functoriality in $T$).
Conditions (i) and (ii) of \Cref{fga3.i-a.1.c} can then be written as

\begin{enumerate}
    \item[i bis.] $\varphi_{t,t}=\operatorname{id}_{\xi'_t}$, for all $T$ and all $t\in S'(T)$.
    \item[ii bis.] $\varphi_{t,t''}=\varphi_{t,t'}\varphi_{t',t''}$, for all $T$ and all $t,t',t''\in S'(T)$.
\end{enumerate}

We can show that (ii bis) implies that $\varphi_{t,t}^2=\varphi_{t,t}$, by taking $t=t'=t''$, and thus, since $\varphi_{t,t}$ is an isomorphism by hypothesis, implies (i bis), which is thus a consequence of (ii bis) (and so (i) is also a consequence of (ii)).
But if we no longer suppose a priori that the $\varphi_{t,t}$ are isomorphisms (i.e. that $\varphi\colon p_1^*(\xi')\to p_2^*(\xi')$ is an isomorphism), then (ii bis) no longer necessarily implies (i bis);
the combination of (ii bis) and (i bis), however, does imply that the $\varphi_{t,t'}$ are isomorphisms (since we then have $\varphi_{t,t'}\varphi_{t',t}=\varphi_{t,t}=\operatorname{id}_{\xi'_t}$).

\subsection{Exact diagrams and strict epimorphisms, descent morphisms, and examples}\label{fga3.i-a.2}

\label{fga3.i-a.2.a}
\begin{definition}\label{fga3.i-a.2-definition-2.1}
    Let $\mathcal{C}$ be a category.
    A diagram of morphisms
    \[
        T \xrightarrow{\alpha}
        T' \overset{\beta_1}{\underset{\beta_2}{\rightrightarrows}}
        T''
    \]
    is said to be \emph{exact} if, for all $Z\in\mathcal{C}$, the corresponding diagram of sets
    \[
        \operatorname{Hom}(Z,T) \to
        \operatorname{Hom}(Z,T') \rightrightarrows
        \operatorname{Hom}(Z,T'')
    \]
    is exact (\Cref{fga3.i-a.1-definition-1.2}).
    We then say that $(T,\alpha)$ (or, by an abuse of language, $T$) is a \emph{kernel} of the pair $(\beta_1,\beta_2)$ of morphisms.
\end{definition}


This kernel is evidently determined up to unique isomorphism.
If $\mathcal{C}$ is the category of sets, then the above definition is compatible with \Cref{fga3.i-a.1-definition-1.2}.
Dually, we define the exactness of a diagram of morphisms in $\mathcal{C}$
\[
    S \xleftarrow{\alpha}
    S' \overset{\beta_1}{\underset{\beta_2}{\leftleftarrows}}
    S''
\]
and then say that $(S,\alpha)$ is a \emph{cokernel} of the pair $(\beta_1,\beta_2)$ of morphisms.


\begin{definition}\label{fga3.i-a.2-definition-2.2}
    A morphism $\alpha\colon S'\to S$ is said to be a \emph{strict epimorphism} if it is an epimorphism and, for every morphism $u\colon S'\to Z$, the following necessary condition is also sufficient for $u$ to factor as $S'\to S\to Z$:
    for every $S''\in\mathcal{C}$ and every pair $\beta_1,\beta_2\colon S''\to S$ of morphisms such that $\alpha\beta_1=\alpha\beta_2$, we also have that $u\beta_1=u\beta_2$.
\end{definition}


If the fibre product $S'\times_S S'$ exists, then it is equivalent to say that the diagram
\[
    S \xleftarrow{\alpha}
    S' \overset{p_1}{\underset{p_2}{\leftleftarrows}}
    S'\times_S S'
\]
is exact, i.e. that $S$ is a cokernel of the pair $(p_1,p_2)$.
In any case, a cokernel morphism is a strict epimorphism.
Note also that, if a strict epimorphism is also a monomorphism, then it is an isomorphism.
We leave to the reader the task of developing the dual notion of a \emph{strict monomorphism}.

To make the relation between the ideas of $\mathcal{F}$-descent morphisms and strict epimorphisms more precise, we introduce the following definitions:

\begin{definition}\label{fga3.i-a.2-definition-2.3}
    A morphism $\alpha\colon S'\to S$ is said to be a \emph{universal epimorphism} (resp. a \emph{strict universal epimorphism}) if, for every $T$ over $S$, the fibre product $T'=S'\times_S T$ exists, and the projection $T'\to T$ is an epimorphism (resp. a strict epimorphism).
\end{definition}

In very nice categories (such as the category of sets, the category of sets over a topological space, abelian categories, etc.), the four notions of "epijectivity" introduced above all coincide;
they are, however, all distinct in a category such as the category of preschemes, or the category of preschemes over a given non-empty prescheme $S$, even if we restrict to $S$-schemes that are finite over $S$.

\begin{definition}\label{fga3.i-a.2-definition-2.4}
    A morphism $\alpha\colon S'\to S$ is said to be a \emph{descent morphism} (resp. a \emph{strict descent morphism}) if it is an $\mathcal{F}$-descent morphism (resp. a strict $\mathcal{F}$-descent morphism) (cf. \Cref{fga3.i-a.1-definition-1.7}), where $\mathcal{F}$ denotes the fibred category over $\mathcal{C}$ of objects of $\mathcal{C}$ over objects of $\mathcal{C}$ (cf. \Cref{fga3.i-a.1-example-1}).
\end{definition}

\begin{proposition}\label{fga3.i-a.2-proposition-2.1}
    If $\mathcal{C}$ has all finite products and (finite) fibre products, then there is an identity between descent morphisms in $\mathcal{C}$ and strict universal epimorphisms in $\mathcal{C}$.
\end{proposition}

\label{fga3.i-a.2.b}
\begin{example}
    Let $\mathcal{C}$ be the category of preschemes.
    Let $S\in\mathcal{C}$, and let $S'$ and $S''$ be preschemes that are \emph{finite} over $S$, i.e. that correspond to sheaves of algebras $\mathcal{A}'$ and $\mathcal{A}''$ over $S$ that are quasi-coherent (as sheaves of modules) and of finite type (i.e. coherent, if $S$ is locally Noetherian).
    Let $\alpha\colon S'\to S$ be the structure morphism of $S'$, and let $\beta_1$ and $\beta_2$ be $S$-morphisms from $S''$ to $S'$, defined by algebra homomorphisms $\mathcal{A}'\to\mathcal{A}''$, which we also denote by $\beta_1$ and $\beta_2$.
    Using the fact that a finite morphism is closed (the first Cohen–Seidenberg theorem), we can easily prove that the diagram in $\mathcal{C}$
    \begin{equation}\tag{+}\label{fga3.i-a.2.b-equation}
        S \xleftarrow{\alpha}
        S' \overset{\beta_1}{\underset{\beta_2}{\leftleftarrows}}
        S''
    \end{equation}
    is exact if and only if the diagram of sheaves on $S$
    \[
        \mathcal{O}_S = \mathcal{A} \xrightarrow{\alpha}
        \mathcal{A}' \overset{\beta_1}{\underset{\beta_2}{\rightrightarrows}}
        \mathcal{A}''
    \]
    is exact.
    In particular, if $\alpha\colon S'\to S$ is a finite morphism corresponding to a sheaf $\mathcal{A}'$ of algebras on $S$, then $\alpha$ is a strict epimorphism if and only if the diagram of sheaves
    \[
        \mathcal{O}_S = \mathcal{A} \to
        \mathcal{A}' \overset{p_1}{\underset{p_2}{\rightrightarrows}}
        \mathcal{A}'\otimes_{\mathcal{A}}\mathcal{A}'
    \]
    is exact (it is an epimorphism if and only if $\mathcal{A}\to\mathcal{A}'$ is injective).
    If $S$ is affine of ring $A$, then $S'$ is affine of ring $A'$, with $A'$ finite over $A$, and so $S'\to S$ is a strict epimorphism if and only if $A\to A'$ is an isomorphism from $A$ to the subring of $A'$ consisting of the $x'\in A'$ such that
    \[
        1_{A'}\otimes_A x' - x'\otimes_A 1_{A'} = 0
    \]
    (it is an epimorphism if and only if $A\to A'$ is injective).
    As we have already mentioned, even if $S$ is the scheme of a local Artinian ring, then a finite morphism $S'\to S$ that is an epimorphism is not necessarily a strict epimorphism.
    However, we can prove that, \emph{if $S$ is a Noetherian prescheme, then every finite morphism $S'\to S$ that is an epimorphism is the composition of a finite sequence of strict epimorphisms} (also finite).
    This also shows that the composition of two strict epimorphisms is not necessarily a strict epimorphism.
\end{example}

\label{fga3.i-a.2.c}
If \Cref{fga3.i-a.2.b-equation} is an exact diagram of finite morphisms, then, for every \emph{flat} morphism $T\to S$ of prescheme, the diagram induced from \Cref{fga3.i-a.2.b-equation} by a change of base $T\to S$ is again exact.
It thus follows that, if $X$ and $Y$ are $S$-preschemes, with $X$ \emph{flat} over $S$, then the following diagram of maps (where $X'$ and $Y'$ are the inverse images of $X$ and $Y$ over $S'$, and $X''$ and $Y''$ are their inverse images over $S''$) is exact:
\[
    \operatorname{Hom}_S(X,Y) \to
    \operatorname{Hom}_{S'}(X',Y') \rightrightarrows
    \operatorname{Hom}_{S''}(X'',Y'').
\]
In particular, if $\mathcal{F}$ denotes the fibred category (over the category $\mathcal{C}$ of preschemes) such that, for $X\in\mathcal{C}$, $\mathcal{F}_X$ is the category of \emph{flat} $X$-preschemes, then the diagram \Cref{fga3.i-a.2.b-equation} is $\mathcal{F}$-exact.
(This result becomes false if we do not impose the flatness hypothesis; in particular, a finite strict epimorphism is not necessarily a descent morphism).
We similarly see that \Cref{fga3.i-a.2.b-equation} is $\mathcal{F}$-exact if $\mathcal{F}$ denotes the fibred category for which $\mathcal{F}_X$ is the category of \emph{flat} quasi-coherent sheaves on the prescheme $X$ (here, again, the flatness hypothesis is essential).
In either case, the question of \emph{effectiveness} of a gluing data (and, more specifically, of a descent data, when $S''=S'\times_S S'$) on a flat object over $S'$ is delicate (and its answer in many particular cases in one of the principal objects of these current articles).
The speaker does not know if, for every finite strict epimorphism $S'\to S$, every descent data on a flat quasi-coherent sheaf on $S'$ is effective (even if we suppose that $S$ is the spectrum of a local Artinian ring, and we restrict to locally free sheaves of rank $1$).
More generally, let $A$ be a ring, and $A'$ an $A$-algebra (where everything is commutative) such that the diagram
\[
    A \to
    A' \rightrightarrows
    A'\otimes_A A'
\]
is exact, which is equivalent to saying that the corresponding morphism $S'\to S$ between the spectra of $A'$ and $A$ is an $\mathcal{F}$-descent morphism, where $\mathcal{F}$ is the fibred category of flat quasi-coherent sheaves.
Let $M'$ be a flat $A'$-module endowed with a descent data to $A$, i.e. with an isomorphism
\[
    \varphi\colon M'\otimes_A A'
    \xrightarrow{\sim} A'\otimes_A M'
\]
of $(A'\otimes_A A')$-modules that satisfies conditions (i) and (ii) of \Cref{fga3.i-a.1.c} (which we leave to the reader to write out in terms of modules).
Is this data effective (relative to the fibred category of flat quasi-coherent sheaves)?
Let $M$ be the subset of $M'$ consisting of the $x'\in M'$ such that
\[
    \varphi(x'\otimes_A 1_{A'}) = 1_{A'}\otimes_A x',
\]
which is a sub-$A$-module of $M'$.
The canonical injection $M\to M'$ defines a homomorphism of $A'$-modules $M\otimes_A A'\to M'$.
\emph{The effectiveness of $\varphi$ then implies the following: $M$ is a flat $A$-module, and the above homomorphism is an isomorphism.}

\begin{remark}\label{fga3.i-a.2.c-remark}
    In the above, we have imposed no flatness hypotheses on the morphisms of the diagram \Cref{fga3.i-a.2.b-equation}, and this obliges us, in order to have a technique of descent, to impose flatness hypotheses on the objects over $S$ and $S'$ that we consider.
    In \Cref{fga3.i-b.2}, we will impose a flatness hypothesis on $\alpha\colon S'\to S$, which will allow us to have a technique of descent for objects over $S$ and $S'$ that are no longer under any flatness hypotheses.
    In any case, there is a flatness hypothesis involved somewhere.
    This is one of the main reasons for the importance of the notion of flatness in algebraic geometry (whose role could not be visible when we restricted to base \emph{fields}, over which everything, in fact, is flat!).
\end{remark}

\subsection{Application to étalements}\label{fga3.i-a.3}

Let $A$ be a local ring, and $B$ a local algebra over $A$ whose maximal ideal induces that of $A$.
We say that $B$ is \emph{étalé} over $A$ (instead of "unramified", as used elsewhere) if it satisfies the following conditions:

\begin{enumerate}[i.]
    \item $B$ is flat over $A$; and
    \item $B/\mathfrak{m}B$ is a separable finite extension of $A/\mathfrak{m}=k$ (where $\mathfrak{m}$ denotes the maximal ideal of $A$).
\end{enumerate}

If $A$ and $B$ are Noetherian, and $k$ is algebraically closed, then this implies that the homomorphism $\overline{A}\to\overline{B}$ between the completions that extends $A\to B$ is an isomorphism.
A morphism $f\colon T\to S$ of finite type is said to be \emph{étale at $x\in T$}, or $T$ is said to be \emph{étalé over $S$ at $x$}, if $\mathcal{O}_x$ is étalé over $\mathcal{O}_{f(x)}$;
$f$ is said to be \emph{étale}, or an \emph{étalement}, or $T$ is said to be \emph{étalé over $S$}, if $f$ is étale at all $x\in T$.
Note that, if $S$ is locally Noetherian, then the set of points of $T$ where $f$ is étale is open, and Zariski's "main theorem" allows us to precisely state the structure of $T/S$ in a neighbourhood around such a point (by an equation of well-known type).


If $S$ is a scheme of finite type over the field of complex numbers, then there exists a corresponding analytic space $\overline{S}$ (in the sense of Serre \cite{Ser1956}), except for the fact that $\overline{S}$ can have nilpotent elements in its structure sheaf, which changes nothing essential in \cite{Ser1956}.
We then easily see that $f$ is an étalement if and only if $\overline{f}\colon\overline{T}\to\overline{S}$ is an étalement, i.e. if every point of $\overline{T}$ admits a neighbourhood on which $\overline{f}$ induces an isomorphism onto an open subset of $\overline{S}$.
In particular, to every \emph{étale covering} $T$ of $S$ (i.e. every finite étale morphism $f\colon T\to S$), there is a corresponding étale covering $\overline{T}$ of $\overline{S}$, which is connected if and only if $T$ is connected \cite{Ser1956}.
We can also easily see that, if $T$ and $T'$ are étale schemes over $S$, then the natural map
\[\operatorname{Hom}_S(T,T') \to \operatorname{Hom}_{\overline{S}}(\overline{T},\overline{T}'')\]
is bijective, i.e. the functor $T\mapsto\overline{T}$ from the category of étale schemes over $S$ to the category of analytic spaces over $S$ is "fully faithful", and thus defines an equivalence between the first category and a subcategory of the second.
A theorem of Grauert–Remmert \cite{GR1958} implies that, if $S$ is normal, then we thus obtain an equivalence between the category of \emph{étale coverings} of $S$ and the category of (\emph{finite}) étale coverings of $S$, i.e. every étale covering $\mathcal{C}$ of $\overline{S}$ is $\overline{S}$-isomorphic to some $\overline{T}$, where $T$ is an étale covering of $S$.
We will show that \emph{the Grauert–Remmert theorem remains true without any normality hypotheses on $S$}.
First let $S'\to S$ be a finite strict epimorphism, and suppose that the theorem has been proven for $S'$; we will show that it holds for $S$.
Let $\mathcal{C}$ be an étale covering of $\overline{S}$, and consider its inverse image $\mathcal{C}'$ over $S'$, which corresponds to a coherent analytic sheaf $\mathfrak{A}'$ of algebras on $S'$ that is the inverse image of a sheaf of algebras $\mathfrak{A}$ on $\overline{S}$ defining $\mathcal{C}$.
By hypothesis, on $S'$, $\mathcal{C}'$ comes from an étale covering $T'$ of $S'$, i.e. $\mathfrak{A}'$ comes from a coherent sheaf of algebras $\mathcal{A}'$ on $S'$.
Also, $\mathfrak{A}'$ is endowed with a canonical descent data with respect to $\overline{S}'\to\overline{S}$, i.e. with an isomorphism between its two inverse images on $\overline{S}'\times_{\overline{S}}\overline{S}'=\overline{S'\times_SS'}$ (satisfying conditions (i) and (ii)), and this isomorphism comes from, by what has already been said, an isomorphism between the corresponding algebraic sheaves, i.e. from a descent data on $\mathcal{A}'$ with respect to $S'\to S$.
We can easily show that the latter is effective (since it is effective on $\mathfrak{A}'$, and the effectiveness of a descent data, as described explicitly in the previous section, is something that can be checked locally on the \emph{completions} of the modules that are involved).
From this, we obtain a coherent sheaf of algebras $\mathcal{A}$ on $S$ that defines a covering $T$ of $S$, and this is the desired covering.
The above result then obviously holds true if $S'\to S$ is just a composition of a finite number of finite strict epimorphisms, i.e. is just an arbitrary finite epimorphism (by the factorisation result stated in \Cref{fga3.i-a.2}).
It thus follows that the Grauert–Remmert theorem holds true if $S$ is a \emph{reduced} scheme, i.e. such that $\mathcal{O}_S$ has no nilpotent elements, as we can see by introducing its normalisation $S'$.
We can easily pass to the general case.



A completely analogous argument, again using the factorisation result for finite strict epimorphisms, and the "formal" nature of the effectiveness of descent data, allows us to prove the following result:
let $S$ be a locally Noetherian prescheme, and let $S'\to S$ be a finite, surjective, and radicial morphism (or, equivalently, a morphism of finite type such that, for every $T$ over $S$, the morphism $T'=S'\times_S T\to T$ is a homeomorphism, which we can also express by saying that $S'\to S$ is a \emph{universal homeomorphism}).
For every $T$ étalé over $S$, consider its inverse image $T'=T\times_S S'$, which is étalé over $S'$.
\emph{Then the functor $T\mapsto T'$ is an equivalence between the category of preschemes $T$ that are étalé over $S$ and the category of preschemes $T'$ that are étalé over $S'$.}
(We use the bijectivity of
\[
    \operatorname{Hom}_S(T_1,T_2) \to \operatorname{Hom}_{S'}(T'_1,T'_2)
\]
for preschemes $T_1$ and $T_2$ that are étalé over $S$, which can be proven directly without difficulty. We also use the fact that the stated theorem is true if $S'=(S,\mathcal{O}_S/\mathcal{J})$, where $\mathcal{J}$ is a nilpotent coherent sheaf of ideals of $\mathcal{O}_S$, cf. \cite{Mur1958}, Lemma 6].
Note also that we do not suppose here that the $T$ in question are finite over $S$.
This result implies, in particular, that the morphism $S'\to S$ induces an isomorphism between the fundamental group of $S'$ and the fundamental group of $S$ ("\emph{topological invariance of the fundamental group of a prescheme}").

\subsection{Relations to 1-cohomology}\label{fga3.i-a.4}

\label{fga3.i-a.4.a}
Let $\mathcal{C}$ be a category such that the product of any two objects always exists, and let $T\in\mathcal{C}$.
For every finite set $I\neq\varnothing$, we can consider $T^I$, and so we obtain a covariant functor from the category of non-empty finite sets to the category $\mathcal{C}$, i.e. what we can call a \emph{simplicial object} of $\mathcal{C}$, denoted by $K_T$.
This object depends covariantly on $T$;
also, \emph{if $u$ and $v$ are morphisms $T\to T'$, then the corresponding morphisms $K_T\to K_{T}$ are homotopic}.
We say that $T$ \emph{dominates} $T'$ if $\operatorname{Hom}(T,T')\neq\varnothing$, and this gives an (upward) directed preorder on $\mathcal{C}$.
It follows from the above that, if $T$ dominates $T'$, then there exists a canonical class (up to homotopy) of homomorphisms of simplicial objects $K_T\to K_{T'}$;
in particular, if $K_T$ and $K_{T'}$ are such that $T$ and $T'$ dominate one another, then $K_T$ and $K_{T'}$ are homotopically equivalent.
Now let $F$ be a (contravariant, to be clear) functor from $\mathcal{C}$ to an \emph{abelian} category $\mathcal{C}'$.
Then
\[
    C^\bullet(T,F) = F(K_T)
\]
is a cosimplicial object of $\mathcal{C}'$, and thus defines, in a well-known way, a (cochain) complex in $\mathcal{C}'$, of which we can take the cohomology:
\[
    \operatorname{H}^\bullet(T,F)
    = \operatorname{H}^\bullet(C^\bullet(T,F))
    = \operatorname{H}^\bullet(F(K_T))
\]
(we may write a subscript "$\mathcal{C}$" on the $\operatorname{H}^\bullet$ if there is any possibility for confusion).
This is a cohomological functor in $F$, of which the variance for $T$ varying follows from what has already been said about the $K_T$;
more precisely, for fixed $F$ and varying $T$ in $\mathcal{C}$ (preordered by the domination relation), the $\operatorname{H}^\bullet(T,F)$ form an inductive system of graded objects of $\mathcal{C}'$;
in particular, if $T$ and $T'$ are such that each one dominates the other, then $\operatorname{H}^\bullet(T,F)$ and $\operatorname{H}^\bullet(T',F)$ are canonically isomorphic.

Suppose that $\mathcal{C}$ has all fibre products.
Then we can, for fixed $S\in\mathcal{C}$, apply the above to the category $\mathcal{C}_S$ of objects of $\mathcal{C}$ over $S$;
we then write $C^\bullet(T/S,F)$ and $\operatorname{H}^\bullet(T/S,F)$ instead of $C^\bullet(T,F)$ and $\operatorname{H}^\bullet(T,F)$ if we wish to make clear that we are working in the category $\mathcal{C}_S$;
then $C^\bullet(T/S,F)$ is a cochain complex in $\mathcal{C}'$ that, in degree $n$, is equal to $F(T\times_S T\times_S\ldots\times_S T)$ (where there are $n+1$ factors $T$).


Note that, as per usual, we can define $\operatorname{H}^0(T/S,F)$ without assuming the category $\mathcal{C}'$ to be abelian:
it is the kernel (\Cref{fga3.i-a.2-definition-2.1}), if it exists, of the pair $(F(p_1),F(p_2))$ of morphisms
\[
    F(T) \to F(T\times_S T)
\]
corresponding to the two projections $p_1,p_2\colon T\times_S T\to T$.
In particular, we then have the natural morphism (called the \emph{augmentation})
\[
    F(S) \to \operatorname{H}^0(T/S,F)
\]
which is an isomorphism in nice cases (in particular, in the case where $T\to S$ is a strict epimorphism and $F$ is "left exact").
Similarly, if $F$ takes values in the category of groups in a category $\mathcal{C}''$, then we can also define $\operatorname{H}^1(T/S,F)$;
in the case where $\mathcal{C}''$ is the category of sets (i.e. when $F$ takes values in the category of non-necessarily-commutative groups), $\operatorname{H}^1(T,F)$ is the quotient of the subgroup $Z^1(T/S,F)$ of $C^1(T/S,F) = F(T\times_S T)$ consisting of the $g$ such that
\[
    F(p_{31})(g) = F(p_{32})(g) F(p_{21})(g)
\]
by the group with operators $F(T)$ acting on $C^1(T/S,F)$, and thus, in particular, on the subset $Z^1(T/S,F)$, by
\[
    \rho(g')\cdot g = F(p_2)(g') g F(p_1)(g')^{-1}
\]

\label{fga3.i-a.4.b}
For example, let $\mathcal{F}$ be a fibred category with base $\mathcal{C}$.
Let $\xi,\eta\in\mathcal{F}_S$, and, for all $S'$ over $S$, let
\[
    F_{\xi,\eta}(S')
    = \operatorname{Hom}(\xi\times_S S', \eta\times_S S').
\]
Then $F_{\xi,\eta}$ is a contravariant functor from $\mathcal{C}_S$ to the category of sets.
With this setup, \emph{saying that the augmentation morphism}
\[
    F_{\xi,\eta}(S) \to \operatorname{H}^0(S'/S,F_{\xi,\eta})
\]
\emph{is an isomorphism for every pair of elements $\xi,\eta\in\mathcal{F}_S$ implies that $\alpha\colon S'\to S$ is an $\mathcal{F}$-descent morphism} (\Cref{fga3.i-a.1-definition-1.7}).

\label{fga3.i-a.4.c}
Similarly, for $\xi\in\mathcal{F}_S$ and any object $S'$ of $\mathcal{C}$ over
\[
    G_\xi(S') = \operatorname{Aut}(\xi\times_S S'),
\]
we thus define a contravariant functor $G_\xi$ from $\mathcal{C}_S$ to the category of groups.
With this setup, we claim that \emph{$Z^1(S'/S,G)$ is canonically identified with the set of descent data on $\xi'=\xi\times_S S'$ with respect to $S'\to S$} (\Cref{fga3.i-a.1-definition-1.6}), and that \emph{$\operatorname{H}^1(S'/S,G)$ can be identified with the set of isomorphism classes of objects of $\mathcal{F}_{S'}$ endowed with a descent data relative to $\alpha\colon S'\to S$ that are isomorphic, as objects of $\mathcal{F}_{S'}$, to $\xi'=\xi\times_S S'$}.
Then, if $\alpha\colon S'\to S$ is an $\mathcal{F}$-descent morphism \emph{(cf. \Cref{fga3.i-a.4.b})}, then $\operatorname{H}^1(S'/S,G)$ contains as a subset the set of isomorphism classes of objects $\eta$ of $\mathcal{F}_S$ such that $\eta\times_S S'$ is isomorphic (in $\mathcal{F}_{S'}$) to $\xi\times_S S'$;
further, \emph{this inclusion is the identity if and only if every descent data on $\xi'=\xi\times_S S'$ with respect to $\alpha\colon S'\to S$ is effective}.
(This will be the case, in particular, if $\alpha\colon S'\to S$ is a strict $S$-descent morphism).

\begin{remark}\label{fga3.i-a.4.c-remark}
    The cochain complexes of the form $C^\bullet(T/S,F)$ contain, as particular cases, the majority of standard known complexes (that of Čech cohomology, of group cohomology, etc.), and play an important role in algebraic geometry (notably in the "Weil cohomology" of preschemes).
\end{remark}

\label{fga3.i-a.4.d}
\begin{example}\label{fga3.i-a.4-example-1}
    Let $S'$ be an object over $S\in\mathcal{C}$, and let $\operatorname{\Gamma}$ be a group of automorphisms of $S'$ such that $S'$ is "formally $\operatorname{\Gamma}$-principal over $S$", i.e. such that the natural morphism
    \[
        \operatorname{\Gamma}\times S' \to S'\times_S S'
    \]
    (where $\operatorname{\Gamma}\times S'$ denotes the direct sum of $\operatorname{\Gamma}$ copies of $S'$) is an isomorphism.
    (We suppose that all the necessary direct sums exist in $\mathcal{C}$).
    Let $F$ be a contravariant functor from $\mathcal{C}$ to the category of abelian groups.
    Then \emph{$C^\bullet(S'/S,F)$ is canonically isomorphic to the simplicial group $C^\bullet(\operatorname{\Gamma},F(S'))$ of standard homogeneous cochains, and so $\operatorname{H}^\bullet(S'/S,F)$ is canonically isomorphic to $\operatorname{H}^\bullet(\operatorname{\Gamma},F(S'))$}.
\end{example}

\label{fga3.i-a.4.e}
\begin{example}\label{fga3.i-a.4-example-2}
    Let $\mathcal{C}$ be the category of preschemes.
    We denote by $\operatorname{G_a}$ (for "additive group") the contravariant functor from $\mathcal{C}$ to the category of abelian groups, defined by
    \[
        \operatorname{G_a}(X) = \operatorname{H}^0(X,\mathcal{O}_X)
    \]
    We similarly define the functor $\operatorname{G_m}$ (for "multiplicative group") by
    \[
        \operatorname{G_m}(X) = \operatorname{H}^0(X,\mathcal{O}_X)^\times
    \]
    (i.e. the group of invertible elements of the ring $\operatorname{H}^0(X,\mathcal{O}_X)$), and, more generally, the functor $\operatorname{GL}(n)$ (for "linear group of order $n$") by
    \[
        \operatorname{GL}(n)(X) = \operatorname{GL}(n,\operatorname{H}^0(X,\mathcal{O}_X)),
    \]
    which is a functor from $\mathcal{C}$ to the category of (not-necessary-commutative, if $n>1$) groups;
    for $n=1$ we recover $\operatorname{G_m}$.
    We can also think of $\operatorname{GL}(n)$ as an automorphism functor (cf. \Cref{fga3.i-a.4.c}) by considering the fibred category $\mathcal{F}$ with base $\mathcal{C}$ such that $\mathcal{F}_X$ is the category of locally free sheaves on $X$, for $X\in\mathcal{C}$, since then $\operatorname{GL}(n)(X)=\operatorname{Aut}_{\mathcal{F}_X}(\mathcal{O}_X^n)$.
    By \Cref{fga3.i-a.4.b}, it follows that, if $\alpha\colon S'\to S$ is an $\mathcal{F}$-descent morphism (cf. \Cref{fga3.i-a.2.c}), then $\operatorname{H}^1(S'/S,\operatorname{GL}(n))$ contains the set of isomorphism classes of locally free sheaves on $S$ whose inverse image on $S'$ is isomorphic to $\mathcal{O}_{S'}^n$, and this inclusion is an equality if and only if every descent data on $\mathcal{O}_{S'}^n$ (with respect to $\alpha\colon S'\to S$) is effective.
    If $S$ is the spectrum of a local ring, then this implies that $\operatorname{H}^1(S'/S,\operatorname{GL}(n))=(e)$, since every locally free sheaf on $S$ is then trivial.
\end{example}

We note that the following conditions concerning a morphism $\alpha\colon S'\to S$ are equivalent:
\begin{enumerate}[i.]
    \item The augmentation homomorphism $\operatorname{H}^0(S,\mathcal{O}_S) = \operatorname{G_a}(S)\to\operatorname{H}^0(S'/S,\operatorname{G_a})$ is an isomorphism.
    \item $\alpha\colon S'\to S$ is an $\mathcal{F}$-descent morphism (where $\mathcal{F}$ is the fibred category over $\mathcal{C}$ described above).
    \item $\alpha\colon S'\to S$ is a strict epimorphism (cf. \Cref{fga3.i-a.2.c}).
\end{enumerate}

Now suppose that $S=\operatorname{Spec}(A)$ and $S'=\operatorname{Spec}(A')$;
then
\[
    C^n(S'/S,\operatorname{G_a})
    = C^n(A'/A,\operatorname{G_a})
    = \underbrace{A'\otimes_A A'\otimes_A\ldots\otimes_A A'}_{n+1\text{ copies of }A'}
\]
with the coboundary operator $C^n(A'/A,\operatorname{G_a})\to C^{n+1}(A'/A,\operatorname{G_a})$ being the alternating sum of the face operators
\[
    \partial_i(x_0\otimes x_1\otimes\ldots\otimes x_n)
    = x_0\otimes\ldots\otimes x_{i-1}\otimes1_{A'}\otimes x_i\otimes\ldots\otimes x_n.
\]
Similarly, $C^n(S'/S,\operatorname{G_m})=C^n(A'/A,\operatorname{G_m})$ can be identified with $(\bigotimes_A^{n+1}A')^\times$, with the simplicial operations for $C^\bullet(A'/A,\operatorname{G_m})$ being induced by those in $C^\bullet(S'/S,\operatorname{G_a})$.
We can write down the simplicial operations for $C^\bullet(A'/A,\operatorname{GL}(n))$ in the same explicit manner.
\emph{In all the cases known to the speaker, $\operatorname{H}^i(A'/A,\operatorname{G_a})=0$ for $i>0$, and, if $A$ is local, then $\operatorname{H}^1(A'/A,\operatorname{G_m})=0$, and, more generally, $\operatorname{H}^1(A'/A,\operatorname{GL}(n))=(e)$} (if $S'\to S$ is an $\mathcal{F}$-descent morphisms, i.e. if the diagram $A\to A'\rightrightarrows A'\otimes_A A'$ is exact, then compare this with \Cref{fga3.i-a.2.c}).
We note that \emph{Hilbert's "Theorem 90" is exactly the fact that $\operatorname{H}^1(S'/S,\operatorname{G_m})=0$ if $A$ is a field and $A'$ is a finite Galois extension of $A$} (cf. \Cref{fga3.i-a.4-example-1}), \emph{and we can also express it by saying that, in the case in question, $S'\to S$ is a strict descent morphisms for the fibred category of locally free sheaves of rank $1$}.
This latter statement is the one that is most suitable to generalise Hilbert's theorem, by varying the hypotheses both on the morphism $S'\to S$ and on the quasi-coherent sheaves in question.

Finally, we note that, for a local \emph{Artinian} $A$ with maximal ideal $\mathfrak{m}$, and an $A$-algebra $A'$ (where we denote, for any integer $k>0$, the ring $A/\mathfrak{m}^{k+1}$ (resp. $A'/\mathfrak{m}^{k+1}A'$) by $A_k$ (resp. $A'_k$)), the following properties are equivalent:

\begin{enumerate}[i.]
    \item  $\operatorname{H}^1(A'_k/A_k,\operatorname{G_a})=0$ for all $k$.
    \item $\operatorname{H}^1(A'_k/A_k,\operatorname{G_m})=0$ for all $k$.
    \item $\operatorname{H}^1(A'_k/A_k,\operatorname{GL}(n))=(e)$ for all $k$ and all $n$.
\end{enumerate}

If $S'\to S$ is a strict epimorphism, then the above conditions imply that it is a \emph{strict} descent morphism for free modules (not necessarily of finite type) over $A'$.

\begin{remark}\label{fga3.i-a.4.e-remark}
    The definition of the groups $\operatorname{H}^i(S'/S,\operatorname{G_m})$ in the case where $S$ (resp. $S'$) is a scheme over the field $A$ (resp. $A'$) is due to Amitsur.
    The group $\operatorname{H}^2(S'/S,\operatorname{G_m})$ is particularly interesting as a "global" variant of the Brauer group, for which we can refer to \cite{GD1960}, VII.
\end{remark}
