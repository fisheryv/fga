% !TeX root = ../../fga.tex
\section{The canonical abelian subscheme of $\underline{\operatorname{Pic}}_{X/S}$, and the Albanese scheme}\label{fga3.vi-3}


\begin{proposition}\label{fga3.vi-3-proposition-3.1}
    Let $k$ be a field, and let $G$ a group scheme of finite type over $k$ that is commutative and "without additive component".
    Then $G_\mathrm{red}^0$ is separable over $k$, and thus a simple group scheme over $k$.
\end{proposition}

\begin{cproof}
    Since the claim is trivial if $k$ is perfect, and in particular for $G_{\overline{k}}$, where $\overline{k}$ is the algebraic closure of $k$, it suffices to show that $(G_{\overline{k}}^0)_\mathrm{red}$ comes from a subscheme of $G$.
    But from the hypothesis that $G_{\overline{k}}$ contains no additive component it easily follows that there exists an integer $m$ such that $(G_{\overline{k}}^0)_\mathrm{red}$ is the "scheme-theoretic" image of the $m$-th power homomorphism in $G_{\overline{k}}$.
    Since the latter homomorphism comes from the analogous homomorphism for $G$, the scheme-theoretic image of this provides the desired object.
\end{cproof}

\begin{corollary}\label{fga3.vi-3-corollary-3.2}
    Let $X$ be a normal and proper scheme over $k$ such that $\underline{\operatorname{Pic}}_{X/k}$ exists.
    Then there exists an abelian subscheme of $\underline{\operatorname{Pic}}_{X/k}$ whose underlying set is $\underline{\operatorname{Pic}}_{X/k}^0$.
\end{corollary}

\begin{cproof}
    By (ii) of \Cref{fga3.vi-2-theorem-2.1}, since $\underline{\operatorname{Pic}}_{X/k}^0$ is proper over $k$, it satisfies the conditions of \Cref{fga3.vi-3-proposition-3.1}.
\end{cproof}


The above result shows that, in certain cases, the classical "Picard variety" (which is $(\underline{\operatorname{Pic}}_{X/\overline{k}})_\mathrm{red}^0$ in the current theory) "is defined over $k$", without supposing the field $k$ to be perfect.

Now let $f\colon X\to S$ be a proper and flat relative scheme, with $\mathcal{O}_S\xrightarrow{\sim} f_*(\mathcal{O}_X)$ for simplicity, such that $\underline{\operatorname{Pic}}_{X/S}$ exists and that $\underline{\operatorname{Pic}}_{X/S}^0$ is proper over $S$.
Suppose further, for (ii) of \Cref{fga3.vi-3-theorem-3.3}, that there exists an open of $\underline{\operatorname{Pic}}_{X/S}$ containing $\underline{\operatorname{Pic}}_{X/S}^0$ that is quasi-projective over $S$;
this condition is satisfied, as we have seen, if $f$ is projective and with separable or irreducible geometric fibres.
Recall that an \emph{abelian scheme} over $S$ is a group scheme over $S$ that is proper and simple over $S$ with connected geometric fibres.
We propose to examine whether or not there exists a group subscheme $A$ of $\underline{\operatorname{Pic}}_{X/S}$ that is an abelian scheme and whose underlying set is $\underline{\operatorname{Pic}}_{X/S}^0$.
We have just seen that such an $A$ always exists if $S$ is the spectrum of a field.
Here is what we know how to say in the general case envisaged here:


\begin{theorem}\label{fga3.vi-3-theorem-3.3}
    Under the above conditions:

    \begin{enumerate}[i.]
        \item If there exists an abelian subscheme of $\underline{\operatorname{Pic}}_{X/S}$ whose underlying set is $\underline{\operatorname{Pic}}_{X/S}^0$, then it is unique.
              Its formation is thus compatible with base change.
        \item For there to exist such an abelian subscheme, it is necessary and sufficient that it exist after every base change $S'\to S$, where $S'$ is local Artinian;
              if $S$ is the spectrum of a local ring, it even suffices to test with the $S'=\operatorname{Spec}(A_n)$ where $A_n=A/\mathfrak{m}^{n+1}$.
              If $S$ is reduced, then it equally suffices to test with the $S'$ that are the spectrum of a discrete valuation ring (complete, with algebraically closed residue field, if one desires).
        \item Suppose that $A$ exists, and let $B=\operatorname{Alb}^0(X/S)$ be the dual abelian scheme (i.e. $B=\operatorname{Pic}_{A/S}^0$ [Mumford–Tate seminar, 1962]).
              Then we can canonically construct a principal homogeneous space $P=\operatorname{Alb}^1(X/S)$ for $B$, and an $S$-morphism $X\to P$ that is universal for the $S$-morphisms from $X$ to para-abelian schemes (i.e. to principal homogeneous spaces for abelian schemes).
              The formation of $\operatorname{Alb}^0(X/S)$, $\operatorname{Alb}^1(X/S)$, and $X\to\operatorname{Alb}^1(X/S)$ commutes with base change.
    \end{enumerate}
\end{theorem}

\begin{cproof}
    We sketch the proof:
    \begin{enumerate}
        \item This is a general property of rigidity for abelian subschemes of commutative group schemes: if two such subschemes agree set-theoretically at a point $s\in S$, then they agree over the entire connected component of $s$ ([Mumford–Tate seminar, 1962]).
              (This result generalises a classical theorem of Chow).
        \item Using Hilbert schemes, we see that the functor that, to every $S'$ over $S$ associates the set (consisting of either one or zero elements) of canonical abelian subschemes of $(\underline{\operatorname{Pic}}_{X/S})\times_S S'$ is representable by a scheme $T$ of finite type over $S$.
              By (i), $T\to S$ is a monomorphism, and by \Cref{fga3.vi-3-corollary-3.2}, it is surjective.
              To say that there exists a canonical abelian subscheme of $\underline{\operatorname{Pic}}_{X/S}$ implies that $T$ is a section over $S$, or that $T\to S$ is an isomorphism.
              This is equivalent to saying that $T\to S$ is étale, or, in the case where $S$ is reduced, that $T\to S$ is proper.
              Whence immediately (ii).
        \item Simply using the definition of $\underline{\operatorname{Pic}}_{X/S}$, we note that, for every abelian scheme $C$ over $S$, the data of an $S$-morphism from $X$ to a principal homogeneous space for $C$ is equivalent to the data of a group homomorphism $C'\to\underline{\operatorname{Pic}}_{X/S}$, where $C'$ is the dual abelian scheme of $C$.
              But if the canonical abelian subscheme $A$ of $\underline{\operatorname{Pic}}_{X/S}$ exists, then these homomorphisms necessarily factor through $A$ (and we can see by again using the points of finite order).
              Whence immediately (iii).
    \end{enumerate}
\end{cproof}

\begin{remark}\label{fga3.vi-3-remarks-3.4}
    We denote by $\underline{\operatorname{Pic}}_{X/S}^{00}$ the canonical abelian subscheme of $\underline{\operatorname{Pic}}_{X/S}$, if it exists.
    This is, unfortunately, not always the case, as we can see by infinitesimally varying the Igusa surface (by first-order modular deformation).
    It is however possible that $\underline{\operatorname{Pic}}_{X/S}^{00}$ exists at least if $S$ is reduced, or, equivalently, by (ii), if $S$ is the spectrum of a discrete valuation ring.
    So let $X_0$ and $X_1$ be the special and generic fibres of $X$ (respectively), and let $A_1=\underline{\operatorname{Pic}}_{X_1/K_1}$, where $K$ is the field of fractions of the valuation ring $V$.
    By Koizumi, there exists an abelian scheme $A$ over $S$, essentially unique, whose general fibre is $A_1$, and we easily see as in [(2.1), (i)] \Cref{fga3.vi-2-theorem-2.1} (supposing from now on that $X$ is simple over $S$) that the identity morphism of $A_1$ extends to a morphism
    \[
        A \to \underline{\operatorname{Pic}}_{X/S}.
    \]
    From this, we obtain a homomorphism

    \begin{equation}\tag{*}\label{fga3.vi-3-equation-star}
        A_0 \to \underline{\operatorname{Pic}}_{X_0/k}^{00}
    \end{equation}

    which we can easily show to be a surjective homomorphism with kernel equal to a finite $p$-group, where $p$ is the characteristic of the residue field $k$ (still by using the points of finite order).
    With this, the following conditions on $X/S$ are equivalent:

    \begin{enumerate}[label=\alph*.]
        \item The above homomorphism \Cref{fga3.vi-3-equation-star} is an isomorphism (which Shimura expresses by saying that the formation of the "Picard variety" is "compatible with specialisations").
        \item $\underline{\operatorname{Pic}}_{X/S}^{00}$ exists (and is then exactly $A$).
        \item (As a reminder) The $\underline{\operatorname{Pic}}_{X_n/S}^{00}$ exist.
    \end{enumerate}


    By the remark that we made concerning the kernel of \Cref{fga3.vi-3-equation-star}, condition (a) is satisfied if the residual characteristic is zero, but this result will be notably generalised in \Cref{fga3.vi-3-theorem-3.5}.

    Of course, if $\underline{\operatorname{Pic}}_{X/S}$ is simple over $S$ at the points of $\underline{\operatorname{Pic}}_{X/S}^0$, then the latter is open in $\underline{\operatorname{Pic}}_{X/S}$ (cf. \Cref{fga3.vi-1-corollary-1.7}) and is thus, endowed with the induced structure, an abelian subscheme of $\underline{\operatorname{Pic}}_{X/S}$, and thus equal to $\underline{\operatorname{Pic}}_{X/S}^{00}$, which exists in this case.
    But we have much better:
\end{remark}

\begin{theorem}\label{fga3.vi-3-theorem-3.5}
    Under the conditions of \Cref{fga3.vi-3-theorem-3.3}, let $s\in S$ be such that $\underline{\operatorname{Pic}}_{X_s/k(s)}$ is simple over $k(s)$ (or, equivalently, such that $\dim\underline{\operatorname{Pic}}_{X_s/k(s)}=\dim\operatorname{H}^1(X_s,\mathcal{O}_{X_s})$).
    Then there exists an open neighbourhood $U$ of $s$ such that $\underline{\operatorname{Pic}}_{X/S}$ is simple over $S$ at the points of $\underline{\operatorname{Pic}}_{X/S}^0|U$, which is thus an open abelian subscheme in $\underline{\operatorname{Pic}}_{X/S}|U$.
    A fortiori, $\underline{\operatorname{Pic}}_{X|U/U}^{00}$ exists.
\end{theorem}

\begin{cproof}
    We describe the principle of the proof.
    The above allows us to reduce to the case where $S$ is the spectrum of an Artinian local ring $A$, and we argue by induction on the infinitesimal order of $A$.
    We can thus suppose that $\underline{\operatorname{Pic}}_{X_n/A_n}^0$ is simple over $A_n$, and reduce to proving that $\underline{\operatorname{Pic}}_{X_{n+1}/A_{n+1}}^0$ is simple over $A_{n+1}$.
    Note that, for this, it suffices to construct an abelian scheme $P_{n+1}$ over $A_{n+1}$ that extends $P_n=\underline{\operatorname{Pic}}_{X_n/A_n}^0$, along with an invertible module $\mathcal{L}_{n+1}$ on $X_{n+1}\times_{A_{n+1}}P_{n+1}$ that extends the invertible module $\mathcal{L}_n$ on $X_n\times_{A_n}P_n$ that arises in the definition of the Picard scheme $\underline{\operatorname{Pic}}_{X_n/A_n}$ as the solution to a universal problem.
    (N.B. We can suppose that $X$ is endowed with a section over $S$...).
    For this construction, we must use the following key result: \emph{every abelian scheme defined over a quotient of an Artinian local ring can be extended} (in other words, the absolute "formal scheme of modules" (\Cref{fga3.ii}) for an abelian scheme over an algebraically closed field is simple over the ring of Witt vectors over $k$);
    this result can be obtained by using the general formal properties of the obstruction to lifting, and the group operations.
    With this result, we start by extending $P_n$ arbitrarily to $P_{n+1}$;
    we then find an obstruction to lifting $\mathcal{L}_n$, found in $\operatorname{H}^2(X_0\times P_0,\mathcal{O}_{X_0\times P_0})\otimes_k V$ (where $V=\mathfrak{m}^{n+1}/\mathfrak{m}^{n+2}$), and more precisely in the subspace $\operatorname{H}^1(X_0,\mathcal{O}_{X_0})\otimes\operatorname{H}^1(P_0,\mathcal{O}_{P_0})\otimes_k V$ (taking into account the fact that the restriction of $\mathcal{L}_n$ to the two factors $X_n$ and $P_n$ is trivial).
    But this latter space is exactly $\operatorname{H}^1(P_0,\mathcal{G}_{P_0/k})\otimes V$, where $\mathcal{G}_{P_0/k}$ is the tangent sheaf to $P_0/k$, and thus also the space that expresses the indeterminacy that there was in the lifting of $P_n$ to $P_{n+1}$ (\Cref{fga3.ii}).
    So we can correct this lifting (in exactly one way, as should be the case) in such a way as to kill the obstruction to lifting $\mathcal{L}_n$.
\end{cproof}

\begin{corollary}\label{fga3.vi-3-corollary-3.6}
    Under the conditions of \Cref{fga3.vi-3-theorem-3.5}, $\operatorname{R}^1 f_*(\mathcal{O}_X)$ is a locally free module on $S$ in a neighbourhood of $s$, and its formation commutes with base change.
\end{corollary}

\begin{cproof}
    This module is exactly the tangent module to $\underline{\operatorname{Pic}}_{X/S}$ along the identity section.
\end{cproof}

\begin{remark}\label{fga3.vi-3-remark-3.7}
    Using the same argument as for \Cref{fga3.vi-3-theorem-3.5}, we can show that, if $S'$ is a subscheme of $S$ defined by a nilpotent coherent ideal, and if we suppose only that $\underline{\operatorname{Pic}}_{X'/S'}^0$ exists and has simple fibres, then $\underline{\operatorname{Pic}}_{X/S}^0$ necessarily exists and is an abelian scheme over $S$.
    This allows us to construct Picard schemes in certain cases, despite the absence of any projective hypothesis;
    for example, the dual abelian scheme of an abelian scheme over an Artinian ring always exists.
    Using \Cref{fga3.vi-3-corollary-3.6} in the case where $X$ is an abelian scheme over $S$, and using the known structure of $\operatorname{H}^\bullet(X_s,\mathcal{O}_{X_s})$ as an exterior algebra over $\operatorname{H}^1(X_s,\mathcal{O}_{X_s})$ (Rosenlicht–Serre), we find that $\operatorname{R}^if_*(\mathcal{O}_X)$ is locally free for \emph{all} $i$, and more precisely that it is isomorphic to the $i$-th exterior power of $\operatorname{R}^1f_*(\mathcal{O}_X)$.
\end{remark}

\begin{remark}\label{fga3.vi-3-remark-3.8}
    In the case of a simple projective morphism $f\colon X\to S$, with $S$ reduced and of residual characteristics zero, the result of \Cref{fga3.vi-3-corollary-3.6} was already known, by transcendental methods, as a consequence of Hodge theory.
    In fact, all the $\operatorname{R}^pf_*(\Omega_{X/S}^q)$ are then locally free.
    We have, however, counterexamples in the case of unequal characteristics for "locally free $\operatorname{R}^1f_*(\mathcal{O}_X)$", by Serre varieties (\cite{Ser1958b}, p.50).
    It seems that we do not have any counterexample in equal characteristic.
\end{remark}

\begin{corollary}\label{fga3.vi-3-corollary-3.8}
    Under the conditions of \Cref{fga3.vi-3-theorem-3.5}, $\underline{\operatorname{Pic}}_{X/S}|U$ is simple over $U$ at all points of $\underline{\operatorname{Pic}}_{X/S}^\sigma|U$.
\end{corollary}

\begin{cproof}
    We apply (i) of \Cref{fga3.vi-2-proposition-2.10}.
\end{cproof}


In particular, taking (iii) of \Cref{fga3.vi-2-proposition-2.10} into account:


\begin{corollary}\label{fga3.vi-3-corollary-3.9}
    Suppose that $S$ is of residual characteristics zero.
    Then $\underline{\operatorname{Pic}}_{X/S}^\tau$ is simple over $S$.
\end{corollary}

\begin{remark}\label{fga3.vi-3-remark-3.10}
    We thus deduce, for example, that if $\underline{\operatorname{Pic}}_{X/S}$ is also \emph{proper} over $S$, then the Néron–Severi torsion group of geometric fibres of $f$ is constant on every connected component of $S$ (which is also evident by transcendental methods when $f$ is simple and projective).
    We note that the direct use of \Cref{fga3.vi-2-theorem-2.5} allows us to show, more generally, that, if $\underline{\operatorname{Pic}}_{X/S}$ is proper over $S$ (for example, if $f\colon X\to S$ is simple and projective), and if $q$ is a prime number distinct from the residual characteristics of $S$, then the $q$-primary component of the Néron–Severi torsion groups of geometric fibres of $X/S$ is constant on every connected component of $S$.
    It is no longer, however, the case in characteristic $p>0$ for the $p$-primary component of the torsion group.
    However, it remains possible that the rank over the $k(s)$ of $\underline{\operatorname{Pic}}_{X_s/k(s)}/\underline{\operatorname{Pic}}_{X_s/k(s)}^{00}=T_{X_s/k(s)}$ is locally constant;
    when $S$ is reduced, we  can show that this is equivalent to showing that $\underline{\operatorname{Pic}}_{X/S}^{00}$ exists and that $\underline{\operatorname{Pic}}_{X/S}$ is \emph{flat} over $S$, and it suffices to test in the cases where $S$ is the spectrum of a discrete valuation ring.
    This is what I have verified in the few examples that I have looked at;
    but since the corresponding statement with $S$ Artinian is false (cf. \Cref{fga3.vi-2-remark-2.9} and \Cref{fga3.vi-3-remarks-3.4}), we must not get carried away.
\end{remark}