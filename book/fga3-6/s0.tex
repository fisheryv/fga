% !TeX root = ../../fga.tex
\section{Supplements to the previous talk (FGA 3.V)}\label{fga3.iv-0}

There has been some progress concerning the questions of existence of Picard preschemes raised in \Cref{fga3.v}:

\begin{enumerate}[label=\alph*.]
    \item (Mumford).
          It is not true in general that, if $f\colon X\to S$ is a projective and separable (i.e. flat with separable fibres) morphism, then the Picard prescheme $\underline{\operatorname{Pic}}_{X/S}$ exists, even if the fibres of $f$ are of dimension $1$ and $S$ is the spectrum of a complete discrete valuation ring.
          A counterexample is given by taking $S=\operatorname{Spec}\mathbb{R}[{[t]}]$, and taking $X$ to be the subscheme of $\mathbb{P}_S^2$ (with homogeneous variables $x,y,z$) defined by the equation $x^2+y^2=tz^2$, which represents a conic degenerating to two geometrically concurrent lines, but the special fibre over the field $\mathbb{R}$ is nevertheless irreducible (it is given by the equation $x^2+y^2=0$ in $\mathbb{R}$).
          We easily see that, after the étale extension $S'\to S$, with $S'=\operatorname{Spec}\mathbb{C}[{[t]}]$, the Picard prescheme of $X'/S'$ exists, and we thus obtain an explicit description of it as a sum of copies of $\widetilde{S}$, where $\widetilde{S}$ is induced by $S$ by copying the origin an infinite number of times.
          We easily observe that the descent data on $\underline{\operatorname{Pic}}_{X'/S'}$ for $S'\to S$ (given here by the actions of the Galois group $\mathbb{Z}/2\mathbb{Z}$ of $S'$ over $S$) is not effective, since the group permutes certain doubled points (so that there are orbits that are not contained in an affine open).
          However, Mumford has shown that, if $f\colon X\to S$ is a separable projective morphism such that, for all $s\in S$, the irreducible components of $X_s$ are geometrically irreducible with respect to $k(s)$, then $\underline{\operatorname{Pic}}_{X/S}$ exists;
          the proof relies on a refinement of his theorem of passage to the quotient, cf. [Mumford–Tate seminar, 1962].
          Note also that it is still possible that, without any hypotheses on the irreducible components of the fibres $X_s$, the scheme $\underline{\operatorname{Pic}}_{X/S}^\tau$ (which will be introduced below) still exists.
    \item (Murre).
          If $X$ is a proper scheme over a field, then $\underline{\operatorname{Pic}}_{X/k}$ exists.
          The proof partially uses the proof of Chevalley \cite{Che1960}, and is fundamentally based on the group structure of the Picard functor.
\end{enumerate}


For certain additional comments concerning the theory of Picard schemes, most notably in relation to abelian schemes, we recommend consulting [Mumford–Tate seminar, 1962].
Finally, a notable shortcoming of the present talk is the absence of "equivalence criteria" that would allow us to compare the Picard scheme of a projective scheme and of its hyperplane sections;
the key theorems for developing such criteria can be found in \cite{Gro1960b}, with which one must combine the existence theorems for Picard schemes.
