% !TeX root = ../../fga.tex
\section{Applications to some particular cases}\label{fga3.ii-c}

\subsection{General remarks on functors represented by preschemes}\label{fga3.ii-c.1}

Let $S$ be a locally Noetherian prescheme.
A prescheme $X$ over $S$ is said to be \emph{locally of finite type} over $S$ if, for all $x\in X$ that project to $y\in Y$, there exists an affine neighbourhood of $y$ of ring $A$, and an affine neighbourhood of $x$ (over the aforementioned affine neighbourhood of $y$) of ring $B$, such that $B$ is an $A$-algebra of finite type.
There are many important examples of preschemes locally of finite type over $S$, that are not of finite type over $S$, given by solutions of classical universal problems;
thus it is important to be able to consider the Picard scheme of a curve as a union of infinitely-many connected components (that we must avoid confusing with the connected component of the identity element, i.e. the "Picard variety").
It is thus sometimes useful to place ourselves in the category $\mathcal{C}$ of preschemes locally of finite type over $S$, in order to study the question of representability of a contravariant functor $F$.
\emph{The main goal of these articles is to develop a general technique that allows us to recognise when such a functor $F$ is representable, and to study the properties of the corresponding $S$-prescheme $X$ by means of the properties of $F$.}
We note in passing that, in this study, we find non-pathological examples of preschemes over $S$ that are not separated over $S$, notably as "Picard preschemes" of excellent $S$-schemes;
we must thus refrain from banishing preschemes that are not schemes from algebraic geometry.

Let $X$ be a prescheme locally of finite type over $S$, and let
\[
    F\colon Y \mapsto \operatorname{Hom}_S(Y,X)
\]
be the associated contravariant functor.
We can consider the restriction $F_0$ of $F$ to the subcategory $\mathcal{C}_0$ of $\mathcal{C}$ consisting of preschemes $Y$ over $S$ that are Artinian and finite over $S$:
if $S=\operatorname{Spec}(\Lambda)$, then $\mathcal{C}_0$ is the category dual to the category of Artinian $\operatorname{G_a}mma$-algebras considered in \Cref{fga3.ii-b}.
If $Y=\operatorname{Spec}(A)$, where $A$ is a \emph{local} Artinian ring, then $Y$ consists of a single point $y$ living above a closed point $s$ of $S$, and an $S$-homomorphism from $Y$ to $X$ (i.e. an element of $F(Y)$) is defined by the data of a point $x\in X$ over $s$, along with an $\mathcal{O}_s$-homomorphism from $\mathcal{O}_x$ to $A$.
If there exists such a homomorphism, then $x$ is necessarily a closed point of $X$ (since its residue field is algebraic over the residue field of $s$).
This thus shows that \emph{the restriction $F_0$ of $F$ to "Artinian $Y$-algebras" is pro-representable, and is represented by the topological $Y$-algebra whose local components are the completions $\widehat{\mathcal{O}_x}$ of the local rings of $X$ at the points $x$ of $X$ that are closed and live above closed points of $Y$}.
This shows that only knowing $F_0$ gives precise information about the structure of $X$ (that is, the structure of the completions of its local rings at the aforementioned points).
We note that, even in the case where $S$ is the spectrum of an algebraically closed field, it is only thanks to the systematic consideration of "varieties" $Y$ such that $\mathcal{O}_Y$ may admit nilpotent elements (and, in particular, working with the spectra of local Artinian rings) that we can arrive at the "good formulation" of classical universal problems, and understand the "infinitesimal" aspect.


If we start with a given functor $F$, and we want to know whether or not it is representable, then studying the functor $F_0$ (using \Cref{fga3.ii-b-theorem-1} and \Cref{fga3.ii-b-theorem-2}) will give quasi-complete hints;
either, as is often the case (by simply testing, for example, the nature of the sets $F(I_K,\xi)$ and their functorial behaviour, cf. \Cref{fga3.ii-a}), $F_0$ is already not pro-representable (which explains the failure of attempts made up until now to define varieties of modules in a reasonably natural way for the classification of vector bundles of rank $>1$);
or we might be able to show that $F_0$ is indeed representable, but that that vector spaces $F(I_K,\xi)$ are not of finite dimension, in which case we must be content with the "formal" solution;
or it could be the case that $F_0$ is indeed representable by a product of complete Noetherian local rings, which gives very strong assumptions for $F$ itself to be representable, and, combined with the analogous properties (but of a more global nature) that we will later develop, will in all likelihood suffice to imply that it is indeed so.
Finally, we come across interesting geometric problems (see \Cref{fga3.ii-c.4} and \Cref{fga3.ii-c.5} below) where we have only the functor $F_0$ (not coming from any "global" functor $F$), and where we will consider ourselves content if we can associate to it a "formal scheme of modules".


To finish these generalities, we note how the theory of schemes explains some apparent anomalies, such as the Igusa surface $V$ whose "Picard variety" $P$ consists of a single point, and for which, however, $\operatorname{H}^1(V,\mathcal{O}_V)\neq0$;
in this case, $P$ is a non-trivial "purely infinitesimal" group, i.e. defined by a local algebra $\mathcal{O}$ of finite rank over the base field $k$ and endowed with a diagonal map corresponding to the multiplicative structure of $P$;
if $\mathfrak{m}$ is the maximal ideal of $\mathcal{O}$, then the dual of $\mathfrak{m}/\mathfrak{m}^2$ is canonically isomorphic to $\operatorname{H}^1(V,\mathcal{O}_V)$ (cf. \Cref{fga3.ii-c.3} below).
It is only when the Picard group is an algebraic group in the classical sense (i.e. simple over the base field $k$) that the dimension of $\operatorname{H}^1(V,\mathcal{O}_V)$ (which is always equal to that of $\mathfrak{m}/\mathfrak{m}^2$) is equal to that of the Picard group.


\subsection{The schemes $\underline{\operatorname{Hom}}_S(X,Y)$, $\mathrm{pr}od_{X/S}Z$, $\underline{\operatorname{Aut}}(X)$, etc.}\label{fga3.ii-c.2}

Let $X$ and $Y$ be preschemes over $S$;
for every prescheme $T$ over $S$, let $X_T=X\times_S T$ and $Y_T=Y\times_S T$, and consider the set
\[
    F(T)
    = \operatorname{Hom}_T(X_T,Y_T)
    = \operatorname{Hom}_S(X_T,Y)
    = \operatorname{Hom}_S(X\times_S T,Y)
\]
as a contravariant functor in $T$.
If it is representable, then we denote by $\underline{\operatorname{Hom}}_S(X,Y)$ the prescheme over $S$ that represents it, and we then have a functorial isomorphism
\[
    \operatorname{Hom}_S(T,\underline{\operatorname{Hom}}_S(X,Y)) \xrightarrow{\sim} \operatorname{Hom}_S(T\times_S X,Y).
\]
There are variants of this universal problem, the solutions to which can be summarised as follows: a prescheme of \emph{$S$-automorphisms of} an $S$-prescheme $X$ (which will be a prescheme in \emph{groups}), a prescheme of \emph{$S$-homomorphisms} from an $S$-prescheme in \emph{groups} to another (which will be a prescheme in commutative groups if the latter scheme in groups is commutative), etc.
We can also generalise the definition of $\underline{\operatorname{Hom}}_S(X,Y)$ by considering a prescheme $Z$ over the prescheme $X$ over $S$, and the functor
\[
    F(T)
    = \operatorname{Hom}_{X_T}(X_T,Z_T)
\]
(the set of "sections" of $Z_T$ over $X_T$);
if this functor is representable, then the $S$-prescheme that represents it will be denoted by $\Pi_{X/S}Z$, and we will thus have, by definition, a functorial isomorphism
\[
    \operatorname{Hom}_S(T,\Pi_{X/S}Z)
    = \operatorname{Hom}_{X_T}(X_T,Z_T).
\]
Setting $Z=Y\times_S X$, we recover $\underline{\operatorname{Hom}}_S(X,Y)$.
From these definitions follows a formula for the new preschemes thus introduced that is as trivial as it is useful, that we will not give here (given that it holds in every category where products and fibred products exist).
More serious is the question of \emph{existence} of schemes of the type $\underline{\operatorname{Hom}}_S(X,Y)$.
We note first of all that, for fixed $X$, $\underline{\operatorname{Hom}}_S(X,Y)$ (resp. $\Pi_{X/S}Z$) can only exist for all $Y$ over $S$ (resp. for all $Z$ over $X$) if $X$ is \emph{flat} over $S$.
Furthermore, we can convince ourselves that it is not reasonable to expect the existence of a solution, for general enough $Y$, except in the case where $X$ is further \emph{proper} over $S$.
It seems, however, that these conditions are sufficient for the existence of $\underline{\operatorname{Hom}}_S(X,Y)$ and $\Pi_{X/S}Z$, with the condition that, if necessary, we make some sort of "quasi-projective" hypothesis on $Y/S$ (resp. $Z/X$);
this is what we can verify anyway in numerous cases (for example, when $Y$ is \emph{affine} over $S$, or, by direct elementary constructions, when $X$ is \emph{finite} over $S$).
Then \Cref{fga3.ii-b-theorem-1} and \Cref{fga3.ii-b-theorem-2} give:

\begin{proposition}\label{fga3.ii-c.2-proposition-2.1}
    Let $\Lambda$ be a Noetherian ring, and $X$ and $Y$ arbitrary preschemes over $\Lambda$.
    Consider the functor
    \[
        F(A)
        = \operatorname{Hom}_A(X_A,Y_A)
    \]
    on the category $\mathcal{C}_0$ of Artinian $\Lambda$-algebras.
    If $X$ is flat over $\Lambda$, then this functor is pro-representable.
\end{proposition}


Furthermore, we can show that, for all $A\in\mathcal{C}_0$ and all $\xi\in F(A)$, we have a canonical isomorphism
\[
    F(I_A,\xi)
    = \operatorname{H}^1\Big(X_A,\underline{\operatorname{Hom}}_{\mathcal{O}_{X_A}}\big(\xi^*(\Omega_{Y_A/A}^1),\mathcal{O}_{X_A}\big)\Big)
\]
where $\Omega_{Y_A/A}^1$ is the sheaf of Kähler $1$-differentials of $Y_A$ with respect to $A$.
Taking $A$ to be a field, we find, using §A, Proposition 5.1 \Cref{fga3.ii-a.5-proposition-5.1} and the finiteness theorem from \Cref{fga2}, the following corollary:


\begin{proposition}\label{fga3.ii-c.2-proposition-2.1-corollary}
    Suppose that $X$ is flat and proper over $S$, and that $Y$ is of finite type over $S$.
    Then $F$ is pro-representable, and the local components of the corresponding topological $\Lambda$-algebra are \emph{Noetherian} rings.
\end{proposition}

\begin{remark}\label{fga3.ii-c.2-remarks}
    The problems considered in this section, and many others, having been generally studied, in the framework of classical algebraic geometry, via the "Chow coordinates" of cycles in projective space, allow us to consider these cycles as points of suitable projective varieties.
    This procedure, and, more generally, the use of Chow coordinates, seems irredeemably insufficient from the point of view of schemes, since it destroys the nilpotent elements in the parameterised varieties, and, in particular, do not lend themselves to a satisfying study of infinitesimal variations of cycles (without taking its non-intrinsic nature, linked to the projective space, into account).
    The language of Chow coordinates has sadly been the only one used by many algebraic geometers for the study of families of varieties or families of cycles, which seems to have been a serious obstacle to the understanding of these notions, despite its certain technical interest (probably temporary).
    If we wish to obtain the analogue of Chow varieties in the theory of schemes, we are led to the following universal problem:
    let $X$ be a prescheme over $S$, and, for every prescheme $T$ over $S$, consider the set $F(T)$ of closed sub-preschemes of $X_T=X\times_S T$ that are \emph{flat} over $T$; we want to represent this functor in $T$ via some prescheme over $S$.
    More generally, we can start with a quasi-coherent sheaf $\mathcal{G}$ on $X$, and take $F(T)$ to be the set of quotient sheaves of $\mathcal{G}_T$ that are flat over $T$.
    It seems that there exists a solution to this problem, with a scheme $C$ that is locally of finite type over $S$, if $X$ is proper over $S$, if $S$ is locally Noetherian, and if $F$ is furthermore coherent.
    In any case, supposing only that $S$ is locally Noetherian, the restriction of $F$ to "Artinian $S$-algebras" is pro-representable, and, if, furthermore, $X$ is proper over $S$, and $F$ is coherent, then the local components of the corresponding topological ring $\mathcal{O}$ are Noetherian.
    Of course, even after having proven the existence of the "Chow scheme" of $X$ over $S$, it remains to find a decomposition of it into disjoint open subsets $C_i$ (corresponding to fixing continuous invariants, such as degree and dimension of the cycles that we vary) over $S$, as well as to make precise the relations between this scheme with the classical Chow varieties, and to make precise when a $C_i$ is \emph{projective} (or at least quasi-projective) over $S$.
\end{remark}

\begin{remark}\label{fga3.ii-c.2-remark}
    \emph{[Comp.]}
    The problems described here are completely resolved in the projective case by "Hilbert schemes" (cf. \Cref{fga3.iv}).
    Examples by Nagata and Hironaka show, however, that the functors described are not necessarily representable if we do not make the projective hypothesis, even if we restrict to the classification of subvarieties of dimension $0$ of a complete non-singular variety of dimension $3$;
    this is linked to the fact that the symmetric square of such a variety does not necessarily exist.
\end{remark}

\subsection{Picard schemes}\label{fga3.ii-c.3}

\emph{[Comp.]}
For a more complete study, see \Cref{fga3.v}.

Let $f\colon X\to S$ be an $S$-prescheme, and consider the multiplicative sheaf $\mathcal{O}_X^\times$ of units of the structure sheaf of $X$, along with the group
\[
    P(X/S)
    = \operatorname{H}^0(S,\operatorname{R}^1f_*(\mathcal{O}_X^\times))
\]
called the \emph{relative Picard group} of $X/S$.
An element of this group is thus defined by giving an open cover $(U_i)$ of $S$, along with an invertible sheaf $\mathcal{L}_i$ on each $f^{-1}(U_i)$, such that $\mathcal{L}_i|f^{-1}(U_i\cap U_j)$ is isomorphic to $\mathcal{L}_j|f^{-1}(U_i\cap U_j)$ for all $i,j$, or, at least locally over $U_i\cap U_j$ (i.e. these two sheaves are "equivalent" in the sense of FGA 3.I, §B.4 \Cref{fga3.i-b.4}).
If $X/S$ admits a section, then $P(X/S)$ is exactly the set of classes of invertible sheaves on $X/S$ up to "equivalence" (\emph{loc. cit.}).
We now set, for all $T$ over $S$,
\[
    F(T)
    = P(X_T/T)
\]
which gives a covariant functor in $T$, that we call the \emph{Picard functor} of $X/S$;
if this functor is representable, then the prescheme over $S$ that represents it is called the \emph{Picard prescheme} of $X/S$, and denoted by $\mathcal{P}(X/S)$.
In this case, we then have an isomorphism of functors:
\[
    \operatorname{Hom}_S(T,\mathcal{P}(X/S)) \xrightarrow{\sim} P(X_T/T).
\]
Taking the Picard prescheme is compatible with extension of the base, and, in particular, the Picard preschemes of the fibres of $X$ over $S$ (which are preschemes over the residue fields $K(s)$ of the $s\in S$) are the fibres of $\mathcal{P}(X/S)$.
Of course, since $P(X_T/T)=F(T)$ is a commutative group, the Picard preschemes are preschemes in groups.
Note as well that the \emph{generalised Jacobians} of Rosenlicht are exactly the connected components of the identity in the Picard schemes of complete curves (possibly with singularities), which should make most of their properties clear (once their existence has been proven).

\begin{remark}\label{fga3.ii-c.3-remark}
    The definition adopted here is only reasonable when every point of $Y$ admits an open neighbourhood $U$ over which $X$ admits a section.
    In the general case, it is necessary to slightly modify the definition of the Picard functor in order to still obtain an existence theorem.

    Here, the plausible existence conditions for a Picard prescheme are the following: $X$ is \emph{proper} and \emph{flat} over $S$; $f_*(\mathcal{O}_X)=\mathcal{O}_S$; and $X$ locally admits a section over $S$.
    This condition naturally arises in the application of the technique of descent, \emph{in eliminating the automorphisms of an invertible sheaf $\mathcal{L}$ on $X$ by endowing them with a marked section over the section $s$} (FGA 3.I, §B.4 \Cref{fga3.i-b.4}).
    Notably, we find the following:
\end{remark}

\begin{proposition}\label{fga3.ii-c.3-proposition-3.1}

    Suppose that $X$ is flat over $S=\operatorname{Spec}(\Lambda)$, where $\Lambda$ is Noetherian, and suppose that, for all $T$ of finite type over $S$, we have ${f_T}_*(\mathcal{O}_{X_T})=\mathcal{O}_T$ (if $f$ is proper and separable and has separable fibres, or if $S$ is the spectrum of a field, then it follows from Künneth that the latter condition is equivalent to $f_*(\mathcal{O}_X)=\mathcal{O}_S$).
    Then the Picard functor of $X/S$ on the category of Artinian $\Lambda$-algebras is pro-representable.

    Furthermore, we then have
    \[
        F(I_A,\xi)
        = \operatorname{H}^1(X_A,\mathcal{O}_{X_A}),
    \]
    and, in particular:
\end{proposition}

\begin{corollary}\label{fga3.ii-c.3-proposition-3.1-corollary}
    If $X$ is proper over $S$, then the local components of the topological $\Lambda$-algebra corresponding to the Picard functor are Noetherian.
\end{corollary}

\begin{remark}\label{fga3.ii-c.3-remarks-i}
    We can generalise the definitions and results from this section to the classification of principal bundles on $X$, with structure group $G$ being a scheme in groups over $S$ that is \emph{affine} and \emph{flat over $S$}, and also \emph{commutative}.
    In the case where $G$ would not be commutative, and thus where the adjoint bundle in groups of a principal bundle (whose sections of the automorphisms of the principal bundle) would no longer be trivial, \Cref{fga3.ii-c.3-proposition-3.1} no longer holds true as it is stated.
    We can, however, modify the universal problem in such a way that we again obtain a solution (at least, for now, in formal geometry).
    \emph{The golden rule to remember, in the context of the current section and in the following, and every time we are looking for "schemes of modules" for classes of objects that are only defined up to isomorphism, is always the following: eliminate the possible automorphisms of the objects that we want to classify, by introducing, if necessary, auxiliary structures (points or elements of marked sections, fixing differential forms, etc.) that we take to be insignificant enough that we do not substantially modify the initial problem.}
\end{remark}

\begin{remark}\label{fga3.ii-c.3-remarks-ii}
    \emph{[Comp.]}
    I have recently shown that the formal scheme of modules for an abelian variety over a field is indeed simple over the Witt ring, or, in other words, that every abelian scheme $X$ over a local Artinian ring that is the quotient of another such scheme $Y$ comes, by reduction, from an abelian scheme over $Y$.
    The proof simply uses the variance properties of the obstruction class of the covering, introduced in FGA II, §6  \Cref{fga2-6}.
    Recall also that the schemes of modules for curves of genus $g$ or for polarised abelian schemes have been constructed by Mumford (cf. \emph{Séminaire Mumford–Tate}, Harvard University (1961–62)).
\end{remark}

\subsection{Formal modules of a variety}\label{fga3.ii-c.4}

Let $\Lambda$ be a \emph{local} Noetherian ring of residue field $k$ (more often than not, $\Lambda$ will be equal to $k$, or to a Cohen $p$-ring), and let $X_0$ be a prescheme over $k$.
For every local Artinian $\Lambda$-algebra $A$, consider the set $F(A)$ of isomorphism classes of $A$-preschemes $X$ that are \emph{flat} over $A$, endowed with an isomorphism

\begin{equation}\tag{*}\label{fga3.ii-c.4-equation-star}
    X\otimes_A k(A) \xleftarrow{\sim} X_0\otimes_k k(A)
\end{equation}

where $k(A)$ is the residue field of $A$;
of course, the isomorphisms between such flat $A$-preschemes should respect the above isomorphism given in the structure.
If $A$ is a (not necessary local) Artinian $\Lambda$-algebra, with local components $A_i$, then we take $F(A)$ to be the product of the $F(A_i)$.
Then $F$ becomes a multiplicative functor in $A$, and we call it the \emph{functor of modules} for $X_0$ (and $\Lambda$).
If this functor is representable, then it has a corresponding local topological $\Lambda$-algebra $O$, of residue field $k$, and the formal spectrum of $O$ is called the \emph{formal scheme of modules} for $X_0$ (and $\Lambda$) (cf. \Cref{fga2} for some details on this).


Here, if we wish to apply the technique of descent, the "finite" automorphisms of $X_0$ are inoffensive, since they have no influence on the existence of automorphisms (in the precise sense above) of $A$-preschemes $X$;
the necessary and sufficient condition, if $A$ is not simply a field, for $X$ to not have any non-trivial $A$-automorphisms is that
\[
    \operatorname{H}^0(X_0,\mathfrak{G}_{X_0/k})
    = 0
\]
where $\mathfrak{G}_{X_0/k}$ denotes the sheaf of $k$-derivations (i.e. the tangent sheaf) of $X_0$.
We can easily show (at least, if $X_0$ is simple over $k$) that
\[
    F(I_A,\xi)
    = \operatorname{H}^1(X_A,\mathfrak{G}_{X_A/A}).
\]
We thus conclude, as per usual:

\begin{proposition}\label{fga3.ii-c.4-proposition-4.1}
    Suppose that $\operatorname{H}^0(X_0,\mathfrak{G}_{X_0/k})=0$.
    Then the formal scheme of modules for $X_0$ exists.
    If, furthermore, $X_0$ is proper over $k$, then the formal scheme of modules is Noetherian.
\end{proposition}


\begin{remark}\label{fga3.ii-c.4-remarks}
    ~
    \begin{enumerate}
        \item If $X_0$ is not assumed to be simple over $k$, then $F(I_A,\xi)$ can be identified with a sub-$A$-module of
              \[
                  \operatorname{Ext}_{\mathcal{O}_{P_A}}^1(P_A;\mathcal{I}_{X_A},\mathcal{O}_{X_A})
              \]
              where we set $P_A=X_A\times_A X_A$, where $\mathcal{O}_{X_A}$ is considered as a coherent sheaf on $P_A$ via the diagonal morphism $X_A\to P_A$, and where $\mathcal{I}_{X_A}$ denotes the coherent sheaf of ideals on $P_A$ defined by the diagonal morphism.
              More precisely, an easy globalisation of Hochschild theory shows that the $\operatorname{Ext}^1$ above can be identified with the set of classes, up to isomorphism, of sheaves of $I_A$-algebras $\mathcal{O}$ that are flat over $X_A$, and endowed with an augmentation isomorphism $\mathcal{O}\otimes_{I_A}A\to\mathcal{O}_{X_A}$ (recall that we set $I_A=At/(t^2)$).
              The submodule $F(I_A,\xi)$ is that which corresponds to the sheaves of \emph{commutative} algebras.
              The simplicity hypotheses are thus not essential in the theory of modules, as \Cref{fga2} implies.
        \item Recall (\emph{loc. cit.}) that, in particular, every \emph{simple} and \emph{proper algebraic curve} $X_0$ over $k$ admits a formal scheme of modules that is simple over $\Lambda$, and of relative dimension equal to $3g-3$ if the genus $g$ is $\geqslant2$, and to $g$ if $g=0,1$.
              These two latter cases no longer figure directly in \Cref{fga3.ii-c.4-proposition-4.1}.
              We can, however, recover them in the case of elliptic curves ($g=1$) thanks to the remarks that will follow.
    \end{enumerate}
\end{remark}

We can, of course, vary \Cref{fga3.ii-c.4-proposition-4.1} \emph{ad libitum} by considering systems of schemes over $k$ endowed with various structures.
Suppose, for example, that $X_0$ is an \emph{abelian scheme} over $k$, with a marked origin (i.e. $X_0$ is considered as a scheme in \emph{groups} over $k$), and let $F(A)$ be the set of isomorphism classes of \emph{abelian} schemes over $A$ (i.e. of schemes in groups that are proper and simple over $A$) endowed with an isomorphism as in \Cref{fga3.ii-c.4-equation-star} of abelian schemes.
We can show that imposing a multiplicative structure (or even only a "unit section") eliminates the infinitesimal automorphisms, and that there thus exists a formal scheme of modules that corresponds to a complete local Noetherian ring $O$.
We can also show that, if $X$ is a proper and simple scheme with "absolutely connected" fibres over a locally Noetherian prescheme $S$, then every multiplicative structure on $X$ that admits a unit section is necessarily associative and commutative (provided that it is associative and commutative on \emph{one} fibre, and provided that $S$ is connected), and is furthermore uniquely determined by the knowledge of the unit section.
Further, supposing that $S$ is the spectrum of a local Artinian ring $A$ of residue field $k$, that $X$ is proper over $A$ and endowed with a section $s$, and finally that $X\otimes_A k$ is endowed with the structure of an abelian scheme over $k$, admitting the point of $X\otimes_A k$ corresponding to $s$ as the zero element, an easy calculation of obstructions, combined with an argument due to Serre, allows us to prove that there exists on $X$ a multiplicative structure admitting the section $s$ as the unit section.
(From here, using the "existence theorem" of \Cref{fga2} to pass to the case where $A$ is complete local Noetherian, and then the technique of descent from \Cref{fga3.i} for the general case, we can prove the analogous claim for all locally Noetherian connected $S$).
This proves that the functor $F(A)$ considered here is isomorphic to the analogous functor defined at the start of this section by abstracting the multiplicative structure on $X_0$.
It then follows that, in particular, if $\mathfrak{m}$ is the maximal ideal of $O$, then $\mathfrak{m}/\mathfrak{m}^2$ is canonically isomorphic to the dual of $\operatorname{H}^1(X_0,\mathfrak{G}_{X_0/k})$, and is thus of dimension $n^2$, where $n=\dim X_0$.
It would be very interesting to determine if $O$ is indeed \emph{simple} over $\Lambda$, i.e. isomorphic to an algebra of formal series in $n^2$ variables over $\Lambda$.
Now §A, Proposition 5.2  \Cref{fga3.ii-a.5-proposition-5.2} allows us to give an equivalent formulation of this problem as an \emph{existence problem of abelian schemes that are reducible along a given abelian scheme}.
In any case, we see, by a transcendental way, that the answer is affirmative if $k$ is of characteristic $0$.
In characteristic $p\neq0$, it evidently suffices to restrict to the case where $\Lambda$ is the ring of Witt vectors constructed over an algebraically closed field $k$.
This could be the moment for the "\emph{Greenberg functor}" to prove its worth...


\subsection{Extension of coverings}\label{fga3.ii-c.5}

Let $\mathfrak{X}$ be a \emph{formal} Noetherian prescheme [\Cref{fga2}], $U$ an open subset of $\mathfrak{X}$ defined locally by the "non-vanishing" of a section of $\mathcal{O}_\mathfrak{X}$ that is not a zero divisor (and thus large enough that every section of $\mathcal{O}_\mathfrak{X}$ over an open subset $V$ that is zero on $U\cap V$ is zero).
(\emph{[Comp.]} It is also necessary to assume that the section defining $U$ is not a zero divisor not only on $\mathfrak{X}$, but also on every $X_n$.)
Let $\mathfrak{J}$ be an "ideal of definition" for $\mathfrak{X}$, and let $X_n=(\mathfrak{X},\mathcal{O}_\mathfrak{X}/\mathfrak{J}^{n+1})$, which is thus a ordinary Noetherian prescheme.
Then, if $\mathfrak{X}'$ and $\mathfrak{X}''$ are \emph{flat coverings} of $\mathfrak{X}$ (i.e. preschemes over $\mathfrak{X}$ defined by sheaves of algebras that are coherent and locally free as sheaves of modules) that are \emph{unramified over $U$}, the evident map
\[
    \operatorname{Hom}_\mathfrak{X}(\mathfrak{X}',\mathfrak{X}'')
    \to \operatorname{Hom}_{X_0}(X'_0,X''_0)
\]
is \emph{injective};
in particular, \emph{an automorphism of $\mathfrak{X}'$ that induces the identity on $X'_0$ is the identity}.
This allows us to apply the technique of descent to the situation.
We start, in particular, with a flat covering $X'_0$ of $X_0$, unramified over $U_0$, and let $G(\mathfrak{X})$ be the set of classes, up to isomorphism (inducing the identity on $X'_0$), of flat coverings $\mathfrak{X}'$ of $\mathfrak{X}$ that induce $X'_0$ on $X_0$ (and that are thus necessarily unramified over $U$).
We similarly define $G(V)$ for every open subset $V$ of $\mathfrak{X}$, and, more generally, $G(\mathfrak{Y})$ for every formal prescheme $\mathfrak{Y}$ over $\mathfrak{X}$.
With this, the results of \Cref{fga2} and \Cref{fga3.i} imply, first of all, the following results:

\begin{enumerate}[]
    \item If $V$ varies amongst open subset of $\mathfrak{X}$, then the $G(V)$ form a \emph{sheaf} on $\mathfrak{X}$, say $\mathcal{G}_\mathfrak{X}=\mathcal{G}$.
          The restriction of this sheaf to $U$ is the \emph{constant} sheaf whose fibres consist of a single element.

          More generally, describing the fibres of $\mathcal{G}_\mathfrak{X}$ is a question of complete local rings, in a precise way:
    \item For all $x\in\mathfrak{X}$, we have
          \[
              \mathcal{G}_x = G(\operatorname{Spec}(\mathcal{O}_{\mathfrak{X},x})) \subset G(\operatorname{Spec}(\widehat{\mathcal{O}}_{\mathfrak{X},x}))
          \]
          (i.e. isomorphism classes of finite free algebras $B$ over $\widehat{\mathcal{O}}_{\mathfrak{X},x}$ endowed with an isomorphism from $B\otimes_{\widehat{\mathcal{O}}_{\mathfrak{X},x}}\mathcal{O}_{X_0,x}$ to $(\widehat{\mathcal{O}}'_0)_x$, where $\mathcal{O}'_0$ is the sheaf of algebras on $X_0$ that defines $X'_0$).
    \item We have a canonical isomorphism $\mathcal{G}_\mathfrak{X}=\varprojlim\mathcal{G}_{\mathfrak{X}_n}$; in other words, for every open subset $V$ of $\mathfrak{X}$, we have $G(V)=\varprojlim G(V_n)$.
    \item Suppose that $\mathfrak{X}$ comes from an ordinary \emph{proper} scheme $X$ over a complete local Noetherian ring $\Lambda$ that has ideal of definition $\mathfrak{m}$ by taking the $\mathcal{J}$-adic completion of $\mathcal{O}_X$, where $\mathcal{J}=\mathfrak{m}\cdot\mathcal{O}_X$.
          Then $G(\mathfrak{X})$ is canonically isomorphic to the set of classes of flat coverings of the \emph{ordinary} scheme $X$ that are "reducible along $X'_0$".
\end{enumerate}

Figuratively speaking, we can say that (a) and (b) establish the fundamental relations between the local and global aspects of the problem; (c) gives the relations between the "finite" and "infinitesimal" aspects; and finally (d) remembers (under precise conditions) the identity between the "formal" and "algebraic" aspects.

Now suppose that $\mathfrak{X}$ is defined by a local complete Noetherian ring $\Lambda$, with $\mathcal{J}=\mathfrak{m}\cdot\mathcal{O}_X$ (and so $X_0$ is a prescheme over ${\Lambda}/\mathfrak{m}$).
For every algebra $A$ that is finite over $\Lambda$, we set
\[
    F(A)
    = G(\mathfrak{X}\times_\Lambda A).
\]
This is a covariant functor in $A$, with values in the category of sets, and, by (c), this functor is completely determined by how it acts on Artinian algebras $A$;
it is equivalent to say either that this functor is pro-representable, i.e. of the form
\[
    F(A)
    = \operatorname{Hom}_{\text{top. }\Lambda\text{-algebras}}(\mathcal{O},A)
\]

where $\mathcal{O}$ is a topological $\Lambda$ algebra of the type described in \Cref{fga3.ii-a.5}, or that this is true when we restrict to only Artinian algebras $A$.
The combination of \Cref{fga3.ii-b-theorem-1} and \Cref{fga3.ii-b-theorem-2} then effectively implies:

\begin{proposition}\label{fga3.ii-c.5-proposition-5.1}
    The above functor is pro-representable.
\end{proposition}

Of course, by (a), if $U=\mathfrak{X}$, then $G(\mathfrak{Y})$ consists of a single element for all $\mathfrak{Y}$ over $\mathfrak{X}$, and the functor $F$ is then not very interesting (we will have $\mathcal{O}=\Lambda$).
It seems that, in practically every other case, the topological local ring $\mathcal{O}$ \emph{is not Noetherian}.
Its existence, however, shows, in a striking manner, the \emph{"continuous" nature} of the set $G(\mathfrak{X})$ of solutions (corresponding intuitively to the fact that there is a "continuous" choice in the way in which the ramification spreads when we take an extension of $X'_0$).
We will compare this result with the point of view of J.-P. Serre [@Ser1958] via local class field theory, drawing attention as well to the \emph{continuous} character of the topological Galois group of the maximal abelian extension of a "geometric" local field, with the dual group (in the sense of Pontrjagin) appearing as an inductive limit of algebraic (or at least quasi-algebraic) groups;
here as well, the classification of extensions is given by infinite-dimensional "varieties".
We can also take, in the above, $\mathfrak{X}$ to be the formal spectrum of a complete local ring (of which $\Lambda$ will be, for example, a Cohen subring), and we might hope that the results of this section can be used in the study of extensions of a local complete ring of dimension $>1$.
Just as much in the local case as in the global case, they might allow us to formulate precise relations between the phenomena of higher ramification and phenomena in characteristic $0$ (approachable via a transcendental way).
In any case, it is the preliminary analysis of \Cref{fga3.ii-c.5-proposition-5.1} that allows us to extend the methods described in \Cref{fga2} for the study of the fundamental group to the "tamely ramified" case, and to resolve, by a transcendental way, the "problem of three points".

To finish, we note that the situation simplifies if $X_0$ is of dimension $1$;
then, by (a) and (b), $G(\mathfrak{X})$ can be identified with $\mathrm{pr}od_i G(\operatorname{Spec}(\mathcal{O}_{\mathfrak{X},x_i}))$, where the $x_i$ are the points of $X_0\setminus U$:
\emph{we can take arbitrary "local" extensions at the ramification points}.
Further, if $X_0$ is normal, then we note that the formal scheme of modules guaranteed by \Cref{fga3.ii-c.5-proposition-5.1} is \emph{simple} over $\operatorname{Spec}(\Lambda)$.
