% !TeX root = ../../fga.tex
\section{Results concerning local cohomology}\label{fga1-3}

Let $A$ be a unital commutative ring endowed with an ideal $\mathfrak{J}$.
We will define, for any $A$-module $M$, functorial homomorphisms

\begin{equation}\tag{3.1}\label{fga1-equation-3.1}
  \begin{aligned}
    \operatorname{Ext}_A^p(A/\mathfrak{J},M)
     & \to \operatorname{Hom}_A(\wedge^p\mathfrak{J}/\mathfrak{J}^2,M\otimes A/\mathfrak{J})
    \\\operatorname{Tor}_p^A(A/\mathfrak{J},M)
     & \leftarrow (\wedge^p\mathfrak{J}/\mathfrak{J}^2)\otimes\operatorname{Hom}_A(A/\mathfrak{J},M)
  \end{aligned}
\end{equation}

where the tensor and exterior products are taken over the ring $A$;
note also that $\mathfrak{J}/\mathfrak{J}^2$ is in fact an $A/\mathfrak{J}$-module, and that its exterior powers as an $A$-module agree with its exterior powers as an $A/\mathfrak{J}$-module.
The definition of the homomorphisms in \Cref{fga1-equation-3.1} come from the definition, for every system $x=(x_1,\ldots,x_p)$ of points of $\mathfrak{J}$, of homomorphisms $\varphi_x$ given by

\begin{equation}\tag{3.2}\label{fga1-equation-3.2}
  \begin{aligned}
    \varphi_x\colon \operatorname{Ext}_A^p(A/\mathfrak{J},M) & \to M\otimes A/\mathfrak{J}
    \\\varphi_x\colon \operatorname{Hom}_A(A/\mathfrak{J},M) &\to \operatorname{Tor}_p^A(A/\mathfrak{J},M)
  \end{aligned}
\end{equation}

such that the following conditions are satisfied:

\begin{enumerate}[i.]
  \item $\varphi_{x_1,\ldots,x_p}$ depends on the system of the $x_i\in\mathfrak{J}$ in an alternating $A$-multilinear way;
  \item $\varphi_{x_1,\ldots,x_p}$ is zero when any of the $x_i$ is in $\mathfrak{J}^2$.
\end{enumerate}


In fact, (ii) follows from (i), since $a\varphi_x=0$ for $a\in\mathfrak{J}$, as we see by noting that all the modules in \Cref{fga1-equation-3.2} are annihilated by $\mathfrak{J}$.

To define the $\varphi_x$, we consider the complex $K_x$ whose underlying $A$-modules are the $\wedge A^p$, and whose differential is the interior product $i_x$ by $x$, considered as a linear form on $A^p$ with components $x_1,\ldots,x_p$.
The differential is of degree $-1$, the degrees of the complex are positive, and the cohomology of this complex in dimension $0$ is $A/(x_1A+\ldots+x_pA)$.
Since the $x_i$ are in $\mathfrak{J}$, we obtain an augmentation $K_{x,0}\to A/\mathfrak{J}$.
Thus $K_x$ is a \emph{free} augmented complex, with augmentation module $A/\mathfrak{J}$.
We thus obtain known homomorphisms

\begin{align*}
  \operatorname{Ext}_A^\bullet(\operatorname{H}_0(K_x),M) & \to \operatorname{H}^\bullet(\operatorname{Hom}_A(K_x,M))
  \\\operatorname{Tor}_\bullet^A(\operatorname{H}_0(K_x),M) &\leftarrow \operatorname{H}_\bullet(K_x\otimes M)
\end{align*}

whence, by composing with the homomorphisms to the $\operatorname{Ext}$ and the $\operatorname{Tor}$ induced by the augmentation homomorphism $\operatorname{H}_0(K_x)\to A/\mathfrak{J}$, we obtain homomorphisms
\begin{equation}\tag{3.3}\label{fga1-equation-3.3}
  \begin{aligned}
    \psi_x\colon \operatorname{Ext}_A^\bullet(A/\mathfrak{J},M)
     & \to \operatorname{H}^\bullet(\operatorname{Hom}_A(K_x,M))
    \\\psi_x\colon \operatorname{Tor}_\bullet^A(A/\mathfrak{J},M)
     & \leftarrow \operatorname{H}_\bullet(K_x\otimes M).
\end{aligned}
\end{equation}

But we immediately note that, in the maximal dimension $p$, the cohomology of the right-hand side is $M(x_1M+\ldots+x_pM)$ (resp. the set of elements of $M$ that are annihilated by each of the $x_i$).
Since the $x_i$ are in $\mathfrak{J}$, we thus obtain homomorphisms


\begin{equation}\tag{3.4}\label{fga1-equation-3.4}
  \begin{aligned}
    \operatorname{H}^p(\operatorname{Hom}_A(K_x,M)) & \to M\otimes A/\mathfrak{J}
    \\\operatorname{H}_p(K_x\otimes M) &\leftarrow \operatorname{Hom}_A(A/\mathfrak{J},M).
  \end{aligned}
\end{equation}


By composing the homomorphisms in \Cref{fga1-equation-3.3} and \Cref{fga1-equation-3.4} we obtain the homomorphisms in \Cref{fga1-equation-3.2} that we wanted to define.
The verification of (i) is tedious, but does not present any difficulties.

\begin{proposition}\label{fga1-proposition-4}
  Let $A$ be a commutative unital ring, and let $(x_1,\ldots,x_p)$  be a sequence of elements of $A$ such that, for $1\leqslant i\leqslant p$, the image of $x_i$ in the quotient of $A$ by the ideal generated by $(x_1,\ldots,x_{i-1})$ is not a zero divisor.
  Let $\mathfrak{J}$ be the ideal generated by the $x_i$.
  Then $\mathfrak{J}/\mathfrak{J}^2$ is a free $(A/\mathfrak{J})$-module, with basis given by the canonical images of the $x_i$;
  the complex $K_x$ is a free resolution of $A/\mathfrak{J}$;
  and, for every $A$-modules $M$, the homomorphisms in \Cref{fga1-equation-3.1} in dimension $p$ are bijective.
  The same is true for the analogous homomorphisms defined for arbitrary degree $i$ as long as $\mathfrak{J}\cdot M=0$.
\end{proposition}

(The essential point in \Cref{fga1-proposition-4}, from which all others follow, is the acyclicity of $K_x$, which is a well-known fact, under the conditions given).


\begin{corollary}\label{fga1-proposition-4-corollary-1}
  With $A$ and $\mathfrak{J}$ as in \Cref{fga1-proposition-4}, suppose further that $A$ is a regular affine algebra of $\dim n$ over a perfect field $k$, and that $A/\mathfrak{J}$ is a regular affine algebra.
  Denote by $\Omega^i(A)$ and $\Omega^i(A/\mathfrak{J})$ the modules of Kähler differentials.
  Then we have a canonical isomorphism
  \begin{equation}\tag{3.5}\label{fga1-equation-3.5}
    \operatorname{Ext}_A^p(\Omega^{n-p}(A/\mathfrak{J}),\Omega^n(A)) = A/\mathfrak{J}
  \end{equation}
  which is compatible with localisation.
\end{corollary}

\begin{cproof}
  Indeed, $\Omega^{n-p}(A/\mathfrak{J})$ is a free $(A/\mathfrak{J})$-module of rank $1$, and, similarly, $\Omega^n(A)$ is a free $A$-module of rank $n$, and so the left-hand side is equal to
  \[\operatorname{Ext}_A^p(A/\mathfrak{J},A) \otimes \Omega^{n-p}(A/\mathfrak{J})' \otimes \Omega^n(A)\]
  (where the $'$ notation denotes the dual $(A/\mathfrak{J})$-module).
  The tensor product of these last two factors can be identified with $\wedge^p(\mathfrak{J}/\mathfrak{J}^2)$, and so the whole thing can be identified with $\operatorname{Ext}_A^p(A/\mathfrak{J},\wedge^p(\mathfrak{J}/\mathfrak{J}^2))$, and thus, by \Cref{fga1-proposition-4}, with
  \[\operatorname{Hom}_A(\wedge^p \mathfrak{J}/\mathfrak{J}^2,\wedge^p \mathfrak{J}/\mathfrak{J}^2)\]
  i.e. to $A/\mathfrak{J}$.
\end{cproof}

In particular, there is a distinguished element in $\operatorname{Ext}_A^p(\Omega^{n-p}(A/\mathfrak{J}),\Omega^n(A))$, corresponding to the unit of $A/\mathfrak{J}$, called the \emph{fundamental class} of the ideal $\mathfrak{J}$ in $A$.
(In fact, it can be defined under rather more general conditions).
We can write \Cref{fga1-proposition-4-corollary-1} in a more geometric and global form:


\begin{corollary}\label{fga1-proposition-4-corollary-2}
  Let $X$ be a non-singular variety over an algebraically-closed field $k$, $Y$ a closed non-singular subvariety of $X$, $\mathcal{O}_X$ the structure sheaf of $X$, and $\mathcal{O}_Y$ the structure sheaf of $Y$, considered as a quotient sheaf of $\mathcal{O}_X$.
  Let $n$ be the dimension of $X$, and $n-p$ the dimension of $Y$.
  Let $\Omega_X$ (resp. $\Omega_Y$) be the sheaf of germs of regular differential forms on $X$ (resp. $Y$).
  Then we have canonical isomorphisms
  \begin{equation}\tag{3.6}\label{fga1-equation-3.6}
    \mathcal{E}xt_{\mathcal{O}_X}^p(\Omega_Y^{n-p},\Omega_X^n) = \mathcal{O}_Y
  \end{equation}
  as well as
  \begin{equation}\tag{3.6 bis}\label{fga1-equation-3.6bis}
    \operatorname{Ext}_{\mathcal{O}_X}^p(\mathcal{O}_Y,\Omega_X^n) = \Omega_Y^{n-p}
  \end{equation}
\end{corollary}


\Cref{fga1-equation-3.6bis} can serve as the \emph{definition} of $\Omega_Y^{n-p}$ when $Y$ is a singular variety.
More precisely:

\begin{proposition}\label{fga1-proposition-5}
  Let $X$ be a non-singular algebraic variety of dimension $n$, and let $Y$ be an algebraic subset of dimension $q=n-p$ of $X$.
  Let $\mathcal{F}$ be a coherent algebraic sheaf on $X$ with support contained in $Y$, and let $\mathcal{L}$ be a locally-free algebraic sheaf on $X$.
  Then the sheaves $\mathcal{E}xt_{\mathcal{O}_X}^i(\mathcal{F},\mathcal{L})$ are zero for $i<p$, and, when $i=p$, there is a canonical isomorphism
  \begin{equation}\tag{3.7}\label{fga1-equation-3.7}
    \mathcal{E}xt_{\mathcal{O}_X}^p(\mathcal{F},\mathcal{L})
    = \mathcal{H}om_{\mathcal{O}_X}(\mathcal{F},\mathcal{E}xt^p(\mathcal{O}_X/\mathfrak{J},\mathcal{L}))
  \end{equation}
  where $\mathfrak{J}$ denotes an arbitrary sheaf of ideals on $X$ that annihilates $\mathcal{F}$ and has $Y$ as its set of zeros.
  In particular, if $\mathcal{F}$ is a coherent algebraic sheaf on $Y$, then
  \begin{equation}\tag{3.7 bis}\label{fga1-equation-3.7bis}
    \mathcal{E}xt_{\mathcal{O}_X}^p(\mathcal{F},\mathcal{L})
    = \mathcal{H}om_{\mathcal{O}_Y}(\mathcal{F},\mathcal{E}xt^p(\mathcal{O}_Y,\mathcal{L}))
  \end{equation}
  Finally, with $\mathcal{F}$ still a coherent algebraic sheaf on $Y$, the sheaves $\mathcal{E}^i=\mathcal{E}xt_{\mathcal{O}_X}^{p+i}(\mathcal{F},\Omega_X^n)$ do not depend on the choice of immersion of the algebraic space $Y$ into the non-singular algebraic variety $X$.
\end{proposition}

\begin{cproof}
  Since the question is local, we can assume that $X$ is affine and that $\mathcal{L}=\mathcal{O}_X$.
  This then reduces to a problem of commutative algebra, and, more specifically, of local algebra:
  if $A$ is a regular locality, and $M$ an $A$-module whose support is of dimension $\leqslant q=n-p$, then we have to prove that $\operatorname{Ext}_A^i(M,A)=0$ for $i<p$ and that $\operatorname{Ext}_A^p(M,A)=\operatorname{Hom}_A(M,\operatorname{Ext}^p(A/\mathfrak{J},A))$, where $\mathfrak{J}$ is an arbitrary ideal of "dimension" $\leqslant q$ that annihilates $M$.
  For the first claim, we proceed by induction on $q$:
  an immediate \emph{dévissage} leads to the case where $M$ is of the form $A/\mathfrak{J}$, and thus leads, by replacing $\mathfrak{J}$ with a smaller ideal and using the induction hypothesis, as well as the exact sequence of the $\operatorname{Ext}$, to the case where $\mathfrak{J}$ is generated by a "system of parameters", as in \Cref{fga1-proposition-4}, where the result is immediate.
  The previous result implies that, if $\mathfrak{J}$ is a fixed ideal of "dimension" $\leqslant q$, then the contravariant functor $E(M)=\operatorname{Ext}_A^p(M,A)$ to the category of $(A/\mathfrak{J})$-modules is left exact;
  furthermore, it sends direct sums to direct products, from which it easily follows that $E(M)=\operatorname{Hom}_A(M,E(A))$.
  Finally, the last claim of \Cref{fga1-proposition-5} is more subtle, and follows from an intrinsic characterisation of the $E^i(F)$ via a local duality theorem which cannot be stated here.
\end{cproof}

\begin{corollary}\label{fga1-proposition-5-corollary}
  Denote by $\omega_Y^q$ the sheaf $\operatorname{Ext}_{\mathcal{O}_X}^p(\mathcal{O}_Y,\Omega_X^n)$.
  Then there is a functorial isomorphism for coherent algebraic sheaves $\mathcal{F}$ on $Y$:
  \begin{equation}\tag{3.8}\label{fga1-equation-3.8}
    \mathcal{E}xt_{\mathcal{O}_X}^p(\mathcal{F},\Omega_X^n) = \mathcal{H}om_{\mathcal{O}_X}(\mathcal{F},\omega_Y^q)
  \end{equation}
\end{corollary}
