% !TeX root = ../../fga.tex
\section{Cohomology class associated to a subvariety}\label{fga1-4}

In all that follows, $X$ denotes an algebraic set of dimension $n$, defined over a field $k$ which we assume, for simplicity, to be algebraically closed.
Except for in \Cref{fga1-6}, we assume that $X$ is non-singular.
We denote by $\mathcal{O}_X$ the structure sheaf of $X$, and by $\Omega_X^\bullet=\bigcup_p\Omega_X^p$ the sheaf of germs of differential forms on $X$.
If $Y$ is a closed subset of $X$, then we identify coherent algebraic sheaves on $Y$ with coherent algebraic sheaves on $X$ that are zero outside of $Y$;
we do this, in particular, with $\mathcal{O}_Y$ and $\Omega_Y$.


\begin{lemma}\label{fga1-lemma-1}
  Let $\mathcal{F}$ be a coherent algebraic sheaf on $X$ whose support is of dimension $\leqslant n-p$, and let $\mathcal{L}$ be a coherent algebraic sheaf on $X$ that is locally free.
  Then $\mathcal{E}xt_{\mathcal{O}_X}^i(X;\mathcal{F},\mathcal{L})$ is zero for $i<p$, and there is a canonical isomorphism
  \begin{equation}\tag{4.1}\label{fga1-equation-4.1}
    \operatorname{Ext}_\mathcal{O}^p(X;\mathcal{F},\mathcal{L}) = \operatorname{H}^0(X,\mathcal{E}xt_\mathcal{O}^p(\mathcal{F},\mathcal{L}))
  \end{equation}
  If $\mathcal{F}$ is a coherent algebraic sheaf on a closed subset $W$ of $X$ of dimension $\leqslant n-p$, then we have a canonical isomorphism
  \begin{equation}\tag{4.1 bis}\label{fga1-equation-4.1bis}
    \mathcal{E}xt_{\mathcal{O}_X}^p(\mathcal{F},\mathcal{L}) = \mathcal{H}om_{\mathcal{O}_X}(\mathcal{F}\otimes\mathcal{L}'\otimes\Omega_X^n,\omega_Y^{n-p})
  \end{equation}
  where $\omega_Y^{n-p}$ is the sheaf on $Y$ defined in the corollary to Proposition 5 \Cref{fga1-proposition-5-corollary} (which can be identified with $\Omega_Y^{n-p}$ if $Y$ is non-singular).
\end{lemma}

\begin{cproof}

  The formula in \Cref{fga1-equation-4.1} is an immediate consequence of the spectral sequence from \Cref{fga1-proposition-1}, as well as \Cref{fga1-proposition-5};
  by the formula in \Cref{fga1-equation-3.8}, we can write

  \begin{align*}
    \mathcal{E}xt_{\mathcal{O}_X}^p(\mathcal{F},\mathcal{L})
    = \mathcal{L}\otimes(\Omega_X^n)'\otimes\mathcal{E}xt_{\mathcal{O}_X}(\mathcal{F},\Omega_X^n)
    \\= \mathcal{L}\otimes(\Omega_X^n)'\otimes\mathcal{H}om_{\mathcal{O}_X}(\mathcal{F},\omega_Y^q)
    \\= \mathcal{H}om_{\mathcal{O}_X}(\mathcal{F}\otimes\mathcal{L}'\otimes\Omega_X^n,\omega_Y^q)
  \end{align*}

  where $q=n-p$, whence the formula in \Cref{fga1-equation-4.1bis}.
\end{cproof}

Setting, in particular, $\mathcal{F}=\mathcal{O}_Y$ and $\mathcal{L}=\Omega_X^p$, we obtain (taking into account the fact that $\Omega_X^n\otimes(\Omega_X^p)'=\Omega_X^{n-p}$) a canonical isomorphism

\begin{equation}\tag{4.2}\label{fga1-equation-4.2}
  \operatorname{Ext}_{\mathcal{O}_X}^p(X;\mathcal{O}_Y,\Omega_X^p) = \operatorname{Hom}_{\mathcal{O}_X}(X;\Omega_X^{n-p},\omega_Y^{n-p})
\end{equation}

Now suppose, for simplicity, that $Y$ is \emph{non-singular}, so that $\omega_Y^{n-p}=\Omega_Y^{n-p}$.
There is a natural homomorphism from $\Omega_X^{n-p}$ to $\Omega_Y^{n-p}$, whence a canonical section $s_Y$ of the sheaf $\mathcal{E}xt_{\mathcal{O}_X}^p(\mathcal{O}_Y,\Omega_X^p)$, that we call, if all the components of $Y$ are of dimension $n-p$, the \emph{fundamental section} of the sheaf $\mathcal{E}xt_{\mathcal{O}_X}^p(\mathcal{O}_Y,\Omega_X^p)$.
By \Cref{fga1-equation-4.1}, this section defines an element of $\operatorname{Ext}_{\mathcal{O}_X}^p(X;\mathcal{O}_Y,\Omega_X^p)$.
But the natural homomorphism $\mathcal{O}_X\to\mathcal{O}_Y$ defines a homomorphism
\[
  \operatorname{Ext}_{\mathcal{O}_X}^p(X;\mathcal{O}_Y,\Omega_X^p)
  \to \operatorname{Ext}_{\mathcal{O}_X}^p(X;\mathcal{O}_X,\Omega_X^p)
  = \operatorname{H}^p(X,\Omega_X^p).
\]
We thus obtain an element of $\operatorname{H}^p(X,\Omega_X^p)$, denoted by $P_X(Y)$, that we call the \emph{cohomology class of $Y$ in $X$};
it is induced by the section $s_Y$ of $\mathcal{E}xt_{\mathcal{O}_X}^p(\mathcal{O}_Y,\Omega_X^p)$ by the following diagram of homomorphisms:

\begin{equation}\tag{4.3}\label{fga1-equation-4.3}
  \begin{CD}
    \operatorname{Ext}_{\mathcal{O}_X}^p(X;\mathcal{O}_Y,\Omega_X^p) @>\sim>> \operatorname{H}^0(X,\mathcal{E}xt_{\mathcal{O}_X}^p(\mathcal{O}_Y,\Omega_X^p))
    \\@VVV
    \\\begin{aligned}
      \operatorname{Ext}_{\mathcal{O}_X}^p(X;\mathcal{O}_X,\Omega_X^y)
      \\=\operatorname{H}^p(X,\Omega_X^p)
    \end{aligned}
  \end{CD}
\end{equation}

We define a \emph{non-singular cycle} of dimension $n-p$ to be any element of the free abelian group generated by the non-singular irreducible subvarieties of dimension $n-p$ in $X$.
Then the function $Y\mapsto P(Y)$ can be extended to a homomorphism from the group of non-singular cycles of dimension $n-p$ on $X$ to the group $\operatorname{H}^p(X,\Omega_X^p)$.

Let $Z^{n-p}$ and $Z'^{n-p'}$ be non-singular cycles of dimension $n-p$ and $n-p'$ (respectively);
we say that they \emph{intersect transversally} if every component of $Z$ intersects transversally with every component of $Z'$.
Then the cycle $Z\cdot Z'$ is defined, and is a non-singular cycle of dimension $n-p-p'$.
With this, we have:

\begin{theorem}\label{fga1-theorem-1}
  If $Z^{n-p}$ and $Z'^{n-p'}$ are non-singular cycles that intersect transversally, then
  \begin{equation}\tag{4.4}\label{fga1-equation-4.4}
    P_X(Z\cdot Z') = P_X(Z)\cdot P_X(Z')
  \end{equation}
  where the product on the right-hand side is the cup product:
  \[\operatorname{H}^p(X,\Omega_X^p)\times\operatorname{H}^{p'}(X,\Omega_X^{p'}) \to \operatorname{H}^{p+p'}(X,\Omega_X^{p+p'})\]
  (We assume that $X$ is isomorphic to a locally closed subset of a projective space).
\end{theorem}

\begin{cproof}
  To prove \Cref{fga1-theorem-1}, we can assume that $Z$ and $Z'$ are irreducible non-singular subvarieties $Y$ and $Y'$ that intersect transversally.
  Let $\mathcal{L}_\bullet$ be a left resolution of $\mathcal{O}_Y$ by locally-free sheaves;
  then, by \Cref{fga1-proposition-3}, the diagram of homomorphisms in \Cref{fga1-equation-4.3} can be identified with the diagram
  \[
    \begin{CD}
      (\underline{\operatorname{R}}^p\operatorname{\Gamma})\big(\mathcal{H}om_{\mathcal{O}_X}(\mathcal{L}_\bullet,\Omega_X^p)\big) @>\beta>> \operatorname{\Gamma}\big(\operatorname{H}^p\big(\mathcal{H}om_{\mathcal{O}_X}(\mathcal{L}_\bullet,\Omega_X^p)\big)\big)
      \\@V\alpha VV .
      \\(\operatorname{R}^p\operatorname{\Gamma})(\Omega_X^p)
    \end{CD}
  \]
  where $\beta$ is an isomorphism, and where $\operatorname{\Gamma}$ is the "group of sections" functor on the category of abelian sheaves on $X$, $\underline{\operatorname{R}}^p\operatorname{\Gamma}$ is its hypercohomology in dimension $p$, and $\operatorname{R}^p\operatorname{\Gamma}$ is its $p$-th derived functor.
  For simplicity, we assume that $\mathcal{L}_0=\mathcal{O}_X$, and that the augmentation $\mathcal{L}_0\to\mathcal{O}_Y$ is the natural homomorphism (which we can indeed safely assume);
  then $\alpha$ is induced by the homomorphism of complexes $\mathcal{O}_X\to\mathcal{L}$ (with $\mathcal{O}_X$ being thought of as a complex concentrated in degree $0$), taking into account the fact that $\underline{\operatorname{R}}^p\operatorname{\Gamma}(\mathcal{K})=\operatorname{R}^p\operatorname{\Gamma}(\mathcal{K}_0)$ if $\mathcal{K}$ is a complex of sheaves concentrated in degree $0$.
  The homomorphism $\beta$ is a well-known "boundary map".
  Consider an analogous diagram, relative to a locally-free resolution $\mathcal{L}'_\bullet$ of $\mathcal{O}_Y$, and consider the commutative diagram of pairings:

  \begin{table}[!ht]
    \centering
    \tabcolsep=0mm
    \renewcommand\arraystretch{1.2}
    \begin{tabular}{ccccc}
      \small$\operatorname{R}^p\operatorname{\Gamma}(\Omega_X^p)$           & $\leftarrow$ & \small$\underline{\operatorname{R}}^p\operatorname{\Gamma}\big(\mathcal{H}om_{\mathcal{O}_X}(\mathcal{L}_\bullet,\Omega_X^p)\big)$                                      & $\substack{\sim\\\rightarrow}$ & \small$\operatorname{\Gamma}\big(\operatorname{H}^p\big(\mathcal{H}om_{\mathcal{O}_X}(\mathcal{L}_\bullet,\Omega_X^p)\big)\big)$                                      \\
      $\times$                                               &              & $\times$                                                                                                                                                 &               & $\times$                                                                                                                                               \\
      \small$\operatorname{R}^{p'}\operatorname{\Gamma}(\Omega_X^{p'})$     & $\leftarrow$ & \small$\underline{\operatorname{R}}^{p'}\operatorname{\Gamma}\big(\mathcal{H}om_{\mathcal{O}_X}(\mathcal{L}'_\bullet,\Omega_X^{p'})\big)$                               & $\substack{\sim\\\rightarrow}$ & \small$\operatorname{\Gamma}\big(\operatorname{H}^{p'}\big(\mathcal{H}om_{\mathcal{O}_X}(\mathcal{L}'_\bullet,\Omega_X^{p'})\big)\big)$                              \\
      $\downarrow$                                           &              & $\downarrow$                                                                                                                                             &               & $\downarrow$                                                                                                                                           \\
      \small$\operatorname{R}^{p+p'}\operatorname{\Gamma}(\Omega_X^{p+p'})$ & $\leftarrow$ & \small$\underline{\operatorname{R}}^{p+p'}\operatorname{\Gamma}\big(\mathcal{H}om_{\mathcal{O}_X}(\mathcal{L}_\bullet\otimes\mathcal{L}'_\bullet,\Omega_X^{p+p'})\big)$ & $\substack{\sim\\\rightarrow}$ & \small$\operatorname{\Gamma}\big(\operatorname{H}^{p+p'}\big(\mathcal{H}om_{\mathcal{O}_X}(\mathcal{L}_\bullet\otimes\mathcal{L}'_\bullet,\Omega_X^{p+p'})\big)\big)$
    \end{tabular}
  \end{table}
  \begin{equation}\tag{4.5}\label{fga1-equation-4.5}
    ~
  \end{equation}

  The pairings in the two columns on the right are induced by the pairing of complexes of sheaves
  \[
    \mathcal{H}om_{\mathcal{O}_X}(\mathcal{L}_\bullet,\Omega_X^p) \times \mathcal{H}om_{\mathcal{O}_X}(\mathcal{L}'_\bullet,\Omega_X^{p'})
    \to \mathcal{H}om_{\mathcal{O}_X}(\mathcal{L}_\bullet\otimes\mathcal{L}',\Omega_X^{p+p'})
  \]
  that we define by using the exterior product $\Omega_X^p\times\Omega_X^{p'}\to\Omega_X^{p+p'}$;
  the pairing in the first column is the cup product (relative to the exterior product).
  I claim that the last line of \Cref{fga1-equation-4.5}can be identified with the diagram of isomorphisms analogous to \Cref{fga1-equation-4.3}, where $Y$ is replaced by $Y\cap Y'$ and $p$ by $p+p'$.
  For this, it suffices to show that $\mathcal{L}\otimes\mathcal{L}'$ is a resolution (evidently locally-free) of $\mathcal{O}_{Y\cap Y'}$.
  But then
  \[
    \operatorname{H}_0(\mathcal{L} \otimes \mathcal{L}')
    = \mathcal{O}_Y \otimes \mathcal{O}_{Y'}
    = \mathcal{O}_{Y \cap Y'}
  \]
  and
  \[
    \operatorname{H}_i(\mathcal{L} \otimes \mathcal{L}')
    = \operatorname{Tor}_i^{\mathcal{O}_X}(\mathcal{O}_Y,\mathcal{O}_{Y'})
    = 0
  \]
  for $i>0$, from the fact that $Y$ and $Y'$ intersect transversally.
  Then \Cref{fga1-theorem-1} follows from the formula:

  \begin{equation}\tag{4.6}\label{fga1-equation-4.6}
    s_Y\cdot s_{Y'} = s_{Y\cdot Y'}
  \end{equation}

  (where the product on the left-hand side is that from the last column of \Cref{fga1-equation-4.5}).
  This formula in \Cref{fga1-equation-4.6}, which is of a purely local nature, can easily be proven by taking $\mathcal{L}_\bullet$ and $\mathcal{L}'_\bullet$ to be the resolutions described in \Cref{fga1-proposition-4}.
  We can similarly prove (even more easily) that $Z\mapsto P_X(Z)$ is compatible with the cartesian product:

  \begin{equation}\tag{4.7}\label{fga1-equation-4.7}
    P_{X\times X'}(Z\times Z') = P_X(Z)\otimes P_{X'}(Z')
  \end{equation}

  (a formula which holds true if $Z$ (resp. $Z'$) is a non-singular cycle on the non-singular variety $X$ (resp. $X'$), with $Z\times Z'$ being thought of as a non-singular cycle on $X\times X'$).
  From \Cref{fga1-equation-4.4} and \Cref{fga1-equation-4.7}, it follows that $P_X(Z)$ is also compatible with the operation given by taking the "inverse image" under a morphism $f\colon X\to X'$ of non-singular varieties:

  \begin{equation}\tag{4.8}\label{fga1-equation-4.8}
    P_X(f^{-1}(Z')) = f^*(P_{X'}(Z))
  \end{equation}

  a formula which holds true if $Z$ is a non-singular cycle on $X'$ such that $f$ is "transversal" to $Z$, i.e. such that the graph of $f$ is transversal to the cycle $X\times Z'$ in $X\times X'$.
\end{cproof}

The last hypothesis in \Cref{fga1-theorem-1} is used only to be able to conclude that every coherent algebraic sheaf on $X$ is a quotient of a locally-free coherent algebraic sheaf (Serre) and thus admits a left resolution by locally-free sheaves.


\begin{corollary}\label{fga1-theorem-1-corollary-1}
  Let $X$ and $X'$ be non-singular varieties that are locally-closed in a projective space, and suppose that $X'$ is complete.
  Let $U$ be a non-singular cycle on $X\times X'$, and let $a$ and $b$ be points of $X'$ such that $U$ intersects transversally with the cycles $X\times(a)$ and $X\times(b)$.
  Let $Z$ and $Z'$ be non-singular cycles on $X$ such that $Z\times(a)=(X\times(a))\cdot U$ and $Z\times(b)=(X\times(b))\cdot U$.
  Then
  \[P_X(Z) = P_X(Z')\]
\end{corollary}

\begin{cproof}
  Let $f_a\colon X\to X\times X'$ be defined by $f_a(x)=(x,a)$.
  Then, by \Cref{fga1-equation-4.8}, we have $P(Z)=f_a^*(P(U))$;
  similarly, $P(Z')=f_b^*(P(U))$.
  But then, using the Künneth formula
  \[
    \operatorname{H}^\bullet(X\times X',\Omega_{X\times X'}^\bullet)
    = \operatorname{H}^\bullet(X,\Omega_X^\bullet)\otimes\operatorname{H}^\bullet(X',\Omega_{X'}^\bullet)
  \]
  and the fact that $\operatorname{H}^0(X',\Omega_{X'})$ is simply the scalars, we easily see that $f_a^*=f_b^*$, whence the result.
\end{cproof}

For all $x\in X$, $(x)$ is a non-singular subvariety of $X$ of codimension $n$, and thus defines an element $\varepsilon_x$ of $\operatorname{H}^n(X,\Omega_X^n)$.
If $X$ is a non-singular projective variety, then it follows from \Cref{fga1-theorem-1-corollary-1} that $\varepsilon_x$ does not depend on the chosen point $x$, and we thus denote it by $\varepsilon_X$ and call it the \emph{fundamental class} of $\operatorname{H}^n(X,\Omega_X^n)$.

\begin{remark}\label{fga1-4-remark-i}
  To have a satisfying theory, we must define $P_X(Z)$ for arbitrary cycles $Z$, and prove \Cref{fga1-theorem-1} for proper intersections of cycles.
  (At the time of writing this talk, this has still not been done in full generality).
  Assuming that we have done this, \Cref{fga1-theorem-1-corollary-1} becomes the following:
  \emph{if $Z$ and $Z'$ are two algebraically-equivalent cycles, then $P_X(Z)=P_X(Z')$}
  (a claim which does not seem to follow from the above, even if $Z$ and $Z'$ are non-singular).
\end{remark}


\begin{remark}\label{fga1-4-remark-ii}
  \emph{[Comp.]}
  As I pointed out in my conference at the international Congress of Mathematicians in 1958 \footnote{Grothendieck, A. "The cohomology theory of abstract algebraic varieties", in \emph{Proceedings of the international Congress of Mathematicians [1958, Edinburgh]}, Cambridge University Press (1960), 103–118.}, the questions raised here are now completely resolved.

  The reader will find more information on the duality of coherent sheaves in \emph{loc. cit.}, pp.112–115, as well as in [cite{GD1960}, III.2], and in \cite{Gro1960b}.
  A more systematic treatment can be found in a later chapter of \cite{GD1960} (chapter IX in the provisional plan).
\end{remark}
