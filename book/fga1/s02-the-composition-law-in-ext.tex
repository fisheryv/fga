% !TeX root = ../../fga.tex
\section{The composition law in $\operatorname{Ext}$}\label{fga1-2}

The results of this section are due, independently, to Cartier and Yoneda;
see a talk by Cartier \cite{Car1957} for more details.
Let $\mathcal{C}$ be an abelian category, and let $K$ and $L$ be two graded objects of $\mathcal{C}$.
We denote by $\operatorname{Hom}(K,L)$ the graded abelian group whose degree-$n$ component consists of homogeneous homomorphisms of degree $n$ from $K$ to $L$ (i.e. systems $(u_i)$ of homomorphisms $K^i\to L^{i+n}$).
If $K$ and $L$ are complexes (with differentials of degree $+1$, to fix conventions), then we endow $\operatorname{Hom}(K,L)$ with the differential operator given by

\begin{equation}\tag{2.1}\label{fga1-equation-2.1}
  \delta(u) = \mathrm{d}u + (-1)^{n+1}u\mathrm{d} \quad\text{where }n=\deg(u)
\end{equation}

which makes it a complex with a differential of degree $+1$.
The cycles of degree $n$ are the maps of degree $n$ that anticommute with $u$ (as homogeneous maps).
We can then consider $\operatorname{H}^\bullet(\operatorname{Hom}(K,L))$, which is an invariant of the homotopy types of $K$ and $L$, and which we sometimes denote by $\operatorname{H}^\bullet(K,L)$.
If we have a third complex $M$, then the composition of homomorphisms defines a pairing $\operatorname{Hom}(K,L)\times\operatorname{Hom}(L,M)\to\operatorname{Hom}(K,M)$ which is compatible with the differential maps, whence, by passing to the cohomology of pairings,

\begin{equation}\tag{2.2}\label{fga1-equation-2.2}
  \operatorname{H}^\bullet(K,L)\times\operatorname{H}^\bullet(L,M) \to \operatorname{H}^\bullet(K,M)
\end{equation}

which we write as $(u,v)\mapsto vu$.
These pairings satisfy an evident associativity property;
in particular, $\operatorname{H}^\bullet(K,K)$ is an associative graded unital ring, and $\operatorname{H}^\bullet(K,L)$ (resp. $\operatorname{H}^\bullet(L,K)$) is a graded right (resp. left) module over this ring, etc.
In dimension $0$, \Cref{fga1-equation-2.2} reduces to the composition of permissible homomorphisms of complexes.
Finally, an exact sequence of complexes $0\to K'\to K\to K''\to0$ such that, for all $i$, $K'^i$ can be identified with a direct factor of $K^i$, gives rise to an exact sequence of complexes of groups $\operatorname{Hom}(K'',L)$, etc., whence a coboundary map $\operatorname{H}^i(K',L)\to\operatorname{H}^{i+1}(K'',L)$.
We similarly define the boundary maps relative to an exact sequence in $L$.
The pairings in \Cref{fga1-equation-2.2} are compatible, in the usual sense, with these coboundary maps.

Now suppose that $\mathcal{C}$ is a category such that every element $A$ of $\mathcal{C}$ admits an injective resolution $C(A)$.
We then note that, using one of the many variants of the theorem of bicomplexes,
\[\operatorname{H}^\bullet(C(A),C(B)) = \operatorname{H}^\bullet(\operatorname{Hom}(C(A),C(B)))\]
is canonically isomorphic to
\[\operatorname{H}^\bullet(\operatorname{Hom}(A,C(B))) = \operatorname{Ext}^\bullet(A,B)\]
The coboundary maps described above give coboundary maps of the $\operatorname{Ext}$.
Furthermore, the pairings in \Cref{fga1-equation-2.2} give associative pairings here:

\begin{equation}\tag{2.3}\label{fga1-equation-2.3}
  \operatorname{Ext}^\bullet(A,B)\times\operatorname{Ext}^\bullet(B,C) \to \operatorname{Ext}^\bullet(A,C)
\end{equation}


and these are compatible with the coboundary maps.
In particular, $\operatorname{Ext}^\bullet(A,A)$ is an associative graded unital ring, etc.
(We can show in an analogous manner that the $\operatorname{Ext}$ functors behave like derived functors of an arbitrary functor;
we do not make use of this fact here).

In the case where the category in question is the category $\mathcal{C}^\mathcal{O}$ of $\mathcal{O}$-modules on $X$, we then obtain pairings


\begin{equation}\tag{2.4}\label{fga1-equation-2.4}
  \operatorname{Ext}_\mathcal{O}^p(X;\mathcal{A},\mathcal{B})\times\operatorname{Ext}_\mathcal{O}^q(X;\mathcal{B},\mathcal{C}) \to \operatorname{Ext}_\mathcal{O}^{p+q}(X;\mathcal{A},\mathcal{C})
\end{equation}


that can be calculated as already described.
The same method, but replacing the category of abelian groups with the category of abelian sheaves on $X$, and the $\operatorname{Hom}$ functors by the $\mathcal{H}om$ functors, again defines pairings, having the same formal properties, and of a "local nature" this time:

\begin{equation}\tag{2.5}\label{fga1-equation-2.5}
  \mathcal{E}xt_\mathcal{O}^p(\mathcal{A},\mathcal{B})\times\mathcal{E}xt_\mathcal{O}^q(\mathcal{B},\mathcal{C}) \to \mathcal{E}xt_\mathcal{O}^{p+q}(\mathcal{A},\mathcal{C})
\end{equation}


These can be understood by noting that the homomorphisms in \Cref{fga1-equation-1.8} are compatible with the pairings between the $\operatorname{Ext}$.

Finally, recall that we also have a multiplicative structure between functors $\operatorname{H}^p(X,A)$, namely the cup product.
We note then that the spectral sequences of \Cref{fga1-proposition-1} are compatible with the multiplicative structures;
more precisely, we have a pairing from the spectral sequence $E(A,B)$ with the spectral sequence $E(B,C)$ to the spectral sequence $E(A,C)$ that abuts to the pairing between the global $\operatorname{Ext}$, and whose initial page comes from the cup product and the local $\operatorname{Ext}$ pairings in the right-hand side of \Cref{fga1-equation-1.4}.
It then follows, in particular, that the "boundary homomorphisms"


\begin{equation}\tag{2.6}\label{fga1-equation-2.6}
  \operatorname{Ext}_\mathcal{O}^n(X;\mathcal{A},\mathcal{B}) \to \operatorname{H}^0(X;\mathcal{E}xt_\mathcal{O}^n(\mathcal{A},\mathcal{B}))
\end{equation}

\begin{equation}\tag{2.7}\label{fga1-equation-2.7}
  \operatorname{H}^n(X,\mathcal{H}om_\mathcal{O}(\mathcal{A},\mathcal{B})) \to \operatorname{Ext}_\mathcal{O}^n(X;\mathcal{A},\mathcal{B})
\end{equation}


are compatible with the multiplicative structures.
So, if we restrict to sheaves that are locally isomorphic to some $\mathcal{O}^m$, then this completely explains the composition of the global $\operatorname{Ext}$ by means of the cup product, taking into account the isomorphisms of \Cref{fga1-equation-1.5}.
