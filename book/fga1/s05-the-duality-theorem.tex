% !TeX root = ../../fga.tex
\section{The duality theorem}\label{fga1-5}

In this section, $X$ denotes a non-singular projective variety of dimension $n$.

\begin{theorem}\label{fga1-theorem-2}
  The fundamental class $\varepsilon_X$ of $\operatorname{H}^n(X,\Omega_X^n)$ is a basis of this vector space.
\end{theorem}

\begin{cproof}
  The proof of this statement will be given later on.
\end{cproof}

With the above theorem, we can thus identify $\operatorname{H}^n(X,\Omega_X^n)$ with the field $k$.
We now consider the pairings described in \Cref{fga1-2}, which give, in particular, a pairing
\[
  \operatorname{Ext}_{\mathcal{O}_X}^p(X;\mathcal{O}_X,\mathcal{F})\times\operatorname{Ext}_{\mathcal{O}_X}^{n-p}(X;\mathcal{F},\Omega_X^n)
  \to \operatorname{Ext}_{\mathcal{O}_X}^n(X;\mathcal{O}_X,\Omega_X^n)
\]
i.e.

\begin{equation}\tag{5.1}\label{fga1-equation-5.1}
  \operatorname{H}^p(X,\mathcal{F})\times\operatorname{Ext}_{\mathcal{O}_X}^{n-p}(X;\mathcal{F},\Omega_X^n) \to \operatorname{H}^n(X,\Omega_X^n)
\end{equation}

Taking \Cref{fga1-theorem-2} into account, this pairing defines a homomorphism

\begin{equation}\tag{5.2}\label{fga1-equation-5.2}
  \operatorname{Ext}_{\mathcal{O}_X}^{n-p}(X;\mathcal{F},\Omega_X^n) \to (\operatorname{H}^p(X,\mathcal{F}))'
\end{equation}


This homomorphism is functorial in $\mathcal{F}$, and commutes with the coboundary maps relative to the exact sequences $0\to\mathcal{F}'\to\mathcal{F}\to\mathcal{F}''\to0$.


\begin{theorem}\label{fga1-theorem-3}
  The homomorphism in \Cref{fga1-equation-5.2} is an isomorphism.
\end{theorem}

In particular, we recover the following result of Serre:

\begin{corollary}\label{fga1-theorem-3-corollary}
  Let $E$ be an algebraic vector bundle on $X$, and $\mathcal{O}_X(E)$ the sheaf of germs of regular sections of $X$.
  Then we have canonical isomorphisms
  \footnote{This equation is labelled (5.3) in the original copy of the notes, but this seems to be a typo, since a later equation shares the same number, and any references to (5.3) seem to indeed point to the later equation instead of this one.}
  \[(\operatorname{H}^p(X,\mathcal{O}_X(E)))' = \operatorname{H}^{n-p}(X,\Omega_X^n\otimes\mathcal{O}_X(E'))\]
\end{corollary}

\begin{cproof}
It suffices to apply \Cref{fga1-theorem-3} and Corollary 1 of Proposition 1 \Cref{fga1-proposition-1-corollary-1}.
\end{cproof}

\Cref{fga1-theorem-2} and \Cref{fga1-theorem-3} will follow from the following claim:

\begin{statement}\label{fga1-d}
  The homomorphism
  \begin{equation}\tag{5.2 bis}\label{fga1-equation-5.2bis}
    \operatorname{Ext}_{\mathcal{O}_X}^{n-p}(X;\mathcal{F},\Omega_X^n) \to (\operatorname{H}^p(X,\mathcal{F}))'\otimes\mathcal{L}
  \end{equation}

  (where $\mathcal{L}=\operatorname{H}^n(X,\Omega_X^n)$) induced by the pairing in \Cref{fga1-equation-5.1} is an isomorphism.
\end{statement}

We will show that \Cref{fga1-d} implies \Cref{fga1-theorem-2}.
Let $k_x=\mathcal{O}_{(x)}$ be the structure sheaf of the variety consisting of a single point $x\in X$, and consider the canonical homomorphism $\mathcal{O}_X\to k_x$, and the associated homomorphism

\begin{equation}\tag{5.3}\label{fga1-equation-5.3}
  \operatorname{H}^0(X,\mathcal{O}_X) \to \operatorname{H}^0(X,k_x)
\end{equation}

Its transpose can be identified with the homomorphism

\begin{equation}\tag{5.4}\label{fga1-equation-5.4}
  \operatorname{Ext}_{\mathcal{O}_X}^n(X;k_x,\Omega_X^n)\otimes\mathcal{L}' \to \operatorname{Ext}_{\mathcal{O}_X}^n(X;\mathcal{O}_X,\Omega_X^n)\otimes\mathcal{L}'
\end{equation}

induced by the homomorphism between the $\operatorname{Ext}^n$ associated to $\mathcal{O}_X\to k_x$, i.e.

\begin{equation}\tag{5.5}\label{fga1-equation-5.5}
  \operatorname{Ext}_{\mathcal{O}_X}^n(X;k_x,\Omega_X^n) \to \operatorname{Ext}_{\mathcal{O}_X}^n(X;\mathcal{O}_X,\Omega_X^n)
\end{equation}

Since \Cref{fga1-equation-5.3} is an isomorphism, so too is \Cref{fga1-equation-5.4}, and thus so too is \Cref{fga1-equation-5.5}.
Since $s_{(x)}$ is a basis of $\operatorname{Ext}_{\mathcal{O}_X}^n(X;k_x,\Omega_X^n)$ by \Cref{fga1-equation-4.2}, it indeed follows that its image $\varepsilon_X$ is a basis of $\operatorname{H}^n(X,\Omega_X^n)$.

It remains only to prove \Cref{fga1-d}, which will follow in a purely formal way from some elementary facts summarised in the following lemmas.
Here we suppose that $X$ is a closed subset (singular or not) of the projective space $\mathbf{P}$ of dimension $r$.
We use the notation $\mathcal{O}_\mathbf{P}(m)$ to denote the sheaf on $\mathbf{P}$ denoted by $\mathcal{O}(m)$ in \cite{Ser1955}, and the notation $\mathcal{O}_X(m)$ for the analogous sheaf on $X$.


\begin{lemma}\label{fga1-lemma-2}
  The statement of \Cref{fga1-d} is true if $X=\mathbf{P}$ and $\mathcal{F}=\mathcal{O}_\mathbf{P}(m)$.
\end{lemma}

\begin{cproof}
  This lemma can be proved by a direct calculation.
  The explicit calculation of the $\operatorname{H}^i(\mathbf{P},\mathcal{O}_\mathbf{P}(m))$ can be found in \cite{Ser1955}, but it also can be done in a simpler way.
  Computing the cup product $\operatorname{H}^i(\mathbf{P},\mathcal{O}_\mathbf{P}(m))\times\operatorname{H}^j(\mathbf{P},\mathcal{O}_\mathbf{P}(m)') \to \operatorname{H}^{i+j}(\mathbf{P},\mathcal{O}_\mathbf{P}(m+m'))$ (which is necessary to calculate the pairing in \Cref{fga1-equation-5.1}) does not present any difficulty.
\end{cproof}


\begin{lemma}\label{fga1-lemma-3}
  Every coherent algebraic sheaf $\mathcal{F}$ on $X$ is isomorphic to a sheaf that is some quotient of $\mathcal{O}_X(-m)^k$, and we can take $m$ to be as large as we wish.
\end{lemma}

\begin{cproof}
  This follows from the fact that $\mathcal{F}\otimes\mathcal{O}_X(m)$ is "generated by its sections" for large enough $m$;
  see \cite{Ser1955}.
\end{cproof}


\begin{lemma}\label{fga1-lemma-4}
  Let $i>0$.
  Then $\operatorname{H}^{r-i}(\mathbf{P},\mathcal{O}_\mathbf{P}(-m))=0$ for large enough $m$;
  and, for every coherent algebra sheaf $\mathcal{B}$ on $X$, we have that $\operatorname{Ext}_{\mathcal{O}_X}^i(X;\mathcal{O}_X(-m),\mathcal{B})=0$ for large enough $m$.
\end{lemma}

\begin{cproof}
  The first claim follows from the explicit calculations mentioned above;
  for the second, we note that we have an isomorphism
  \[\operatorname{Ext}_{\mathcal{O}_X}^i(X;\mathcal{O}_X(-m),\mathcal{B}) = \operatorname{H}^i(X,\mathcal{B}\otimes\mathcal{O}(m))\]
  (by Corollary 1 of Proposition 1 \Cref{fga1-proposition-1-corollary-1}), whence the conclusion, by a well-known result of \cite{Ser1955}.
\end{cproof}


Combining the previous two lemmas, we find:

\begin{corollary}\label{fga1-lemma-3-and-4-corollary}
  Let $i>0$.
  Then the functor $\mathcal{F}\mapsto\operatorname{H}^{r-i}(\mathbf{P},\mathcal{F})$ on the category of coherent algebraic sheaves on $\mathbf{P}$ is coeffaceable, and so too is the functor $\operatorname{Ext}_{\mathcal{O}_X}^i(X;\mathcal{F},\mathcal{B})$ on the category of coherent algebraic sheaves on $X$.
\end{corollary}


\begin{lemma}\label{fga1-lemma-5}
  Let $\mathcal{A}$ and $\mathcal{B}$ be coherent algebraic sheaves on $X$, and let $\mathcal{A}(m)=\mathcal{A}\otimes\mathcal{O}_X(m)$.
  Then, for large enough $m$, the canonical homomorphism
  \[
  \operatorname{Ext}_{\mathcal{O}_X}^i(X;\mathcal{A}(-m),\mathcal{B})
  \to \operatorname{H}^0(X,\mathcal{E}xt_{\mathcal{O}_X}^i(\mathcal{A}(-m),\mathcal{B}))
  = \operatorname{H}^0(X,\mathcal{E}xt_{\mathcal{O}_X}^i(\mathcal{A},\mathcal{B})(m))
  \]
  is an isomorphism.
\end{lemma}
\begin{cproof}

  This follows immediately from the spectral sequence in \Cref{fga1-proposition-1} applied to $\mathcal{A}(-m)$ and $\mathcal{B}$, since we then have that
  \[
  E_2^{p,q}(\mathcal{A}(-m),\mathcal{B})
  = \operatorname{H}^p(X,\mathcal{E}xt_{\mathcal{O}_X}^q(\mathcal{A}(-m),\mathcal{B}))
  = \operatorname{H}^p(X,\mathcal{E}xt_{\mathcal{O}_X}^q(\mathcal{A},\mathcal{B})(m))
  \]
  which is zero for $p>0$ and large enough $m$.
\end{cproof}

We now prove \Cref{fga1-d} in the case where $X=\mathbf{P}$.
We will first prove that \Cref{fga1-equation-5.2bis} is an isomorphism for $p=n$;
since both sides are then left-exact functors (since $\operatorname{H}^{r+i}(\mathbf{P},\mathcal{F})=0$), it follows from \Cref{fga1-lemma-3} that it suffices to prove the claim for $\mathcal{F}=\mathcal{O}_\mathbf{P}(-m)$, but this is covered by \Cref{fga1-lemma-2}.
Since the homomorphisms in \Cref{fga1-equation-5.2bis} are functorial and compatible with the coboundary maps, and since, for $p<n$, both sides of \Cref{fga1-equation-5.2bis} are coeffaceable functors in $\mathcal{F}$ (the corollary to Lemmas 3 and 4 \Cref{fga1-lemma-3-and-4-corollary}), it follows, by a standard argument, that \Cref{fga1-equation-5.2bis} is an isomorphism for all $p$.
This proves the duality theorem for the projective space.

Now suppose that $X$ is arbitrary, but non-singular.
By the duality theorem for $\mathbf{P}$, we have an isomorphism
\[
  \operatorname{H}^n(X,\mathcal{F})
  = \operatorname{H}^n(\mathbf{P},\mathcal{F})'
  = \operatorname{Ext}_{\mathcal{O}_\mathbf{P}}^{r-n}(\mathbf{P};\mathcal{F},\Omega_\mathbf{P}^r).
\]
By \Cref{fga1-lemma-1}, the far-right-hand side can be identified with
\[
  \operatorname{Hom}_{\mathcal{O}_\mathbf{P}}(\mathbf{P};\mathcal{F},\omega_X^n)
  = \operatorname{Hom}_{\mathcal{O}_X}(X;\mathcal{F},\Omega_X^n)
  = \operatorname{Ext}_{\mathcal{O}_X}^0(X;\mathcal{F},\Omega_X^n)
\]
whence we have an isomorphism

\begin{equation}\label{fga1-equation-5.6}
  \operatorname{H}^n(X,\mathcal{F})' = \operatorname{Hom}_{\mathcal{O}_X}(X;\mathcal{F},\Omega_X^n) = \operatorname{Ext}_{\mathcal{O}_X}^0(X;\mathcal{F},\Omega_X^n)
\end{equation}

Taking $\mathcal{F}=\Omega_X^n$, we obtain an isomorphism

\begin{equation}\label{fga1-equation-5.7}
  \eta\colon \operatorname{H}^n(X,\Omega_X^n) \xrightarrow{\sim} k
\end{equation}


We can prove that the isomorphism in \Cref{fga1-equation-5.6} is exactly \Cref{fga1-equation-5.2bis} with $p=n$ and $\mathcal{L}=k$, by \Cref{fga1-equation-5.7}.
Subsequently, \Cref{fga1-equation-5.2bis} is an isomorphism for $p=n$.
To prove that it is an isomorphism for all $p$, it again suffices to prove that, for $p<n$, the two sides of \Cref{fga1-equation-5.2bis} are coeffaceable functors in $\mathcal{F}$, and, a fortiori (taking \Cref{fga1-lemma-3} into account), that the two sides are zero when we take $\mathcal{F}=\mathcal{O}_X(-m)$ with large enough $m$.
But, for the left-hand side, this is true by \Cref{fga1-lemma-4}, and for the right-hand side we can write, using the duality theorem for $\mathbf{P}$,
\[\operatorname{H}^p(X,\mathcal{O}_X(-m))' = \operatorname{Ext}_{\mathcal{O}_\mathbf{P}}^{r-p}(\mathbf{P};\mathcal{O}_X(-m),\Omega_\mathbf{P}^r)\]
The right-hand side is zero for $p<n$ and large enough $m$, as follows from \Cref{fga1-lemma-5} (where in fact $X=\mathbf{P}$) and from the fact that $\mathcal{O}_X$ is of cohomological dimension $\leqslant r-n$ when thought of as a coherent algebraic sheaf on $\mathbf{P}$ (since $X$ is non-singular), whence
\[\mathcal{E}xt_{\mathcal{O}_\mathbf{P}}^{r-p}(\mathcal{O}_X,\Omega_\mathbf{P}^r) = 0 \quad\text{for }p<n\]
