% !TeX root = ../../fga.tex
\section{Generalisation of the duality theorem}\label{fga1-8}


Let $X$ be a non-singular algebraic variety such that every coherent algebraic sheaf $\mathcal{F}$ on $X$ is isomorphic to a locally-free coherent algebraic sheaf (which is the case if $X$ is locally closed in some projective space).
Then every coherent algebraic sheaf $\mathcal{F}$ on $X$ admits a finite resolution $\mathcal{L}$ by locally-free sheaves, and, for any two such resolutions, we can always find a third, along with homomorphisms, from it to the first two, that are compatible with the augmentations.
Similarly, if $\mathcal{L}$ is a finite locally-free resolution of $\mathcal{F}$, and if we have a homomorphism $\mathcal{F}'\to\mathcal{F}$, then there exists a finite locally-free resolution $\mathcal{L}'$ of $\mathcal{F}'$ along with a homomorphism $\mathcal{L}'\to\mathcal{L}$ that is compatible with $\mathcal{F}'\to\mathcal{F}$, that we can even assume to be surjective if $\mathcal{F}'\to\mathcal{F}$ is surjective.
This allows us to define, given integers $r,s\geqslant0$, two cohomological multifunctors, with arguments $\mathcal{A}_1,\ldots,\mathcal{A}_r;\mathcal{B}_1,\ldots,\mathcal{B}_s$ in the category of coherent algebraic sheaves on $X$;
one takes values in the category of coherent algebraic sheaves on $X$, and the other in the category of modules over $\operatorname{H}^0(X,\mathcal{O}_X)$.
We define them by the formulas

\begin{equation}\tag{8.1}\label{fga1-equation-8.1}
  \begin{aligned}
     & \underline{T}_r^{s\bullet}(\mathcal{A}_1,\ldots,\mathcal{A}_r;\mathcal{B}_1,\ldots,\mathcal{B}_s)
    \\&= \operatorname{H}^\bullet(\mathcal{H}om_{\mathcal{O}_X}(\underline{L}(\mathcal{A}_1)\otimes\ldots\otimes\underline{L}(\mathcal{A}_r), \underline{L}(\mathcal{B}_1)\otimes\ldots\otimes\underline{L}(\mathcal{B}_s))),
    \\&T_r^{s\bullet}(\mathcal{A}_1,\ldots,\mathcal{A}_r;\mathcal{B}_1,\ldots,\mathcal{B}_s)
    \\&= \underline{\operatorname{R}}^\bullet\operatorname{G_a}mma(\mathcal{H}om_{\mathcal{O}_X}(\underline{L}(\mathcal{A}_1)\otimes\ldots\otimes\underline{L}(\mathcal{A}_r), \underline{L}(\mathcal{B}_1)\otimes\ldots\otimes\underline{L}(\mathcal{B}_s)))
  \end{aligned}
\end{equation}

where $\underline{L}(\mathcal{F})$ denotes a finite locally-free resolution of the coherent algebraic sheaf $\mathcal{F}$, and $\underline{\operatorname{R}}^\bullet\operatorname{G_a}mma(\mathcal{K})$ denotes the hypercohomology of the space $X$ with respect to the complex of sheaves $\mathcal{K}$.
If $r$ (resp. $s$) is zero, then we replace the tensor product of the $\underline{L}(\mathcal{A}_i)$ (resp. of the $\underline{L}(\mathcal{B}_j)$) by $\mathcal{O}_X$.
In particular, $\underline{T}_0^0$ and $T_0^0$ are graded functors with no arguments:
$\underline{T}_0^0$ is concentrated in degree $0$, where it is the sheaf $\mathcal{O}_X$;
and $T_0^0$ is equal to $\operatorname{H}^\bullet(X,\mathcal{O}_X)$.
The fact that the right-hand sides of \Cref{fga1-equation-8.1} do not depend on the chosen resolutions is evident for $\underline{T}$ (since the question is then local), and for $T$ it follows from preceding general remarks, taking into account the spectral sequence for the hypercohomology of the complex of sheaves $\mathcal{K}=\mathcal{H}om_{\mathcal{O}_X}(\underline{L}(\mathcal{A}_1)\otimes\ldots,\underline{L}(\mathcal{B}_1)\otimes\ldots)$ that abuts to the hypercohomology of $X$ with respect to $\mathcal{K}$, and whose initial page is $\operatorname{H}^p(X,\operatorname{H}^q(\mathcal{K}))$, i.e.

\begin{equation}\tag{8.2}\label{fga1-equation-8.2}
  E_2^{p,q} = \operatorname{H}^p(X,(\underline{T}_r^s)^{(q)}(\mathcal{A}_1,\ldots,\mathcal{A}_r;\mathcal{B}_1,\ldots,\mathcal{B}_s))
\end{equation}

We then see that this spectral sequence itself does not depend on the chosen resolutions, and its abutment is the left-hand side of \Cref{fga1-equation-8.1}.
We can easily define the coboundary maps relative to miscellaneous arguments $\mathcal{A}_i,\mathcal{B}_j$ by noting that every exact sequence $0\to\mathcal{F}'\to\mathcal{F}\to\mathcal{F}''\to0$ can be resolved by an exact sequence of finite locally-free complexes.

We define, on the system of functors $\underline{T}_r^{s\bullet}$ (resp. $T_r^{s\bullet}$), operations that are analogous to those of tensor calculus, and whose definitions are immediate from the defining formulas in \Cref{fga1-equation-8.1}.
We thus have a \emph{composition} (generalising that described in \Cref{fga1-2}):

\begin{equation}\tag{8.3}\label{fga1-equation-8.3}
  \begin{aligned}
    T_r^{s\bullet}(\mathcal{A}_1,\ldots,\mathcal{A}_r;\mathcal{B}_1,\ldots,\mathcal{B}_s)
    \times T_{r'}^{s'\bullet}(\mathcal{A}'_1,\ldots,\mathcal{A}'_{r'};\mathcal{B}'_1,\ldots,\mathcal{B}'_{s'})
    \\\to T_{r+r'}^{(s+s')\bullet}(\mathcal{A}_1,\ldots,\mathcal{A}_r,\mathcal{A}'_1,\ldots,\mathcal{A}'_{r'};\mathcal{B}_1,\ldots,\mathcal{B}_s,\mathcal{B}'_1,\ldots,\mathcal{B}'_{s'})
  \end{aligned}
\end{equation}

that satisfies the evident properties of associativity, compatibility with the functorial homomorphisms and the coboundary homomorphisms, and spectral sequences.
Similarly, we have symmetry operations, whose explicit descriptions we leave to the reader.
We further have a \emph{contraction} operation every time one of the arguments $\mathcal{A}_i$ is equal to one of the arguments $\mathcal{B}_j$:

\begin{equation}\tag{8.4}\label{fga1-equation-8.4}
  \begin{aligned}
    T_r^{s\bullet}(\mathcal{A}_1,\ldots,\mathcal{A}_{i-1},\mathcal{C},\mathcal{A}_{i+1},\ldots,\mathcal{A}_r;\mathcal{B}_1,\ldots,\mathcal{B}_{j-1},\mathcal{C},\mathcal{B}_{j+1},\ldots,\mathcal{B}_s)
    \\\to -T_{r-1}^{(s-1)\bullet}(\mathcal{A}_1,\ldots,\widehat{\mathcal{A}_i},\ldots,\mathcal{A}_r;\mathcal{B}_1,\ldots,\widehat{\mathcal{B}_j},\ldots,\mathcal{B}_s)
  \end{aligned}
\end{equation}

Furthermore, if some argument $\mathcal{A}_i$ is a locally-free sheaf, then we can suppress it by replacing one of the $\mathcal{A}_j$ (for $j\neq i$) by $\mathcal{A}_j\otimes\mathcal{A}_i$, or one of the $\mathcal{B}_k$ by $\mathcal{B}_k\otimes\mathcal{A}'_i$ (where $\mathcal{A}'_i=\mathcal{H}om_{\mathcal{O}_X}(\mathcal{A}_i,\mathcal{O}_X)$);
we have an analogous rule for the case where one of the arguments $\mathcal{B}_j$ is locally free.
In particular, we can always suppress any argument that is equal to $\mathcal{O}_X$.
If all the arguments are locally free, except for at most one of the arguments $\mathcal{B}_i$, then the rule that we have just stated gives a functorial isomorphism

\begin{equation}\tag{8.5}\label{fga1-equation-8.5}
  T_r^{s\bullet}(\mathcal{A}_1,\ldots,\mathcal{A}_r;\mathcal{B}_1,\ldots,\mathcal{B}_s)
  = \operatorname{H}^\bullet(X,\mathcal{A}'_1\otimes\ldots\otimes\mathcal{A}'_{r}\otimes\mathcal{B}_1\otimes\mathcal{B}_s)
\end{equation}

(since we can restrict to the case where $r=0$ and $s=1$, and there it is immediate;
we can also directly use the spectral sequence whose initial term is \Cref{fga1-equation-8.2}).
The corresponding operations of all the above can also be defined for the $\underline{T}_r^s$.
The relations between the various operations thus introduced are the same as for the analogous relations in tensor calculus.

Now let $n$ be the dimension of $X$.
By successively applying a tensor composition \Cref{fga1-equation-8.3} and contractions \Cref{fga1-equation-8.4} on repeated arguments, we obtain a pairing

\begin{equation}\tag{8.6}\label{fga1-equation-8.6}
  \begin{aligned}
    (T_r^s)^p(\mathcal{A}_1,\ldots;\mathcal{B}_1,\ldots)
    \times (T_r^s)^{n-p}(\mathcal{B}_1,\ldots;\mathcal{A}_1,\ldots,\mathcal{A}_r\otimes\Omega_X^n)
    \\\to\operatorname{H}^n(X,\Omega_X^n)
  \end{aligned}
\end{equation}

\begin{theorem}\label{fga1-theorem-6}
  If $X$ is a non-singular projective variety, then the pairings in \Cref{fga1-equation-8.6} are dualities.
\end{theorem}

\begin{cproof}

  This follows in a purely formal way from the \Cref{fga1-theorem-3-corollary}.
  In fact, it easily follows from this corollary that, if $\mathcal{K}$ is a complex of \emph{locally-free} coherent algebraic sheaves, then the hypercohomology of $X$ with respect to $\mathcal{K}$ is in duality with the hypercohomology of $X$ with respect to $\mathcal{K}'\otimes\Omega_X^n$ via the natural pairings

  \begin{equation}\tag{8.7}\label{fga1-equation-8.7}
    \underline{\operatorname{R}}^p\operatorname{G_a}mma(\mathcal{K})
    \times \underline{\operatorname{R}}^{n-p}\operatorname{G_a}mma(\mathcal{K}'\otimes\Omega_X^n)
    \to \underline{\operatorname{R}}^n\operatorname{G_a}mma(\Omega_X^n)
    = \operatorname{H}^n(X,\Omega_X^n)
  \end{equation}

  We can see this by using the spectral sequence with initial page $\operatorname{H}^p(\operatorname{H}^q(X,\mathcal{K}))$ and the analogous spectral sequence for $\mathcal{K}'\otimes\Omega_X^n$.
  From the above result, \Cref{fga1-theorem-6} can be deduced from the definition \Cref{fga1-equation-8.1}.
\end{cproof}

\begin{remark}\label{fga1-8-remarks}
  ~
  \begin{enumerate}
    \item For the definitions preceding \Cref{fga1-theorem-6}, it was not necessary for $X$ to be non-singular, since it was not necessary to work with only \emph{finite} resolutions.
          But, if $X$ is singular, then we can no long be sure, a priori, that the $(\underline{T}_r^s)^p(\mathcal{A}_1,\ldots;\mathcal{B}_1,\ldots)$ are \emph{coherent} sheaves, since, in the complex of sheaves
          \[\mathcal{H}om_{\mathcal{O}_X}(\underline{L}(\mathcal{A}_1)\otimes\ldots,\underline{L}(\mathcal{B}_1)\otimes\ldots)\]
          there will be an infinite number of components of any given total degree.
    \item We can easily verify that, in the formulas in \Cref{fga1-equation-8.1}, we can replace \emph{one} of the $\underline{L}(\mathcal{B}_i)$ with $\mathcal{B}_i$.
          Taking \Cref{fga1-proposition-3} into account, this shows that we have
          \begin{equation}\tag{8.8}\label{fga1-equation-8.8}
            \begin{aligned}
              \underline{T}_1^{1\bullet}(\mathcal{A};\mathcal{B})
               & = \mathcal{E}xt_{\mathcal{O}_X}^\bullet(\mathcal{A},\mathcal{B})
              \\T_1^{1\bullet}(\mathcal{A};\mathcal{B})
               & = \operatorname{Ext}_{\mathcal{O}_X}^\bullet(X;\mathcal{A},\mathcal{B}).
            \end{aligned}
          \end{equation}
          In particular, taking $r=s=1$ and $\mathcal{A}_1=\mathcal{O}_X$ in \Cref{fga1-equation-8.6}, we recover \Cref{fga1-theorem-3}.
          \Cref{fga1-equation-8.8} also implies that $T_0^{1\bullet}(\mathcal{B})=\operatorname{H}^\bullet(X,\mathcal{B})$, and that $T_1^{0\bullet}(\mathcal{A})=\operatorname{Ext}_{\mathcal{O}_X}^\bullet(X;\mathcal{A},\mathcal{O}_X)$.
    \item We see, in \Cref{fga1-equation-8.1}, that the functors $\underline{T}_r^{s\bullet}$ and $T_r^{s\bullet}$ have, in general, components in positive \emph{and} negative degrees.
          Using the above remark, we see that, if the dimension of $X$ is $n$, then the non-zero components of $\underline{T}_r^{s\bullet}$ are concentrated between degrees $-(s-1)n$ and $rn$ if $s>0$, and between degrees $0$ and $rn$ if $s=0$; the non-zero components of $T_r^{s\bullet}$ are concentrated between degrees $-(s-1)n$ and $(r+1)n$ if $s>0$, and between degrees $0$ and $(r+1)n$ if $s=0$ (and, unless I am mistaken, if $r>0$, between degrees $-(s-1)n$ and $rn$, resp. $0$ and $rn$).
  \end{enumerate}
\end{remark}